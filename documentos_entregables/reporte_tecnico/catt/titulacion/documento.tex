\documentclass{article}

\usepackage{datos}

% Formato de página
\usepackage[
  letterpaper,
  top=2cm,
  left=3cm,
  bottom=2cm,
  right=2cm]{geometry}
\setlength{\parskip}{5em}
\renewcommand{\baselinestretch}{1.5}

% Fondo
\usepackage{draftwatermark}
\SetWatermarkAngle{0.0}

% Tamaños de fuente
\makeatletter
\renewcommand\Huge{%
  \@setfontsize\Huge{20pt}{0pt}
  \abovedisplayskip 10\p@ \@plus2\p@ \@minus5\p@%
  \abovedisplayshortskip \z@ \@plus2\p@%
  \belowdisplayshortskip 5\p@ \@plus2\p@ \@minus3\p@%
  \belowdisplayskip \abovedisplayskip%
  \let\@listi\@listI}
\Huge
\renewcommand\huge{%
  \@setfontsize\huge{16pt}{0pt}
  \abovedisplayskip 10\p@ \@plus2\p@ \@minus5\p@%
  \abovedisplayshortskip \z@ \@plus2\p@%
  \belowdisplayshortskip 5\p@ \@plus2\p@ \@minus3\p@%
  \belowdisplayskip \abovedisplayskip%
  \let\@listi\@listI}
\huge
\renewcommand\Large{%
  \@setfontsize\Large{14pt}{0pt}
  \abovedisplayskip 10\p@ \@plus2\p@ \@minus5\p@%
  \abovedisplayshortskip \z@ \@plus2\p@%
  \belowdisplayshortskip 5\p@ \@plus2\p@ \@minus3\p@%
  \belowdisplayskip \abovedisplayskip%
  \let\@listi\@listI}
\huge
\makeatother

\usepackage{afterpage}
\newcommand\blankpage{%
    \null
    \thispagestyle{empty}%
    \addtocounter{page}{-1}%
    \newpage}

\begin{document}
  \SetWatermarkText{%
    \includegraphics{recursos/fondo_uno.png}}
  \vspace*{-10mm} 
  \begin{center}
    \textbf{{\Huge\instituto}} \\
    \textbf{{\huge\escuela}} \par
    \textbf{{\Huge\acronimoDeEscuela}} \par
    \textit{{\Large Trabajo de titulación}} \\
    \vspace{1.5em}
    \textbf{{\huge ``\titulo''}} \par
    \textit{{\Large Que para cumplir con la opción de titulación curricular \\
    en la carrera de}} \\
    \vspace{1.5em}
    \textbf{{\huge ``Ingeniería en Sistemas Computacionales''}} \par
    \textit{{\Large Presentan}} \\
    \vspace{1.5em}
    \textbf{{\Large Daniel Ayala Zamorano}} \\
    \textbf{{\Large Laura Natalia Borbolla Palacios}} \\
    \textbf{{\Large Ricardo Quezada Figueroa}} \par
    \textit{{\Large Directora}} \\
    \vspace{1.5em}
    \textbf{{\Large Doctora Sandra Díaz Santiago}} \\
  \end{center}
  \newpage
  \setlength{\parskip}{3em}
  \SetWatermarkText{%
    \includegraphics{recursos/fondo_dos.png}}
  \vspace*{-15mm} 
  \begin{center}
    \textbf{{\huge\instituto}} \\
    \textbf{{\huge\escuela}} \\
    \textbf{{\huge SUBDIRECCIÓN ACADÉMICA}} \par
  \end{center}
  \noindent
  Registro de titulación: \registro 
  \hfill Fecha: Junio 2019 \par
  \begin{center}
    {\Large Documento técnico} \\
    \vspace{1.5em}
    \textbf{{\huge ``\titulo''}} \par
    \textit{{\Large Presentan}} \\
    \vspace{1.5em}
    \textbf{{\Large Daniel Ayala Zamorano%
      \footnote{\texttt{daz23ayala@gmail.com}}}} \\
    \textbf{{\Large Laura Natalia Borbolla Palacios%
      \footnote{\texttt{ln.borbolla.42@gmail.com}}}} \\
    \textbf{{\Large Ricardo Quezada Figueroa%
      \footnote{\texttt{qf7.ricardo@gmail.com}}}} \par
    \textit{{\Large Directora}} \\
    \vspace{1.5em}
    \textbf{{\Large Doctora Sandra Díaz Santiago}} \par
    \textbf{{\Large RESUMEN}}
  \end{center}
  \vspace{-3em}
  Hoy en día, es bien conocido que los datos de tarjetas bancarias son datos 
  sensibles y, por tanto, garantizar su privacidad, es de suma importancia. 
  Una solución que se ha vuelto muy popular es sustituir el dato sensible por 
  un token, es decir, un valor que no revela la información original. 
  Lamentablemente, la mayor parte de las soluciones que existen en el mercado 
  son muy poco claras sobre sus métodos de generación de tokens, ya que, al no 
  existir un estudio criptográfico formal, no hay ninguna certeza sobre la 
  seguridad de sus métodos. En este trabajo se analizarán e implementarán 
  diversos algoritmos para la generación de tokens. \\
  Palabras clave: criptografía, seguridad, tecnologías web, tarjetas bancarias.
\end{document}
