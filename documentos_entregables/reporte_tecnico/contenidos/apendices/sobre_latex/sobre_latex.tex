%
% Apéndice sobre el documento.
% Reporte técnico.
% Proyecto Lovelace.
%

\capitulo[TEX]{Sobre este documento}{sec:sobre_este_documento}
{
  \epigrafe
  {%
    The trouble with computer science today is an obsessive concern with
    form instead of content.%
  }
  {%
    Form and Content in Computer Science, 1970 \acrshort{gl:acm} Turing Lecture,
    \textsc{Marvin Minsky}.%
  }
}

\noindent
En este apéndice se discuten algunos de los aspectos más relevantes de la
elaboración de este documento. Primero se presentan algunas de las dependencias
más importantes de \LaTeX; después se presenta una lista de los comandos hechos
específicamente para este reporte y se muestran, a modo de ejemplo, algunos de
ellos; por último se presentan algunos de los problemas surgidos durante la
elaboración junto con su respectiva solución.

Primero, ¿por qué \LaTeX? Una de las principales ventajas con las que se
promociona a \LaTeX~es la separación de responsabilidades: el autor de un
documento solamente debe de procuparse por el contenido, mientras que el
programa, \LaTeX, es el responsable de la forma, esto es, de la apariencia
final del documento~\cite{introduccion_latex}. En el caso de este trabajo lo
anterior no es enteramente cierto, dado que hay un alto grado de modificaciones
sobre los comportamientos por defecto (de lo contrario no existiría este
apéndice). Lo que sí es cierto es que el uso de \LaTeX~implica una separación
base entre la forma y el contenido de una trabajo y, siendo cuidadosos, esta
separación se puede mantener aún cuando se modifique el comportamiento por
defecto.

La segunda razón de peso tiene que ver con las circunstancias del producto
requerido: se trata de un documento muy extenso que requiere modificaciones
simultáneas de todo el equipo; en particular, se necesita que las
modificaciones al documento estén adentro del seguimiento del control de
versiones (dependencia \hipervinculo{dep:git}). A diferencia de un procesador
de textos \gls{gl:wysiwyg}, \LaTeX~trabaja con archivos de texto plano, lo cual
permite mantener el control de versiones tan claro como se quiera. El punto
anterior también permite importar buenas prácticas de programación en la
elaboración del documento: la modularización del código en archivos y carpetas
significativos (este documento está distribuido en alrededor de~500 archivos,
la estructura de los archivos emula a la estructura del documento); líneas
cortas (80~caracteres en este caso) para facilitar la lectura directo en
archivos fuente y para facilitar la comparación de versiones.

% Qué dramático:
La inclusión de este apéndice es una forma de reconocimiento hacia el tiempo y
esfuerzo dedicados para que este documento llegara a ser lo que es.

\section{Dependencias}

El documento depende actualmente de 138 paquetes de \LaTeX; aquí se enlistan
aquellos cuyo uso responde a un proceso de decisión, y no a una simple
necesidad.

\dependencia{Elaboración de diagramas}{Ti\textit{k}Z}{\acrshort{gl:lppl} v1.3}{%
  https://www.ctan.org/pkg/pgf}{%
  Existen varias ventajas al utilizar Ti\textit{k}Z. Las dos principales son
  las mismas que las del propio \LaTeX: al trabajar con texto plano, los
  diagramas se encuentran dentro del seguimiento del control de versiones; con
  un poco de cuidado, se puede mantener separada a la forma del contenido, lo
  cuál permite crear diagramas con exactamente el mismo aspecto. Otra ventaja
  es la integración entre los diagramas y el propio documento: Ti\textit{k}Z
  produce comandos en \gls{gl:pgf} que se encuentran incrustados directamente
  en el \gls{gl:pdf} producto, lo cual implica que el texto de los diagramas es
  seleccionable (en algunos diagramas incluso hay hipervínculos, por ejemplo,
  el diagrama de estados~\ref{estados_actores}); la tipografía de la fuente es
  la misma para diagramas y documento. La única desventaja está dada por el
  tiempo de aprendizaje de la sintaxis, que en comparación con un editor
  gráfico, es algo mucho más lento. A modo de ejemplo, el código
  fuente~\ref{codigo:diagrama_tikz} muestra un fragmento del archivo fuente del
  diagrama~\ref{diagrama:aes_add_round_key}}.

% TODO:
% La ruta a este archivo está demasiado larga.

\codigoFuente[codigo:diagrama_tikz]{1}{38}{latex}{%
  documentos_entregables/reporte_tecnico/contenidos/marco_teorico/%
  bloques/diagramas/addRoundKey.tikz.tex}{%
  Ejemplo de diagrama en Ti\textit{k}Z;
  diagrama~\ref{diagrama:aes_add_round_key}}

\section{Comandos propios}

\section{Otras configuraciones}
