%
% Explicación de FFX en página estática de inicio.
% Reporte técnico y aplicación web de sistema tokenizador.
% Proyecto Lovelace.
%

Este es un cifrado que permite cifrar cadenas de cualquier
tamaño, compuestas por cualquier tipo de caracteres, el cual es
usado para tokenizar. Fue publicado por Mihir Bellare, Phillip
Rogaway y Terence Spies en 2009, y, en 2016, el National
Institute of Standards and Technology (NIST) le dio el nombre de
FF1 a este algoritmo.

De forma general, el algoritmo usa redes Feistel junto con
primitivas criptográficas (funciones hash o cifrados por bloques)
adaptadas en la función de ronda de la red para lograr preservar
el formato del dato dado para tokenizar; esto significa que el
token tendrá la misma longitud y se mantendrá en el mismo
dominio que el dato original.

Para conocer más sobre este algoritmo revise el siguiente
artículo:
\href{http://citeseerx.ist.psu.edu/viewdoc/download?doi=10.1.1.304.1736&rep=rep1&type=pdf}
{The FFX Mode of Operation for Format-Preserving Encryption}.
