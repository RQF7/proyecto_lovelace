%
% Capítulo de introducción.
% Proyecto Lovelace.
%

\chapter{Introducción}

% Laura
% Historia del comercio línea

% Fraudes

% Importancia de mantener segura la información

% Hablar sobre la criptografía

% PCI SSC y estándares (DSS)

% Paradigma de tokens

\section{Objetivos}

\subsection{Objetivo general}

Implementar algoritmos criptográficos y no criptográficos para generar
\glspl{gl:token} con el propósito de proveer confidencialidad a los datos de las
tarjetas bancarias\footnote{Tomando en cuenta la clasificación propuesta por el
\gls{gl:pci} \gls{gl:ssc}}.

\subsection{Objetivos específicos}

\begin{itemize}
  \item Revisar diversos algoritmos para la generación de \glspl{gl:token}.
  \item Diseñar e implementar un servicio web que proporcione el servicio de
    generación de \glspl{gl:token} de tarjetas bancarias a, al menos, una tienda
    en línea.
  \item Implementar una tienda web que use el servicio de generación de
    \glspl{gl:token}.
\end{itemize}

\section{Justificación}

%% TODO: Explicar a grandes rasgos, tienda por tienda, el método que proponen.

En el estándar del PCI DSS se explica muy claramente lo que debe de cumplir un
sistema tokenizado, sim embargo, no se explica cómo debe de hacerlo. Esto ha
dado pie a que surjan soluciones que, a pesar de cumplir con el estándar, no
son claros con respecto a su funcionamiento. Se hizo una investigación
comparativa entre las principales soluciones que existen en el mercado.

% Shift4
% Ricardo

% Merchant Link Payment Gateway,
% Laura

% Bluepay ToknShield,
% Daniel

% Braintree,
% Ricardo

% Jack Henry Card Processing Solution,
% Laura

% Thales Tokenization.
% Daniel

%TODO: citar ejemplos de las confusiones por algunas de las publicaciones:
% ¿Qué es mejor criptografía vs. tokenización? Hay quién dice que no debes
% utilizar una llave, porque es un modo de agregar reversibilidad (insuguro).
% Ricardo

% Conclusión
% Ricardo

\section{Organización del documento}
% Daniel


