%
% Sección de BPS. Capítulo de análisis y diseño de generación de tokens
% Proyecto Lovelace.
%

\subsubsection{Algoritmo \textit{BPS}}

La información que aquí se presenta puede ser consultada con mayor detalle en
\cite{bps}.

\textit{BPS} es uno de los algoritmos de cifrado que preservan el formato
existentes, y es capaz de cifrar cadenas de longitudes casi arbitrarias que
estén formadas por cualquier conjunto de caracteres.

\textit{BPS} está conformado por 2 partes fundamentales, un cifrado interno
$BC$, encargado de cifrar bloques de longitud fija; y un modo de operación,
encargado de extender la funcionalidad de el cifrador $BC$ y permitir que
\textit{BPS} cifre cadenas de varias longitudes.

%%%%%%%%%%%%%%%%%%%%%%%%%%%%%%%%%%%%%%%%%%%%%%%%%%%%%%%%%%%%%%%%%%%%%%%%%%%%%%%

\paragraph{El cifrado interno $BC$}

%------------------------------------------------------------------------------

El cifrado por bloques que usa \textit{BPS} internamente se define como
\begin{equation}
  BC_{F,s,b,w}(X,K,T)
\end{equation}

Donde:
\begin{itemize}
  \item $F$ es un cifrador por bloques de $f$ bits de salida,
    por ejemplo: \gls{gl:tdes}, \gls{gl:aes}, \gls{gl:sha}-2.
  \item $s$ es la cardinalidad del conjunto de caracteres del bloque a cifrar.
  \item $b$ es la longitud del bloque a cifrar,
    cumpliendo con $b \leq 2 \cdot |log_s(2^{f-32})|$.
  \item $w$ es el número (par) de rondas de la red Feistel interna
    (véase \ref{sec:red_feistel}).
  \item $X$ es la cadena o bloque de longitud $b$ a cifrar.
  \item $K$ es una llave acorde al cifrador por bloques $F$.
  \item $T$ es un \textit{tweak} de 64 bits.
\end{itemize}

%------------------------------------------------------------------------------

\textbf{Proceso de cifrado BC}.

Para poder cifrar la cadena $X$:

\begin{enumerate}

  \item Se tiene que dividir el \textit{tweak} $T$ de 64 bits en 2
    sub\textit{tweak}s $T_L$ y $T_R$ de 32 bits. Viendo a $T$ como un número
    entero codificado en binario se puede calcular $T_R\: =\: T\: \mod\: 2^{32}$
    y $T_L\: =\: (T\: -\: T_R) / 2^{32}$

  \item Igualmente, se tiene que dividir en 2 la cadena $X$ para obtener las
    subcadenas $X_L$ y $X_R$ con una longitud $l$ y $r$ respectivamente.
    Dado que la longitud $b$ de la cadena no siempre es par, se tiene que, si
    $b$ es par, entonces tanto $l$ como $r$ son igual a $b/2$, pero en caso
    de que $b$ sea impar, $l$ va a ser igual a $(b+1)/2$ y $r$ igual a
    $(b-1)/2$.

  \item Partiendo de que el cifrador $BC$ se compone de $w$ rondas de una red
    Feistel, se define $L_i$ y $R_i$ (parte izquierda y parte derecha de la
    red en la i-ésima ronda), y se inicializan en:
    \begin{align}
      L_0\: &=\: \sum_{j=0}^{l-1} X_L[j] \cdot s^j \\
      R_0\: &=\: \sum_{j=0}^{r-1} X_R[j] \cdot s^j
    \end{align}

  \item Ahora por cada ronda $i\: <\: w$ y cifrando con el cifrador por
    bloques $E$.

    Si $i$ es par:
    \begin{align}
      L_{i+1}\: &=\: L_i\: \boxplus\:
                    E_K((T_R \oplus i) \cdot 2^{f-32}\: +\: R_i)\qquad
                    (mod\ s^l) \\
      R_{i+1}\: &=\: R_i
    \end{align}

    Si $i$ es impar:
    \begin{align}
      R_{i+1}\: &=\: R_i\: \boxplus\:
                    E_K((T_L \oplus i) \cdot 2^{f-32}\: +\: L_i)\qquad
                    (mod\ s^r) \\
      L_{i+1}\: &=\: L_i
    \end{align}

  \item Por último se tiene que descomponer tanto a $L_w$ como a $R_w$ para
    obtener a $Y_L$ y a $Y_R$ respectivamente, las cuales concatenadas
    ($Y_L \parallel Y_R$) dan la cadena de salida $Y$.

    El proceso para hacer la descomposición se muestra en el pseudocódigo
    \ref{descomposicion_Lw_Rw}.

    \begin{pseudocodigo}[caption={Proceso de descomposición de $L_w$ o $R_w$.},
    label={descomposicion_Lw_Rw}]
      entrada:   bloque $N_w$ de longitud $n$.
      salida:    bloque $Y_N$
      inicio
        para $i=0$ hasta $n-1$
          $Y_N[i] = N_w\ mod\ s$
          $N_w = (N_w - Y_N[i])/s$
      fin
    \end{pseudocodigo}

\end{enumerate}

De forma general, el proceso de cifrado se describe en el pseudocódigo
\ref{cifrado_BC}.

%Pseudocodigo excediéndose de los 80 para una correcta visualización el en pdf ---------------------------
\begin{pseudocodigo}[caption={Proceso de cifrado $BC$.},
label={cifrado_BC}]
  entrada:    la llave $K$, el tweak $T$, la cadena $X$ de longitud $b$ formada por el conjunto $S$
              de cardinalidad $s$, la función de cifrado $F$, y el número de rondas $w$.
  salida:     La cadena cifrada $Y$.
  inicio
    calcular $T_R\: =\: T\: \mod\: 2^{32}$ y $T_L\: =\: (T\: -\: T_R) / 2^{32}$
    asignar $l = r = b/2$
    inicializar $L_0\: =\: \sum_{j=0}^{l-1}\: X[j] \cdot s^j$
    inicializar $R_0\: =\: \sum_{j=0}^{r-1}\: X[j+l] \cdot s^j$
    para $i=0$ hasta $i=w-1$
      si $i$ es par
        $L_{i+1}\: =\: L_i\: \boxplus\: F_K((T_R \oplus i) \cdot 2^{f-32}\: +\: R_i)\qquad (mod\ s^l)$
        $R_{i+1}\: =\: R_i$
      si $i$ es impar
        $R_{i+1}\: =\: R_i\: \boxplus\: F_K((T_L \oplus i) \cdot 2^{f-32}\: +\: L_i)\qquad (mod\ s^r)$
        $L_{i+1}\: =\: L_i$
    para $i=0$ hasta $i=l-1$
      $Y_L[i] = L_w\ mod\ s$
      $L_w = (L_w - Y_L[i])/s$
    para $i=l$ hasta $i=r-1$
      $Y_R[i] = R_w\ mod\ s$
      $R_w = (R_w - Y_R[i])/s$
    determinar $Y = Y_L \parallel Y_R$
  fin
\end{pseudocodigo}
%Pseudocodigo excediéndose de los 80 para una correcta visualización el en pdf ---------------------------

%------------------------------------------------------------------------------

\textbf{Proceso de descifrado $BC^{-1}$.}

Para poder descifrar la cadena $Y$:

\begin{enumerate}

  \item Se tiene que dividir en 2 la cadena $Y$, para obtener las subcadenas
    $Y_L$ y $Y_R$ con una longitud $l$ y $r$ respectivamente, de igual forma
    que se hizo con la cadena $X$ en el proceso de cifrado.

  \item Partiendo de que el proceso de descifrado se compone de $w$ rondas,
    se define $L_i$ y $R_i$ y se inicializan en:
    \begin{align}
      L_w\: &=\: \sum_{j=0}^{l-1} Y_L[j] \cdot s^j \\
      R_w\: &=\: \sum_{j=0}^{r-1} Y_R[j] \cdot s^j
    \end{align}

  \item Ahora, comenzando con $i=w-1$, para cada ronda $i\: \geq\: 0$.

    Si $i$ es par:
    \begin{align}
      L_i\: &=\: L_{i+1}\: \boxminus\:
                E_K((T_R \oplus i) \cdot 2^{f-32}\: +\: R_{i+1})\qquad
                (mod\ s^l) \\
      R_i\: &=\: R_{i+1}
    \end{align}

    Si $i$ es impar:
    \begin{align}
      R_i\: &=\: R_{i+1}\: \boxminus\:
                E_K((T_L \oplus i) \cdot 2^{f-32}\: +\: L_{i+1})\qquad
                (mod\ s^r) \\
      L_i\: &=\: L_{i+1}
    \end{align}

  \item Finalmente se tienen que descomponer $L_0$ y $R_0$ (con el mismo
    proceso de descomposición descrito en el cifrado) para obtener a $X_L$ y
    $X_R$, las cuales concatenadas ($X_L \parallel X_R$) dan la cadena de
    salida $X$.

\end{enumerate}

De forma general, el proceso de descifrado se describe en el pseudocódigo
\ref{descifrado_BC}.

%Pseudocodigo excediéndose de los 80 para una correcta visualización el en pdf ---------------------------
\begin{pseudocodigo}[caption={Proceso de descifrado $BC^{-1}$.},
label={descifrado_BC}]
  entrada:    la llave $K$, el tweak $T$, la cadena $Y$ de longitud $b$ formada por el conjunto $S$
              de cardinalidad $s$, la función de cifrado $F$, y el número de rondas $w$.
  salida:     La cadena $X$.
  inicio
    calcular $T_R\: =\: T\: \mod\: 2^{32}$ y $T_L\: =\: (T\: -\: T_R) / 2^{32}$
    asignar $l = r = b/2$
    inicializar $L_w\: =\: \sum_{j=0}^{l-1}\: Y[j] \cdot s^j$
    inicializar $R_w\: =\: \sum_{j=0}^{r-1}\: Y[j+l] \cdot s^j$
    para $i=w-1$ hasta $i=0$
    si $i$ es par:
      $L_i\: =\: L_{i+1}\: \boxminus\: F_K((T_R \oplus i) \cdot 2^{f-32}\: +\: R_{i+1})\qquad (mod\ s^l)$
      $R_i\: =\: R_{i+1}$
    si $i$ es impar:
      $R_i\: =\: R_{i+1}\: \boxminus\: F_K((T_L \oplus i) \cdot 2^{f-32}\: +\: L_{i+1})\qquad (mod\ s^r)$
      $L_i\: =\: L_{i+1}$
    para $i=0$ hasta $i=l-1$
      $X_L[i] = L_w\ mod\ s$
      $L_w = (L_w - X_L[i])/s$
    para $i=l$ hasta $i=r-1$
      $X_R[i] = R_w\ mod\ s$
      $R_w = (R_w - X_R[i])/s$
    determinar $X = X_L \parallel X_R$
  fin
\end{pseudocodigo}
%Pseudocodigo excediéndose de los 80 para una correcta visualización el en pdf ---------------------------

%%%%%%%%%%%%%%%%%%%%%%%%%%%%%%%%%%%%%%%%%%%%%%%%%%%%%%%%%%%%%%%%%%%%%%%%%%%%%%%

\paragraph{El modo de operación}

En cuanto al modo de operación de \textit{BPS}, se puede decir que es un
equivalente al modo de operación CBC (véase \ref{sec:cbc}), ya que el bloque
$BC_n$ utiliza el texto cifrado de la salida del bloque $BC_{n-1}$, con la
distinción de que en lugar de aplicar operaciones \textit{xor} usa sumas
modulares carácter por carácter, y de que no utiliza un
\gls{gl:vector_de_inicializacion}, a pesar de soportar su uso.

Algo importante a resaltar de este modo de operación es que, en caso de que el
texto en claro no tenga una longitud total que sea múltiplo de la longitud de
bloque $b$, al cifrar el último bloque se recorre el cursor que determina
el inicio del mismo, hasta que su longitud concuerde con $b$, esto se puede
ver de forma gráfica en la figura \ref{cursor_BPS}.

\begin{figure}[H]
  \begin{center}
    \includegraphics[width=0.8\linewidth]
    {../../../../../diagramas_comunes/bps/cursor_bps}
    \caption{Ejemplo del corrimiento de cursor para la selección del ultimo
      bloque en el modo de operación de \textit{BPS}.}
    \label{cursor_BPS}
   \end{center}
\end{figure}

En la figura \ref{modo_de_operacion_BPS} se ve de manera gráfica el
funcionamiento del modo de operación de \textit{BPS}.

\begin{figure}[H]
  \begin{center}
    \includegraphics[width=0.85\linewidth]
    {../../../../../diagramas_comunes/bps/modo_de_operacion_bps}
    \caption{Modo de operación de \textit{BPS}.}
    \label{modo_de_operacion_BPS}
   \end{center}
\end{figure}

Otra particularidad del modo de operación es el uso del contador $u$ de 16
bits, que es utilizado para aplicar una operación \textit{xor} al
\textit{tweak} $T$ que entra a cada uno de los bloques $BC$. Recordando que $T$
es de 64 bits, el \textit{xor} se aplica a los 16 bits más significativos de
ambas mitades de \textit{tweak}, esto debido a cada mitad de \textit{tweak}
funciona de manera independiente en el cifrador $BC$, y a que no se desea un
traslape entre el contador externo $u$ y el contador $i$ interno en $BC$.

%%%%%%%%%%%%%%%%%%%%%%%%%%%%%%%%%%%%%%%%%%%%%%%%%%%%%%%%%%%%%%%%%%%%%%%%%%%%%%%

\paragraph{Características generales}

Como se observó, \textit{BPS} está basado en las redes Feistel, lo cual puede
verse como una ventaja, debido al amplio estudio que tienen. Además, usa
algoritmos de cifrado o funciones hash estandarizadas de forma interna, lo
cual hace más comprensible y fácil su implementación.

\textit{BPS} es un cifrado que preserva el formato capaz de cifrar cadenas de
un longitud de $2$ hasta $max(b) \cdot 2^{b}$ caracteres (donde $max(b)$ es el
tamaño máximo de bloque), formadas por cualquier conjunto.

Se puede considerar que \textit{BPS} es eficiente, debido a que la llave $K$
usada en cada bloque $BC$ es constante, y a que además usa un número reducido
de operaciones internas en comparación con otros algoritmos de cifrado que
preservan el formato.

Por último, se puede resaltar que el uso de \textit{tweaks} protege a
\textit{BPS} de ataques de diccionario, los cuales son fáciles de cometer cuando
el dominio de la cadena a cifrar es muy pequeño.

%%%%%%%%%%%%%%%%%%%%%%%%%%%%%%%%%%%%%%%%%%%%%%%%%%%%%%%%%%%%%%%%%%%%%%%%%%%%%%%

\paragraph{Recomendaciones}

Se recomienda que el número de rondas $w$ de la red Feistel sea $8$, dado
que es una número de rondas eficiente, y se ha estudiado la seguridad de
\textit{BPS} con este $w$.

Es recomendable que como \textit{tweak} se use la salida truncada de una función
hash, en donde la entrada de la función puede ser cualquier información
relacionada a los datos que se deseen proteger, como por ejemplo fechas, lugares,
o parte de los datos que no se deseen cifrar.
