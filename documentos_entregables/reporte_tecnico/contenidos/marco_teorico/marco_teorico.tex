%
% Capítulo de antecedentes.
% Proyecto Lovelace.
%

\capitulo{Marco teórico}{sec:marco_teorico}
{
  \epigrafe
  {%
    Human beings, who are almost unique in having the ability to learn from the
    experience of others, are also remarkable for their apparent disinclination
    to do so.%
  }
  {%
     \textsc{Douglas Adams}.%
  }
  \epigrafe
  {%
    Desvarío laborioso y empobrecedor el de componer vastos libros; el de
    explayar en quinientas páginas una idea cuya perfecta exposición oral cabe
    en pocos minutos. Mejor procedimiento es simular que esos libros ya existen
    y ofrecer un resumen, un comentario.%
  }
  {%
     \textsc{Jorge Luis Borges}.%
  }
}

\noindent
El marco teórico contiene la mayoría de los conceptos utilizados para la
implementación de los algoritmos generadores de \glspl{gl:token}. Comienza con
la introducción a la criptografía, sus objetivos, ataques y tipos.
Posteriormente, se tiene un resumen de los cifrados por bloque, haciendo énfasis
en el funcionamiento de las redes Feistel y el cifrador \gls{gl:aes} y los modos
de operación \gls{gl:ecb}, \gls{gl:cbc} y \gls{gl:ctr}; después, se agregan unas
nociones básicas sobre cifradores por flujos, funciones hash, códigos
\gls{gl:mac} y \gls{gl:tes}. El capítulo finaliza con los cifrados que preservan
el formato. Cada sección indica al inicio las fuentes bibliográficas que fueron
consultadas en caso de que el lector desee profundizar en algún tema tratado.

\subimport{intro/}{intro}
\subimport{bloques/}{bloques}
%\subimport{cifrados_de_flujo/}{cifrados_de_flujo}
\subimport{hash/}{hash}
\subimport{mac/}{mac}
\subimport{tes/}{tes}
\subimport{fpe/}{fpe}
