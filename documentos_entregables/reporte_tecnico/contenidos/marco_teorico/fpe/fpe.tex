%
% Cifrados que preservan el formato, sección de antecedentes, reporte técnico.
% Proyecto Lovelace.
%

\section{Cifrados que preservan el formato}
\label{sec:fpe}

El objetivo de los cifrados que preservan el formato (en inglés, \gls{gl:fpe})
está bastante bien definido: lograr que los mensajes cifrados \textit{se vean}
igual que los mensajes en claro. Obviamente para que esto tenga sentido hay que
definir qué es lo que hace que un mensaje se parezca a otro; los cifradores
tradicionales, como \gls{gl:aes}, reciben cadenas binarias y entregan cadenas
binarias, por lo que, en cierto sentido, son ya un tipo de cifrados
que preservan el formato. En una clase más general de algoritmos \gls{gl:fpe}
el formato debe ser una parámetro: cadenas binarias, caracteres \gls{gl:ascii}
imprimibles, dígitos decimales, dígitos hexadecimales, etcétera.

Desde un inicio, los cifrados que preservan el formato se perfilaron como
una posible solución para el problema de la tokenización: usando un alfabeto
de dígitos decimales se logra que los \textit{tokens} y los \gls{gl:pan} tengan
el mismo aspecto. De las posibles soluciones presentadas en la sección
\ref{sec:algoritmos}, \gls{gl:fpe} es la que presenta un esquema más tradicional
dentro de la criptografía, ya que el proceso de cifrado y descifrado es el mismo
que un cifrador simétrico.

La utilidad de los cifrados que preservan el formato se centra principalmente
en \textit{agregar} seguridad a sistemas y protocolos que ya se encuentran
en un entorno de producción. Estos son algunos ejemplos de dominios
comunes en \gls{gl:fpe}:

\begin{itemize}

  \item Números de tarjetas de crédito. \\
    \texttt{5827 5423 6584 2154} $ \rightarrow $ \texttt{6512 8417 6398 7423}

  \item Números de teléfono. \\
    \texttt{55 55 54 75 65} $ \rightarrow $ \texttt{55 55 12 36 98}

  \item CURP. \\
    \texttt{GHUJ887565HGBTOK01} $ \rightarrow $ \texttt{QRGH874528JUHY01}

\end{itemize}

\subsection{Clasificación de los cifrados que preservan el formato}

En \cite{sinopsis_rogaway} Phillip Rogaway propone una clasificación para los
cifrados que preservan el formato que se basa en el tamaño del espacio de
mensajes ($ N = |X| $):

\begin{description}

  \item[Espacios minúsculos]

    El espacio es tan pequeño que es aceptable gastar $ O(N) $ en un preproceso
    de cifrado. Esto es, cifrar todos los posibles mensajes de una sola vez,
    para que las subsiguientes solicitudes de cifrado y descifrado consistan en
    simples consultas a una base de datos.

    El tamaño de $ N $ depende del contexto en el que se vaya a utilizar el
    algoritmo; para el contexto del problema de la tokenización
    ($ N \approx 10^{16} $) no resulta viable utilizar esta técnica.

    Algunos ejemplos de cómo hacer el preproceso de cifrado son el
    \textit{Knuth shuffle} (también conocido como \textit{Fisher-Yates shuffle},
    pseudocódigo \ref{knuth_shuffle}) o un \gls{gl:cifrado_con_prefijo}.

  \item[Espacios pequeños]

    En esta clasificación se colocan a los espacios de mensaje cuyo tamaño
    no es más grande que $ 2^w $ en donde $ w $ es el tamaño de bloque del
    cifrado subyacente. Para AES, en donde $ w = 128 $, $ N = 2^{128}
    \approx 10^{38} $.

    En este esquema, el mensaje se ve como una cadena de $ n $ elementos
    pertenecientes a un alfabeto de cardinalidad $ m $ (i. e. $ N = m^n $).
    Por ejemplo, para números de tarjetas de crédito, $ n \approx 16 $ y
    $ m = 10 $, por lo que $ N = 10^{16} $ (diez mil trillones); lo cual es
    aproximadamente $ 2.93 \times 10^{-21} \% $ de $ 2^{128} $.

    Los algoritmos que preservan el formato expuestos en la sección
    \ref{sec:algoritmos} pertenencen a esta categoría.

  \item[Espacios grandes]

    El espacio es más grande que $ 2^w $. Para estos casos, el mensaje se ve
    como una cadena binaria. Las técnicas utilizadas incluyen cualquier cifrado
    cuya salida sea de la misma longitud que la entrada (p. ej. los
    TES: CMC, EME, HCH, etc. Sección \ref{sec:tes}).

\end{description}

Como se puede observar de los ejemplos dados, el problema de la tokenización
de números de tarjetas de crédito es un problema de espacios pequeños.

\begin{pseudocodigo}[%
  caption={\textit{Knuth shuffle}, \cite{DBLP:books/aw/Knuth69}.},
  label={knuth_shuffle}%
]
    entrada: lista $ l[0, \dots, n - 1] $
    salida:  misma lista barajeada
    inicio
      para_todo i desde $ n - 1 $ hasta 0:
        j $ \gets $ rand($ 0 $, $ i $)
        swap($ l[j] $, $ l[i] $)
      fin
    fin
\end{pseudocodigo}
