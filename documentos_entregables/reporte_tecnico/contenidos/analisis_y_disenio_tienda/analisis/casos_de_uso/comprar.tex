%
% Caso de uso para realizar una compra,
% análisis y diseño de tienda en línea.
% Proyecto Lovelace.
%

\casoDeUso[lib_cu:comprar]
{Comprar}
{
  Caso que permite que un \textbf{cliente} realice la compra de los libros que
  hay en el carrito.

  \begin{trayectoriaPrincipal}

    \item El cliente presiona el botón \textit{Finalizar compra} en la
      interfaz \hipervinculo{lib_iu:finalizar_compra}.

    \item El sistema obtiene las formas de pago del cliente en sesión.

    \item El sistema muestra la interfaz
      \hipervinculo{lib_iu:seleccionar_forma_de_pago}.

    \item El cliente selecciona la forma de pago deseada;
      [\hipervinculoLocal{ta:cancelar}].

    \item El cliente presiona el botón \textit{Continuar};
      [\hipervinculoLocal{ta:cancelar}].

    \item El sistema muestra la interfaz
      \hipervinculo{lib_iu:seleccionar_direccion_de_entrega}.

    \item El cliente selecciona la dirección de entrega deseada;
      [\hipervinculoLocal{ta:cancelar}].

    \item El cliente presiona el botón \textit{Continuar};
      [\hipervinculoLocal{ta:cancelar}].

    \item El sistema muestra la interfaz
      \hipervinculo{lib_iu:resumen_de_compra}.

    \item El cliente presiona el botón \textit{Aceptar};
      [\hipervinculoLocal{ta:cancelar}].

    % Supongo que en estos últimos pasos se refleja lo incompleto de la tienda:
    % una tienda más en serio tendría que preocuparse por el envío de los
    % libros.

    \item El sistema registra la compra.

    \item El sistema muestra el mensaje
      \hipervinculo{lib_msj:compra_registrada}.

  \end{trayectoriaPrincipal}

  %%%%%%%%%%%%%%%%%%%%%%%%%%%%%%%%%%%%%%%%%%%%%%%%%%%%%%%%%%%%%%%%%%%%%%%%%%%%%%

  \begin{trayectoriaAlternativa}[lib_ta:cancelar]
    {El cliente cancela la operación}

    \item El cliente presiona el botón \textit{cancelar}.

    \item El sistema muestra la interfaz
      \hipervinculo{lib_iu:finalizar_compra}.

  \end{trayectoriaAlternativa}

}
