%
% Selección de tecnologías.
% Análisis y diseño de programa tokenizador, reporte técnico.
%
% Proyecto Lovelace.
%

\section{Tecnologías}
\label{sec:tecnologias}

Las primeras decisiones tomadas con respecto a la implementación y al diseño
fueron sobre el paradigma de programación y el lenguaje a utilizar. Para
tomar estas decisiones se tomaron en cuenta varias consideraciones: primero,
el paradigma orientado a objetos ofrece considerables ventajas con respecto
a la programación estructurada, por lo que, dentro de lo posible, se buscaría
un lenguaje con soporte a este paradigma; segudo, las implementaciones
criptográficas tienen altos requerimientos de rendimiento, por lo que, de los
posibles lenguajes, necesitabamos elegir uno que, si bien no el más rápido,
sí se encontrara entre los de mayor velocidad.

A lo largo de nuestra carrera hemos tenido contacto con bastantes lenguajes que,
si bien no dominamos, sí poseemos una base firme como para ser usada: C, C++,
Java, Python, Javascript y PHP. Por la cuestión del rendimiento, todos los
lenguajes interpretados (Python, Javascript, PHP) quedaron fuera de
consideración; la decisión (dentro de los lenguajes con soporte al paradigma
orientado a objetos) quedó entre Java y C++: Java (aún en las últimas versiones,
en donde se han hecho considerables progresos) es mucho más lento que C++,
dado que el trabajo de la máquina virtual en tiempo de ejecución es bastante
considerable; por lo tanto, si de un lenguaje orientado a objetos se trataba,
sería C++. La última decisión se dió entre C y C++: orientación a objetos contra
rendimiento. Al final se optó por la orientación a objetos: aún cuando C es
más rápido que C++, la diferencia no es tan grande, mientras que un programa
con un buen diseño orientado a objetos sí puede ser mucho más mantenible que
uno con un enfoque estructurado.

La verisón de C++ que utilizamos es C++14. Por razones de compatibilidad con
algunas de nuestras dependencias (en particular, el conector de la base de
datos) no pudimos utilizar la última versión al momento, C++17.

El gestor de base de datos que más hemos utilizado en la carrera es MySQL,
por lo que es el que seleccionamos para el programa tokenizador; en realidad,
se trata de una bifurcación de MySQL: MariaDB, cuyo uso ha sido ampliamente
difundido en varias distribuciones de Linux.

Otra decisión importante a tomar antes de hacer las implementaciones de los
algoritmos tokenizadores fue la librería de funciones criptográficas que
utilizaríamos. De todas las posibles librerías que están validadas
por el \gls{gl:nist} y que cuentan con una licencia para el uso público, al
final hicimos pruebas con dos: Openssl y Crypto++. Openssl está escrita
en C y cuenta con un historial que le da mucha reputación (es la librería
que utiliza Opengpg y Openssh). Cryptopp cuenta con una extensa red de
colaboradores, implementa una gran cantidad de algoritmos y está escrita sólo
en C++. De ambas, la que encontramos más fácil de usar fue Crypto++, por lo que
es la que utilizamos en nuestras implementaciones.
