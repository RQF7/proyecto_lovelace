%
% Lista de requerimientos funcionales,
% Capítulo de análisis y diseño de api web.
% Proyecto Lovelace
%

\subsubsection{No funcionales}

\requerimientoNoFuncional[rqnf_api:tamanios_de_pantalla]
{Sobre los tamaños de pantallas}
{
  Las interfaces gráficas de la aplicación deben estar diseñadas para
  responder a distintos tamaños de pantalla. En particular, se debe
  mostrar cómo se ve la interfaz para los siguientes tamaños (las unidades están
  en pixeles): muy grande (1920x1080), grande (1280x800), mediano (900x960),
  pequeño (600x960) y muy pequeño (480x800).
}

\requerimientoNoFuncional[rqnf_api:alamcenamiento_de_contrasenias]
{Sobre el almacenamiento de contraseñas}
{
  Las contraseñas nunca se almacenan en la base de datos. Lo que se almacena es
  un hash (sección \ref{sec:hash}); al momento de identificar a un usuario
  (\hipervinculo{rq:iniciar_sesion}) se calcula el hash de la
  contraseña introducida y se compara contra el que está almacenado en la base
  de datos. El algoritmo utilizado para estos hash debe de ser \gls{gl:sha}-256.
}

\requerimientoNoFuncional[rqnf_api:accesos_no_autorizados]
{Sobre los accesos no autorizados}
{
  Para una definición de lo que se entiende por un \textit{acceso no
  autorizado}, ver la regla de negocios \hipervinculo{rn:acceso_no_autorizado}.
  El trato para los accesos no autorizados depende del contexto; en los
  siguientes subrequerimientos se detallan las acciones a tomar para cada caso.

  \subrequerimientoNoFuncional[rqnf_api:accesos_fidedignos]
  {Accesos fidedignos}
  {
    Ver \hipervinculo{rn:acceso_fidedigno}. En caso de tratarse de un acceso
    fidedigno en la aplicación web, se debe redirigir a la pantalla
    \hipervinculo{iu:iniciar_sesion}. En caso de tratarse de un acceso fidedigno
    en el servicio tokenizador se debe mandar una respuesta \gls{gl:http} con el
    código 302.
  }

  \subrequerimientoNoFuncional[rqnf_api:accesos_erroneos]
  {Accesos erroneos}
  {
    Ver \hipervinculo{rn:acceso_erroneo}.
  }

  \subrequerimientoNoFuncional[rqnf_api:accesos_malintencionados]
  {Accesos malintencionados}
  {
    Ver \hipervinculo{rn:acceso_malintencionado}.
  }
}
