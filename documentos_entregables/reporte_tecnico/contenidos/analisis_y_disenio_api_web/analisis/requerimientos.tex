%
% Lista de requerimientos para sistema tokenizador,
% Capítulo de análisis y diseño de api web.
% Proyecto Lovelace.
%

\subsection{Requerimientos}

% Estos son los requerimientos que el analista «deduce» de su plática con el
% cliente. Teóricamente antes de estos tendrían que ir lo de «alto nivel»,
% en donde se exprese directamente las palabras del cliente, pero como eso
% pasó en mi mente, o no pasó en absoluto, van directamente los de bajo
% nivel.
% Requerimientos de bajo nivel:

\requerimiento[rq_api:actores]{API}
{Sobre los actores del sistema}
{
  La aplicación debe contemplar las acciones de tres tipos de actores: un
  \textit{visitante}, que representa a los posibles usuarios que visitan la
  página de promoción del sistema; un \textit{usuario}, que utiliza la interfaz
  web del servicio de tokenización; un \textit{administrador}, que representa a
  la entidad (empresa u organización) que ofrece el servicio de tokenización
  (ver regla de negocio \hipervinculo{rn:servicio_tokenizacion}).

  \subrequerimiento[rq_api:visitante]{API}
  {Sobre los visitantes}
  {
    Los \textit{visitantes} deben poder acceder solamente al sitio público de la
    aplicación: \hipervinculo{iu:bienvenida}, \hipervinculo{iu:documentacion},
    \hipervinculo{iu:iniciar_sesion}, \hipervinculo{iu:registrarse}.
  }

  \subrequerimiento[rq_api:usuario]{API}
  {Sobre los usuarios}
  {
    Los \textit{usuarios} deben poder acceder al sitio público de la aplicación
    (\hipervinculo{rq_api:visitante}) y al sitio de administración de tokens:
    \hipervinculo{iu:inicio_usuario}, \textit{}
  }

  \subrequerimiento[rq_api:administrador]{API}
  {Sobre los administradores}
  {

  }

  \subrequerimiento[rq_api:cambios_de_actores]{API}
  {Flujos de actores en la aplicación}
  {

  }
}

% Depende de rq_api:actores
\requerimiento[rq_api:interfaces_de_usuario]{API}
{Sobre las interfaces gráficas del sistema}
{
  La aplicación web debe contar con las interfaces especificadas en los
  siguientes subrequerimientos.

  \subrequerimiento[rq_api:sitio_publico]{API}
  {Sitio público de aplicación}
  {
    Estas páginas deben ser accesibles para los tres actores definidos en
    \hipervinculo{rq_api:actores}.

    \subsubrequerimiento[rq_api:inicio_sitio_publico]{API}
    {Inicio de sitio público}
    {
      Página que debe promocionar el servicio de tokenización. Debe de explicar
      qué es la tokenización y qué es lo que hace el servicio ofrecido. Esta
      interfaz se muestra en \hipervinculo{iu:bienvenida}.
    }

    \subsubrequerimiento[rq_api:documentacion]{API}
    {Documentación}
    {
      Página que debe explicar como utilizar la \gls{gl:api} web del
      servicio de tokenización. Esta interfaz se muestra en
      \hipervinculo{iu:documentacion}.
    }
  }

  \subrequerimiento[rq_api:administracion_de_tokens]{API}
  {Sitio de administración de tokens}
  {

  }

  \subrequerimiento[rq_api:administracion_de_usuarios]{API}
  {Sitio de administración general}
  {

  }
}
