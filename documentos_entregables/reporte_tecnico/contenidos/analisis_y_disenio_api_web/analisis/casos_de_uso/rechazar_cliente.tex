%
% Caso de uso para rechazar a un cliente.
% Capítulo de análisis y diseño de api web,
% Proyecto Lovelace.
%

\casoDeUso[cu:rechazar_cliente]
{Rechazar cliente}
{
  Permite a un usuario tipo administrador rechazar a un usuario tipo cliente;
  haciéndolo pasar del estado \textbf{verificado} a \textbf{rechazado} (véase
  \hipervinculo{rn:estados_cliente}).

  \begin{trayectoriaPrincipal}

    \item El administrador presiona el botón \textit{Verificar clientes}
      en el menú disponible en todas las pantallas.

    \item El sistema obtiene los datos de todos los clientes en la base de datos
    cuyo estado es \textbf{verificado}; véase
    [\hipervinculoLocal{ta:sin_verificados}].

    \item El sistema muestra la interfaz \textit{Inserte pantalla aquí} con la
      información obtenida.

    \item El administrador presiona el botón \textit{Rechazar} del cliente que
      desea aprobar.

    \item El sistema muestra el mensaje \hipervinculo{msj:rechazar_cliente}.

    \item El administrador presiona \textit{Aceptar}.

    \item El sistema cambia el estado de dicho cliente a \textbf{rechazado}
      tomando en cuenta la regla de negocios \hipervinculo{rn:estados_cliente};
      véase [\hipervinculoLocal{ta:error_actualizar_cliente}].

    \item [verificados_refrescados] El sistema obtiene los datos de todos los
      clientes en la base de datos cuyo estado es \textbf{verificado}.

    \item El sistema refresca la interfaz \textit{Inserte pantalla aquí}.

  \end{trayectoriaPrincipal}

  \begin{trayectoriaAlternativa}[ta:sin_verificados]
    {No hay clientes listos para ser aprobados en el sistema.}

    \item La consulta de clientes con estado \textbf{verificado} es vacía.

    \item El sistema muestra el mensaje
      \hipervinculo{msj:sin_clientes_verificados}.

  \end{trayectoriaAlternativa}

  \begin{trayectoriaAlternativa}[ta:error_actualizar_cliente]
    {Ocurrió un error al actualizar el estado del cliente.}

    \item El sistema muestra el mensaje
      \hipervinculo{msj:error_actualizar_cliente}.

    \item Regresar al paso \referenciaLocal{verificados_refrescados} de la
      trayectoria principal.

  \end{trayectoriaAlternativa}
}
