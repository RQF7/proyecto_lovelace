%
% Lista de casos de uso para sistema tokenizador
% Capítulo de análisis y diseño de api web
% Proyecto Lovelace.
%
% IMPORTANTE: Ver caso_de_uso.sty; los casos de uso definen un sistema de
% referencias locales, para poder ocupar las mismas etquetas en casos de uso
% distintos.
%

\subsection{Casos de uso}

Aquí va la descripción de los actores del sistema.

\newpage
\casoDeUso[cu:prueba]       % Etiqueta
{API}                       % Prefijo
{Caso de uso de prueba}     % Título
{
  Esta es la descripción del caso de uso. Aquí va el objetivo y el resumen del
  caso. Generalmente será un párrafo más bien largo. Generalmente será un
  párrafo más bien largo. Generalmente será un párrafo más bien largo.
  Generalmente será un párrafo más bien largo. Generalmente será un párrafo más
  bien largo. Generalmente será un párrafo más bien largo. Para satisfacer parte
  del requerimiento \hipervinculo{rqapi:prueba} y de a cuerdo a la regla de
  negocio \hipervinculo{rn:prueba}.

  \begin{trayectoriaPrincipal}
    \item El cliente hace tal.
    \item El sistema hace tal [\hipervinculoLocal{ta:errorEnTal}].
    \item[pasoDeRetorno] El cliente hace tal.
    \item Muestra la interfaz \hipervinculo{iu:prueba}.
  \end{trayectoriaPrincipal}

  \begin{trayectoriaAlternativa}[ta:errorEnTal]  % Etiqueta
    {El dato ingresado está mal}                 % Título
    \item El cliente hace tal.
    \item El sistema hace tal.
    \item El cliente hace tal.
    \item Regresa al paso \referenciaLocal{pasoDeRetorno} de la trayectoria
      principal.
  \end{trayectoriaAlternativa}

  \begin{trayectoriaAlternativa}[ta:errorEnTalDOS]
    {Error en servidor}
    \item El cliente hace tal.
    \item El sistema hace tal.
    \item El cliente hace tal.
    \item El sistema hace tal.
  \end{trayectoriaAlternativa}
}

