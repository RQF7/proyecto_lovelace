%
% Capítulo de conclusiones,
% Reporte técnico.
%
% Proyecto Lovelace.

\label{sec:conclusiones}
\capitulo{Conclusiones}
{%
  Cada día, Sancho, te vas haciendo menos simple y más discreto.%
}
{%
   Miguel de Cervantes.%
}

\noindent
En este curso se da por terminado el primero de tres módulos que componen al
trabajo: un programa generador de \textit{tokens}. Aunque de enunciado fácil,
el desarrollo de este programa encierra toda la variedad de temas que se ha
visto hasta el momento: estudio de los aspectos de criptografía involucrados,
investigación de los algoritmos generadores de \textit{tokens} existentes,
análisis de los estándares y recomendaciones oficiales relacionados,
comparaciones de desempeño, pruebas de aleatoriedad, etcétera. Con esto se
cumple, en gran medida, el objetivo general del trabajo: la implementación de
algoritmos generadores de \textit{tokens} con el propósito de proveer
confidencialidad a los datos de tarjetas bancarias. También se cubre en su
totalidad el primero de los objetivos específicos: la revisión de diversos
algoritmos para la generación de \textit{tokens}.

Los planes a futuro, para trabajo terminal II, incluyen el desarrollo de los
dos módulos restantes: la interfaz en red para utilizar el programa tokenizador
y la tienda en línea como cliente de la interfaz en red. Esto permitirá que lo
desarrollado en este semestre llegue hasta los usuarios finales, esto es, hasta
los clientes de las tiendas en línea. De esta forma se cubren los otros dos
objetivos específicos del trabajo.
