%
% Código de módulo de tienda en línea,
% Reporte técnico.
% Proyecto Lovelace.
%

\section{Aspectos relevenates de la implementación}

Al estar desarrollada sobre las mismas tecnologías, la implementación de la
tienda en línea es bastante similar a la del sistema tokenizador. Por esta
razón, el enfoque para mostrar ejemplos de código es distinto al de la sección
\ref{sec:codigo_sistema_tokenizador} (los aspectos relevantes de la
implementación del sistema tokenizador), en donde se mostró un ejemplo de cada
módulo del programa, tanto en el servidor como en el cliente. Aquí se muestran
los códigos fuente relacionados con la implementación de un solo caso de uso:
\hipervinculo{lib_cu:comprar}.

El primer paso consiste en la acción de inicio del usuario: el cliente presiona
el botón \textit{Finalizar compra} en \hipervinculo{lib_iu:finalizar_compra}.
En el código~\ref{codigo:pantalla_carrito} se muestra el fragmento de
\gls{gl:html} correspondiente a ese botón. Como se puede observar en la
línea~122, al presionar el botón se llama a la función \texttt{finalizarCompra}.
Esta función se muestra en el código~\ref{codigo:controlador_carrito}.

\codigoFuente[codigo:pantalla_carrito]{107}{127}{html}{%
  tienda/archivos_web/html/carrito.html}{%
  Botón \textit{Finalizar compra}, en pantalla de carrito.}

\codigoFuente[codigo:pantalla_carrito]{24}{50}{js}{%
  tienda/archivos_web/js/controladores/carrito.controlador.js}{%
  Función de finalización de compra.}

Lo primero que hacee esta función es verificar una de las condiciones de inicio
del caso \hipervinculo{lib_cu:iniciar_sesion}: si el usuario presiona
\textit{Finalizar compra} sin haber iniciado sesión antes, se le redirige a la
pantalla \hipervinculo{lib_iu:iniciar_sesion}. Si el usuario ya inició sesión,
se abre la ventana emergente con el proceso de finalización de compra. El
controlador asociado recibe, entre otras cosas, un objeto con los libros del
carrito, un arreglo con las tarjetas asociadas al usuario (paso número dos del
caso de uso) y un arreglo con las direcciones de entrega del usuario.
