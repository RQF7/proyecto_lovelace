%
% Pruebas Estadísticas del NIST para los RNGs y PRNGs usados en
% aplicaciones criptográficas, capítulo de análisis y diseño para
% la generación de tokens.
% Proyecto Lovelace.
%

\subsection{Pruebas para generadores de números aleatorios y pseudoaleatorios}
\label{sec:pruebas_para_generadores}

% NIST 800-22 R1A ---

\subsubsection{Definiciones} %=================================================

Existen dos tipos básicos de generadores, los \gls{gl:rng} y los \gls{gl:prng}.

\paragraph{Generadores aleatorios} %-------------------------------------------

Este tipo de generador usa fuentes de \gls{gl:entropia} no determinísticas
para producir números aleatorios, dichas fuentes de \gls{gl:entropia}
normalmente consisten en la medición de eventos físicos, como el ruido
en un circuito eléctrico.

La salidas de los \gls{gl:rng} puede usarse directamente para obtener números
aleatorios, o como una entrada de un \gls{gl:prng}. En el primer caso, es
necesario que la salida satisfaga pruebas estadísticas para determinar que
dicha salida parece aleatoria.

Los \gls{gl:rng} que se usan con fines criptográficos tienen que ser
impredecibles (ver sección \ref{sec:impredecibilidad}), por lo cual es
recomendado combinar varias fuentes de \gls{gl:entropia}, dado que algunas
(como los vectores de tiempo) son demasiado predecibles. De cualquier forma,
en la mayoría de los casos es mejor usar los \gls{gl:prng} para obtener de
manera directa números aleatorios.

\paragraph{Generadores pseudoaleatorios} %-------------------------------------

Estos son generadores de múltiples números aleatorios a partir de una o varias
entradas llamadas \gls{gl:semilla}. Cuando es necesario que la salida sea
impredecible, la \gls{gl:semilla} debe ser aleatoria y también impredecible
por sí sola, razón por lo cual esta se debe de obtener por medio de un
\gls{gl:rng}.

La salida de los \gls{gl:prng} normalmente son funciones determinísticas de
la \gls{gl:semilla}, por lo que la verdadera aleatoriedad queda en la
generación de esta misma, motivo por el que se usa el término de
\textit{pseudoaleatorio} y lo que causa la necesidad de mantener en secreto
la \gls{gl:semilla}.

Las mejores cualidades de los \gls{gl:prng} son que, al contrario de lo que
se creería por el nombre, suelen parecer más aleatorias que las \gls{gl:rng},
así como que producen números aleatorios más rápido.

\paragraph{Aleatoriedad} %-----------------------------------------------------

Una secuencia de bits aleatoria puede interpretarse como una secuencia en la
que el valor de cada bit es independiente, y los valores que lo conforman
están uniformemente distribuidos (ver \gls{gl:distribucion_uniforme}).

\paragraph{Impredecibilidad} %-------------------------------------------------
\label{sec:impredecibilidad}

Los generadores aleatorios y pseudoaleatorios deben de ser impredecibles.
La impredecibilidad es la incapacidad de conocer la salida de un bit $n+1$
aun conociendo los valores del bit $0$ al $n$ si la \gls{gl:semilla} es
desconocida. Esta también es la incapacidad de determinar la \gls{gl:semilla}
usando el conocimiento de los valores generados por la misma.

Hay que resaltar que conociendo tanto la \gls{gl:semilla} como el algoritmo
generador, el \gls{gl:prng} es completamente predecible, por lo cual en la
mayoría de los casos el algoritmo es público y la \gls{gl:semilla} se mantiene
secreta.

\paragraph{Pruebas estadísticas} %--------------------------------------------

Estas sirven para determinar si la secuencia a evaluar puede ser considerada
una secuencia aleatoria. Existe una gran diversidad de pruebas estadísticas
para comparar y evaluar una secuencia generada por un \gls{gl:rng} o un
\gls{gl:prng} contra una secuencia verdaderamente aleatoria, cada una de
estas buscando la existencia o ausencia de patrones para poder indicar
si la secuencia parece o no aleatoria.

Es necesario aclarar que ningún conjunto de pruebas puede ser considerado
completo, y que el resultado de determinada prueba es complementado por
otras, motivo por el cual los resultados obtenidos deben ser interpretados
cuidadosamente para no llegar a conclusiones incorrectas.

En \cite{nist_pruebas} se determina que las pruebas estadísticas están
formuladas para determinar si una secuencia binaria cumple con la hipótesis
nula \textit{(H0)}, o la hipótesis alternativa \textit{(Ha)}, las cuales dicen
que la secuencia probada es aleatoria y no aleatoria respectivamente. Cada
prueba descrita en la sección \ref{sec:lista_pruebas} consiste en aceptar
alguna de las dos hipótesis mencionadas.

Para cada prueba, es necesario elegir una medida estadística relevante de
aleatoriedad, para así establecer una \gls{gl:distribucion_probabilidad} y un
valor crítico que sirvan de referencia, y durante la prueba, se pueda
comparar el valor estadístico de la secuencia probada con el valor elegido;
aceptando la hipótesis nula en caso de que no se exceda del valor crítico y
rechazándola en caso contrario. De manera práctica, las pruebas estadísticas
funcionan dado que la \gls{gl:distribucion_probabilidad} de referencia y el
valor crítico son seleccionados asumiendo que la secuencia es aleatoria.

Hay que mencionar que estas pruebas estadísticas pueden tener dos tipos de
errores, que la secuencia sea aleatoria y se concluya que no lo es,
\textit{(error tipo I)}, y que la secuencia no sea aleatoria y se concluya
que sí, \textit{(error tipo II)}. La probabilidad de que ocurra el error de
tipo I es normalmente llamada \textit{nivel de significancia} de la prueba, y se
denota como $\alpha$, este valor dentro de la criptografía es común que ronde
cerca de $0.01$. La probabilidad de que ocurra el error de tipo II se denota
con $\beta$ y a diferencia de $\alpha$, no es un valor fijo y es más difícil
de calcular.

Uno de los principales objetivos de las pruebas es minimizar la probabilidad
de $\beta$, por lo cual se selecciona un tamaño de muestra $n$ y un valor para
$\alpha$, para después establecer un valor crítico que produzca el valor
$\beta$ más pequeño. Esto es posible gracias a que las probabilidades $\alpha$
y $\beta$ están relacionadas entre sí y con el tamaño de muestra $n$.

Cada prueba se basa en un valor estadístico de prueba calculado, que es una
función de los mismos datos, por lo cual si el valor estadístico de prueba
es $S$ y el valor crítico es $t$, entonces:

\begin{align}
  \alpha &= P(S > t\: ||\: H0\: es\: verdadera)
          = P(rechazar\: H0\: ||\: H0\: es\: verdadero)\\
  \beta  &= P(S \leq t\: ||\: H0\: es\: falso)
          = P(aceptar\: H0\: ||\: H0 es\: falso)
\end{align}

Las pruebas estadísticas calculan un valor $P$, que indica la probabilidad de
que un \gls{gl:rng} perfecto haya producido una secuencia menos aleatoria que
la secuencia que se probó, dado el tipo de aleatoriedad evaluada por la prueba.
En caso de que el valor de $P$ sea de $1$ en una prueba, se considera que la
secuencia parece tener una aleatoriedad perfecta y en caso de que el valor
sea $0$, la secuencia parece ser por completo no aleatoria.

Eligiendo un nivel de significancia $\alpha$ para las pruebas, si $P \geq
\alpha$, entonces la hipótesis nula es aceptada, es decir, la secuencia
parece ser aleatoria; y si $P < \alpha$, entonces la hipótesis nula es
rechazada, es decir, la secuencia parece ser no aleatoria. Normalmente
el valor de $\alpha$ se elige en el rango de $[0.001,\: 0.01]$.

\subsubsection{Pruebas} % =====================================================
\label{sec:lista_pruebas}

El paquete de pruebas descrito en el estándar del \gls{gl:nist}
\textit{800-22R1A: Statistical Suite for Random Number Generators}, disponible
en \cite{nist_pruebas}, es un conjunto de 15 algoritmos que se encargan de
probar la aleatoriedad de secuencias binarias de longitudes arbitrariamente
largas producidas por un \gls{gl:rng} o un \gls{gl:prng}.

\paragraph{Prueba de frecuencia} %---------------------------------------------
\textit{(The frequency (Monobit) Test)}

Esta prueba se encarga de medir la proporción de ceros y unos en toda la
secuencia binaria. Su propósito es determinar si la cantidad de ceros y de
unos es parecida a la que se esperaría en una secuencia verdaderamente
aleatoria. Algo a tener en cuenta es que todos las pruebas subsecuentes
dependen de pasar esta prueba.

%Pseudocodigo excediéndose de los 80 para una correcta visualización el en pdf.
%\begin{pseudocodigo}[caption={Proceso de la prueba de frecuencia.},
%label={pp_frecuencia}]
%    entrada:    una secuencia de bits generada por un RNG o PRNG $\epsilon$ de longitud $n$.
%    salida:     valor $P$ y determinación de si la secuencia $\epsilon$ es aleatoria.
%    inicio
%      calcular $S_n = sum_{i=0}^{n-1 X_i}$; donde $X_i = 2\epsilon_i - 1$.
%      calcular $S_{obs} = |S_n| / \sqrt{n}$.
%      calcular $P = erfc(\frac{S_{obs}}{\sqrt{2}})$; donde $erfc$ es la función de error complementario.
%      si $(P \geq \alpha)$
%        la secuencia es aleatoria.
%      si $(P < \alpha)$
%        la secuencia no es aleatoria.
%    fin
%\end{pseudocodigo}

%Se puede notar que cuando el valor de $P$ es muy pequeño es debido a que se
%tuvieron valores muy grandes de $|S_n|$ o $|S_obs|$ lo que significa que hubo
%una presencia muy desigual entre ceros y unos.

%Se recomienda que las secuencias a analizar tengan una longitud mínima de
%100 bits.

\paragraph{Prueba de frecuencia en un bloque} %--------------------------------
\textit{(Frequency Test within a Block)}

Esta prueba se centra en medir la proporción de unos dentro de $N$ bloques de
M-bits de la secuencia a probar. El propósito de la prueba es saber si la
frecuencia de unos se aproxima a $M/2$ como sería de esperar de una secuencia
aleatoria.

%Pseudocodigo excediéndose de los 80 para una correcta visualización el en pdf.
%\begin{pseudocodigo}[caption={Proceso de la prueba de frecuencia en un bloque.},
%label={pp_frecuencia_bloque}]
%    entrada:    una secuencia de bits generada por un RNG o PRNG $\epsilon$ de longitud $n$
%                y la longitud de bloque $M$.
%    salida:     valor $P$ y determinación de si la secuencia $\epsilon$ es aleatoria.
%    inicio
%      partir $\epsilon$ en $N = |\frac{n}{M}|$ bloques que no se trastalpen, descartando los bits que no se usen.
%      determinar la proporción $\pi_i$ de cada bloque usando la siguiente ecuación $\pi_i = \frac{\sum_{j=1}^{M}\epsilon_{i-1}M+j}{M}$, para $1 \leq i \leg N$.
%      calcular $\chi^2(obs) = 4 M \sum_{i=1}^{N}{(\pi_i 1/2)}^2$.
%      calcular $P = igamc(N/2, \chi(obs)/2)$; donde igamc es la función gamma incompleta.
%      si $(P \geq \alpha)$
%        la secuencia es aleatoria.
%      si $(P < \alpha)$
%        la secuencia no es aleatoria.
%    fin
%\end{pseudocodigo}

%Se puede observar que un valor pequeño en $P$ significa que hay una diferencia
%grande entre el número de ceros y unos en al menos un bloque.

%Por último es recomendable que las secuencias analizar tengan mínimo 100 bits,
%y que el valor de $M$ sea de al menos 20, mayor a $0.01n$ y menor a 100.

\paragraph{Prueba de carreras} %-----------------------------------------------
\textit{(The Runs Test)}

% ¿Porque el orden inverso en muchos lugares?
% Por ejemplo, aquí podrías definir primero qué se entiende por carrera y
% después hablar de la prueba.
% Lo que haces es un recurso válido, pero lo utilizas demasiado.

Esta prueba se encarga de calcular el número de carreras en la secuencia que
se está probando, donde una carrera es una subsecuencia ininterrumpida de
bits del mismo valor encerrada entre bits del valor opuesto. El objetivo de
la prueba es determinar si el número de carreras en la secuencia es parecido
al que se esperaría de una secuencia aleatoria, particularmente, establece si
la oscilación entre ceros y unos es muy rápida o muy lenta. Cabe resaltar que
la prueba tiene como requisito pasar la prueba de frecuencia.

%Pseudocodigo excediéndose de los 80 para una correcta visualización el en pdf.
%\begin{pseudocodigo}[caption={Proceso de la prueba de carreras.},
%label={pp_carreras}]
%    entrada:    una secuencia de bits generada por un RNG o PRNG $\epsilon$ de longitud $n$.
%    salida:     valor $P$ y determinación de si la secuencia $\epsilon$ es aleatoria.
%    inicio
%      calcular la proporción de unos $\pi = \frac{\sum_{j=0}^{n}\epsilon_j}{n}$ de la secuencia.
%      determinar si la prueba de frecuencia fue aprobada:
%        si se puede demostrar que $|\pi - 1/2| \geq \tau$, donde $\tau = 2/\sqrt{2}$, esta prueba no necesita realizarse.
%      calcular el número total de carreras $V_n(obs) = \sum_{k=1}^{n-1}r(k)+1$; donde $r(k) = 0$ si $\epsilon_k=\epsilon_{k+1}$ y $r(k) = 1$ en otro caso.
%      calcular el valor de $P = erfc(\frac{|V_n(obs) - 2n\pi(1-\pi)|}{2\sqrt{2n}\pi(1-\pi)})$.
%      si $(P \geq \alpha)$
%        la secuencia es aleatoria.
%      si $(P < \alpha)$
%        la secuencia no es aleatoria.
%    fin
%\end{pseudocodigo}

%Se puede interpretar a $V_n(obs)$ como la velocidad de oscilación entre unos
%y ceros de la secuencia probada, por lo cual, debe ser un valor intermedio,
%dado que en una secuencia aleatoria no se espera que la oscilación sea muy
%rápida ni muy lenta.

%Al igual que en las pruebas previas se recomienda que la longitud de la
%secuencia $\epsilon$ sea de al menos 100 bits.

\paragraph{Prueba de la carrera más larga en un bloque} %----------------------
\textit{(Tests for the Longest-Run-of-Ones in a Block)}

El propósito de esta prueba es determinar si la longitud de la carrera más
larga en la secuencia es consistente con la longitud de la carrera más larga
de una secuencia aleatoria. Se tiene que resaltar que encontrar una
irregularidad en la carrera más larga de unos implica una irregularidad en
la carrera más larga de ceros, por lo cual solo es necesario hacer la prueba
para la carrera más larga de unos.

\paragraph{Prueba del rango de matriz binaria} %-------------------------------
\textit{(The Binary Matrix Rank Test)}

% ¿Y qué es una dependencia lineal?

La prueba tiene como finalidad revisar si existen dependencias lineales entre
subcadenas de longitud fija de la secuencia a probar.

\paragraph{Prueba Espectral} %-------------------------------------------------
\textit{(The Discrete Fourier Transform (Spectral) Test)}

Esta prueba se encarga de obtener los picos más altos de la transformada
discreta de Fourier de la secuencia. Su propósito es detectar comportamientos
periódicos en la secuencia que puedan indicar que esta no puede asumirse como
aleatoria. De manera más precisa, la prueba detecta si el número de picos que
exceden un umbral del 95\% son significativamente diferentes al 5\%.

\paragraph{Prueba de coincidencia sin superposición} %-------------------------
\textit{(The Non-overlapping Template Matching Test)}

La prueba tiene como finalidad detectar generadores que producen una gran
cantidad de ocurrencias de un patrón no periódico en sus secuencias de salida.
Para esta prueba se utiliza una ventana de $m$ bits que recorre la secuencia,
para buscar patrones específicos de $m$ bits. Dado que esta prueba es sin
traslape, en caso de que se encuentre el patrón se recorre la ventana m bits
y de lo contrario se recorre uno.

\paragraph{Prueba de coincidencia con superposición} %-------------------------
\textit{(The Overlapping Template Matching Test)}

La prueba tiene como finalidad detectar generadores que producen una gran
cantidad de ocurrencias de un patrón no periódico en sus secuencias de salida.
Para esta prueba se utiliza una ventana de $m$ bits que recorre la secuencia,
para buscar patrones específicos de $m$ bits. Dado que esta prueba es con
traslape, independientemente de si se encontra o no un patrón, la ventana se
recorre un bit.

\paragraph{Prueba estadística universal de Maurer} %---------------------------
\textit{(Maurer’s “Universal Statistical” Test)}

Esta prueba se centra en medir el número de bits existentes entre patrones
que coinciden, medida que se relaciona con el tamaño mínimo para comprimir
una secuencia. La prueba tiene como finalidad determinar si una secuencia
puede comprimirse de manera significativa sin perder información, lo cual
significa que no es aleatoria.

\paragraph{Prueba de complejidad lineal} %-------------------------------------
\textit{(The Linear Complexity Test)}

El objetivo de la prueba es determinar si la secuencia a probar es lo bastante
compleja como para considerarse aleatoria, esto por medio de medir el tamaño de
su \gls{gl:registro_de_desplazamiento_con_retroalimentacion_lineal}
correspondiente, ya que las secuencias aleatorias se caracterizan por tener
valores altos en este parámetro.

\paragraph{Prueba serial} %----------------------------------------------------
\textit{(The Serial Test)}

Esta prueba tiene como objetivo determinar si el número de ocurrencias de
$2^m$ patrones de $m$ bits son aproximadamente las mismas que se esperarían
de una cadena aleatoria, considerando que al ser una secuencia aleatoria,
es una secuencia uniforme, y cada patrón de $m$ bits tiene la misma
probabilidad de aparecer que los demás.

\paragraph{Prueba de entropía aproximada} %------------------------------------
\textit{(The Approximate Entropy Test)}

La prueba se encarga de comparar la frecuencia de sobreponer dos bloques de
longitudes consecutivas con el resultado esperado de una secuencia que es
aleatoria.

\paragraph{Prueba de sumas acumulativas} %-------------------------------------
\textit{(The Cumulative Sums (Cusums) Test)}

Su objetivo es determinar si las sumas acumulativas de una secuencia parcial
de la secuencia probada son muy grandes o pequeñas con relación a las sumas
acumulativas de una secuencia aleatoria. Las sumas aculumativas son
consideradas como caminatas aleatorias, y para las secuencias aleatorias
la realización de una caminata aleatoria debe tener un valor cercano a cero.

% A esta no le entendí muy bien.
\paragraph{Prueba de excursiones aleatorias} %---------------------------------
\textit{(The Random Excursions Test)}

Esta prueba tiene como finalidad determinar si el número de apariciones de un
estado $x$ en un ciclo es parecido al número que se esperaría de una secuencia
aleatoria. Esta prueba está conformada de 8 subpruebas, cada una para los
valores de estado $x$ de -4, -3, -2, -1, 1, 2, 3, 4.

% A esta no le entendí muy bien.
\paragraph{Prueba variante de excursiones aleatorias} %------------------------
\textit{(The Random Excursions Variant Test)}

La prueba se centra en calcular el número de veces que un estado en particular
ocurre en una suma acumulativa de una camita aleatoria. Su propósito, es
detectar desviaciones del número esperado de ocurrencias de varios estados
en una caminata aleatoria. Esta prueba está conformada de 18 subpruebas, cada
un para los valores de estado $x$ de -9 a de -1 y  1 a 9.
