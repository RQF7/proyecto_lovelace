%
% Descripción del modo de operación de BPS, presentación de cifrados que
% preservan el formato.
% Proyecto Lovelace.
%

\subsection{Modo de operación}

%------------------------------------------------------------------------------
\begin{frame}{BPS}{Modo de operación}

  El modo de operación usado por \textit{BPS} es equivalente al modo de
  operación CBC, ya que el bloque $BC_n$ utiliza el texto cifrado de la
  salida del bloque $BC_{n-1}$, con la distinción de que en lugar de
  aplicar operaciones \textit{xor} usa sumas modulares caracter por
  caracter, y de que no utiliza un vector de inicialización.

\end{frame}

%------------------------------------------------------------------------------
\begin{frame}{BPS}{Modo de operación}

  \begin{figure}[H]
    \begin{center}
      \includegraphics[width=1.00\linewidth]
        {../../../diagramas_comunes/bps/modo_de_operacion_bps}
      \caption{Modo de operación de \textit{BPS}.}
     \end{center}
  \end{figure}

\end{frame}

%------------------------------------------------------------------------------
\begin{frame}{BPS}{Modo de operación}

  Como se observó en la figura anterior, se utiliza un contador $u$ de 16
  bits para aplicar una \textit{xor} al \textit{tweak} $T$ en la entrada de
  cada $BC$.

  El \textit{xor} se aplica a los 16 bits más significativos de ambas mitades
  de \textit{tweak}, debido a cada mitad de \textit{tweak} funciona de manera
  independiente en $BC$, y a que no se desea un traslape entre el contador
  externo e interno.

\end{frame}

%------------------------------------------------------------------------------
\begin{frame}{BPS}{Modo de operación}

  Con este modo de operación, el tamaño de cada bloque $B_i$ debe ser igual 
  a $max(b)$ y cuando el texto en claro a cifrar no tenga una longitud total 
  que sea múltiplo de este valor, en el último bloque el cifrador $BC$ tendrá 
  una longitud igual a la de ese bloque.

  \begin{figure}[H]
    \begin{center}
      \includegraphics[width=0.9\linewidth]
        {../../../diagramas_comunes/bps/cursor_bps}
      \caption{Corrimiento de cursor de selección del ultimo bloque.}
     \end{center}
  \end{figure}

\end{frame}

%------------------------------------------------------------------------------
