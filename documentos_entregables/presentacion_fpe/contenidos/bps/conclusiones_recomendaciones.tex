%
% Conclusion y Recomendaciones, presentación de cifrados que preservan el formato.
% Proyecto Lovelace.
%

\subsection{Conclusiones y Recomendaciones}

%------------------------------------------------------------------------------
\begin{frame}{BPS}{Conclusiones}

  \begin{itemize}
    \item \textit{BPS} está basado en las redes Feistel y primitivas 
      criptográficas estandarizadas, lo cual puede verse como una ventaja, 
      debido al amplio estudio que tienen, y a que hacen más comprensible 
      y fácil su implementación.
    
    \item \textit{BPS} es un cifrado que preserva el formato capaz de cifrar 
      cadenas formadas por cualquier conjunto y de un longitud de $2$ hasta 
      $max(b) \cdot 2^{b}$.
  \end{itemize}
  
\end{frame}

%------------------------------------------------------------------------------
\begin{frame}{BPS}{Conclusiones}

  \begin{itemize}
    \item Se puede considerar que \textit{BPS} es eficiente, debido a que la 
      llave $K$ usada en cada bloque $BC$ es constante, y a que usa un número 
      reducido de operaciones internas.
    
    \item El uso de tweaks protege a \textit{BPS} de ataques de diccionario, 
      los cuales son fáciles de cometer cuando el dominio de la cadena a 
      cifrar es muy pequeño.
  \end{itemize}
  
\end{frame}

%------------------------------------------------------------------------------
\begin{frame}{BPS}{Recomendaciones}

  \begin{itemize}
    \item Se recomienda que el número de rondas $w$ de la red Feistel sea 
      $8$, dado que es una número de rondas eficiente, y se ha estudiado 
      la seguridad de \textit{BPS} con este $w$.
    
    \item Es recomendable que como tweak se use la salida truncada de una 
      función hash, en donde la entrada de la función puede ser cualquier 
      información relacionada a los datos que se deseen proteger, como por 
      ejemplo fechas, lugares, o parte de los datos que no se deseen cifrar. 
  \end{itemize}
  
\end{frame}

%------------------------------------------------------------------------------
