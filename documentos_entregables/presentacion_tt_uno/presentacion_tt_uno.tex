%
% Presentación de TT uno.
% Proyecto Lovelace.
%

\documentclass{beamer}
\usepackage{formato}
\newcommand{\Titulo}{Generación de tokens para proteger
                     los datos de tarjetas bancarias}
\renewcommand{\Autor}{Daniel Ayala Zamorano \\\vspace{-2mm}
                      {\tiny \texttt{daz23ayala@gmail.com}} \\
                      Laura Natalia Borbolla Palacios \\\vspace{-2mm}
                      {\tiny \texttt{ln.borbolla42@gmail.com}} \\
                      Ricardo Quezada Figueroa \\\vspace{-2mm}
                      {\tiny \texttt{qf7.ricardo@gmail.com}}}
\newcommand{\Fecha}{Ciudad de México, 9 de mayo de 2018}
\usepackage{../presentacion_fpe/formato}

\begin{document}

  {\setbeamertemplate{footline}{}
  \frame{\titlepage}}

  \begin{frame}
    \frametitle{Contenido}
    \tableofcontents[pausesections]
  \end{frame}

  % Espacio entre párrafos
  \setlength{\parskip}{0.5em}

  %\import{presentacion_fpe/contenidos/tarjetas/}{tarjetas}
  %\import{presentacion_fpe/contenidos/fpe/}{fpe}
  %\import{presentacion_fpe/contenidos/ffx/}{ffx}
  %\import{presentacion_fpe/contenidos/bps/}{bps}

  \section{Planteamiento del problema}

  \begin{frame}{}

  \end{frame}

  \section{Planteamiento de la solución}

  \begin{frame}{}

  \end{frame}

  \section{Algoritmos generadores de \textit{tokens}}

  \begin{frame}{}

  \end{frame}

  \section{Trabajo a futuro}

  \begin{frame}{}

  \end{frame}

  \begin{frame}[allowframebreaks]{Bibliografía}
    \printbibliography
  \end{frame}

\end{document}
