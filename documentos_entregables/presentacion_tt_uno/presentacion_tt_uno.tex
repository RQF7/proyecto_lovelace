%
% Presentación de TT uno.
% Proyecto Lovelace.
%

\documentclass{beamer}
\usepackage{formato}
\newcommand{\Titulo}{Generación de tokens para proteger
                     los datos de tarjetas bancarias}
\newcommand{\Fecha}{Ciudad de México, 9 de mayo de 2018}
\usepackage{formato_presentaciones}

\begin{document}

  {\setbeamertemplate{footline}{}
  \frame{\titlepage}}

  \begin{frame}
    \frametitle{Contenido}
    \setcounter{tocdepth}{1}
    \tableofcontents[pausesections]
  \end{frame}

  % Espacio entre párrafos
  \setlength{\parskip}{0.5em}

  \section{Planteamiento del problema}

  \import{presentacion_tt_uno/contenidos/}{planteamiento_problema}

  \import{presentacion_tt_uno/contenidos/planteamiento_de_solucion/}
    {planteamiento_de_solucion}

  \import{presentacion_tt_uno/contenidos/tokens/}{tokens}

  \import{presentacion_tt_uno/contenidos/conclusiones/}{conclusiones}

  \begin{frame}[allowframebreaks]{Bibliografía}
    \printbibliography
  \end{frame}

  % Espacio entre párrafos
  \setlength{\parskip}{0.0em}

  {\setbeamertemplate{footline}{}
  \frame{\titlepage}}

\end{document}
