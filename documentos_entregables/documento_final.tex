%
% Archivo principal de documento final.
% Proyecto Lovelace.
%

\documentclass[11pt, letterpaper, oneside]{book}
\usepackage[spanish, mexico]{babel}
\usepackage[utf8]{inputenc}
\newcommand{\Titulo}{Generación de tokens para proteger
                     datos de tarjetas bancarias}
\newcommand{\Subtitulo}{Número 20180008}
\newcommand{\Director}{Dra. Sandra Díaz Santiago}
\newcommand{\Autor}{Daniel Ayala Zamorano \\
                    Laura Natalia Borbolla Palacios \\
                    Ricardo Quezada Figueroa}
\newcommand{\Fecha}{Enero de 2018}
\usepackage{formato}

\begin{document}

  \Portada

  % TODO: hacer que el título del índice funcione como sección, no como
  % capítulo; crear dos índices, el primero general (sólo dos niveles) y el
  % segundo a detalle (todos los niveles).
  \pdfbookmark[1]{\contentsname}{Contenido}
  \setcounter{tocdepth}{4}
  \tableofcontents
  \newpage

  %
% Capítulo de introducción,
% reporte técnico.
%
% Proyecto Lovelace.
%

\capitulo{Introducción}{sec:introduccion}
{
  \epigrafe
  {%
    In the beginning the Universe was created. This has made a lot of people
    very angry and has been widely regarded as a bad move.%
  }
  {%
    The Hitchhiker's Guide to the Galaxy,\\
    \textsc{Douglas Adams}.%
  }
  \epigrafe
  {%
    Welcome to a new year at Hogwarts! Before we begin our banquet, I would
    like to say a few words. And here they are: Nitwit! Blubber!
    Oddment! Tweak!%
  }
  {%
    Harry Potter and the Sorcerer's Stone,\\
    \textsc{J. K. Rowling}.%
  }
}

\noindent
A finales de los ochenta, el uso de las computadoras y el internet comenzó a
popularizarse; compañías ya establecidas, como aerolíneas y tiendas
departamentales, y comerciantes independientes vieron una oportunidad de
expandirse y el comercio en línea comenzó a tomar fuerza; sin embargo, casi
nadie previó el impacto y auge que iba a tener, por lo que la mayoría de los
sitios no se encontraban preparados para los ataques y robos de información.
Las principales emisoras de tarjetas (MasterCard y Visa) reportaron entre 1988 y
1998 pérdidas de 750 millones de dólares debidas a fraudes con
tarjetas bancarias: el crecimiento del comercio electrónico aunado a
sistemas débilmente protegidos dio lugar a un rápido crecimiento de los
fraudes relacionados con tarjetas bancarias; de hecho, para el año 2001,
según \cite{wallethub}, se tuvieron pérdidas de 1.7 miles de millones de
dólares y, para el siguiente año habían aumentado a 2.1 miles de millones de
dólares.

Las compañías emisoras de tarjetas comenzaron a proponer soluciones y,
a finales de 1999, Visa publicó un documento con una serie de recomendaciones
de seguridad para quienes realizaban transacciones en línea llamado
\gls{gl:cisp}, este programa es el primer precursor del estándar actual
\gls{gl:pci}~\gls{gl:dss}. Lamentablemente, las recomendaciones no estaban
unificadas y había inconsistencias entre ellas, por lo que pocas compañías
fueron capaces de satisfacer completamente alguna de las normas publicadas.

Fue hasta finales de 2004 cuando se publicó el primer estándar unificado,
respaldado por las compañías emisoras de tarjetas más importantes: el
\gls{gl:pci}~\gls{gl:dss} 1.0, en el cuál se indica a los comercios cómo
mantener los datos bancarios seguros mediante protocolos de seguridad, y se hizo
obligatorio para todos aquellos que realizaran más de 20,000 transacciones
anuales. Aunque cada vez más compañías comenzaron a seguirlo e invertir para
satisfacerlo, de nuevo fueron pocas las que alcanzaron a cumplirlo
completamente, pues tiene una gran cantidad de requerimientos (controles
estrictos de acceso, monitoreo regular de las redes, mantener programas de
vulnerabilidades y políticas de seguridad de información, etcétera) y las
compañías tendían a subestimar los costos que implica seguir el
estándar~\cite{uk_association, search_security}.

A finales del 2006 ocurrió una de las primeras grandes violaciones de datos
que puso en guardia a todos: hubo una intrusión en los servidores de la empresa
americana TJX y piratas cibernéticos robaron información de tarjetas de crédito,
débito y transacciones de 94 millones de clientes registrados en su sistema.
Ataques como este han continuado a lo largo de los años: la cadena
de supermercados Hannaford Bros. sufrió un embate en 2008 y se vieron
comprometidas 4.2 millones de cuentas, Target fue atacado en 2013 y 40 millones
de cuentas fueron afectadas, y, al año siguiente, arremetieron contra Home Depot
y la información de 56 millones de usuarios fue robada.

Durante la primera década del siglo XXI, el enfoque que se tenía era
salvaguardar la información sensible en todo el sistema; tómese por ejemplo el
caso de una tienda en línea: el número de tarjeta queda registrado
en el área de clientes, pues se puede asociar con un perfil y evita tener que
estar ingresando continuamente toda la información de la tarjeta; también queda
registrada en el área de ventas, pues queda asociada a una compra o transacción;
la información sensible parece estar en todos lados (al menos, no está
concentrada) y tener que protegerla constantemente resulta muy costoso. Pasados
unos años, surge la idea de cambiar completamente el paradigma que se había
estado utilizando hasta ahora: ¿por qué intentar proteger la información
sensible en todos los lados donde sea que se encuentre, cuando se puede cambiar
esa información por un valor sustituto y solo proteger esa pequeña parte del
sistema? Así que, a mediados del 2011, el \gls{gl:pci}~\gls{gl:dss} publicó
un documento cuyo título en inglés es «\textit{PCI DSS tokenization
guidelines}»; a su vez, varias compañías comenzaron a ofrecer soluciones de
tokenización\footnotemark{} que quitaban casi por completo el peso de cumplir
con el \gls{gl:pci}~\gls{gl:dss} a los comerciantes, pues ellas se encargan de
generar los \glspl{gl:token} y almacenar los datos sensibles; mientras que
ellos, los comerciantes, se quedan solo con el \gls{gl:token}; así, si la
seguridad de su sitio es violada, los datos siguen estando seguros, pues no se
puede robar lo que no existe dentro del sistema.

%
% El grupo vacío después de «footnotemark» no es necesario por la sintaxis,
% sin embargo, si no se pone, LaTeX deja un espacio muy corto entre el número
% y la siguiente palabra. Extraño.
%
% https://tex.stackexchange.com/questions/42982/wrong-space-with-footnotemark
%

\footnotetext{
  El término \textit{tokenización} es un anglicismo y, en este documento,
  se refiere a la acción y efecto de \textit{tokenizar}; es decir, dado un
  valor, obtener un valor sustituto (\gls{gl:token}). La \textit{detokenización}
  es el proceso inverso: dado el valor sustituto (\gls{gl:token}), obtener el
  valor original. Estos préstamos lingüísticos se utilizan de manera constante
  a lo largo de este reporte.
}

Aunque en este trabajo se hace especial referencia a la tokenización de los
números de tarjetas, o \gls{gl:pan}, este proceso sirve para proteger otros
datos, como números de seguridad social, claves de registro, números de
teléfono, etcétera. De hecho, la tokenización no está limitada al mundo digital
y ha estado presente desde hace mucho tiempo, por ejemplo, los billetes y
monedas, o las fichas en un casino: uno deposita dinero y obtiene su
equivalente en fichas (este proceso es el de tokenización) para jugar; aunque
las fichas (o \glspl{gl:token}) están vigiladas, si alguien las roba, el casino
no pierde tanto (siempre y cuando no sean cambiadas por dinero, o sea, realizar
el proceso de detokenización), pues es una mera representación, y no el dinero
mismo. Lo mismo sucede con los \glspl{gl:token} digitales: sustituyen
información valiosa, por valores que carecen de significado y cuya pérdida
no representa un peligro inminente o una violación de la privacidad.

Todos los métodos para generar \glspl{gl:token} que se presentan en este trabajo
(y la gran mayoría de los métodos conocidos) competen enteramente a la
criptografía. Entre otras cosas, la criptografía permite proteger información de
terceros no autorizados, esto es, permite obtener confidencialidad. La idea
básica es transformar un mensaje de forma que solo el destinatario sea capaz
de hacer la transformación inversa y leer el mensaje original. Aunque con
métodos un poco distintos a los actuales, la criptografía tiene una larga
historia: el uso más antiguo del que se tiene noticia es en jeroglíficos
egipcios de alrededor de 1900 a. C.; a partir de ahí, hay evidencias de su uso
en muchas otras culturas antiguas~\cite{codebreakers}. La historia de la
criptografía moderna está muy relacionada con la historia de la computación: en
muchas ocasiones, la principal motivación para el desarrollo de máquinas más
potentes fue la posibilidad de crear un método de cifrado infalible (o del
método para romperlo); el ejemplo más claro de esto es el desarrollo de
métodos para descifrar los mensajes de los nazis, durante la segunda guerra
mundial, lo que permitió a Alan Turing sentar las bases de la computación
actual~\cite{simon_singht}. Esta dinámica de juego, en la cuál cada participante
está siempre buscando un método que los adversarios no pueden romper,
hace de la criptografía una ciencia en constante movimiento: una vez que
se encuentra una nueva vulnerabilidad en algún método, se trabaja para
corregirlo o para reemplazarlo.

Uno de los aspectos más importantes de las herramientas criptográficas
modernas es que buscan probar, dentro de los límites posibles, la
seguridad de un algoritmo. Es por esto que hoy en día no basta con
describir un nuevo método y esperar que sea lo suficientemente fuerte,
sino que la presentación de un método debe de ir acompañada de los
argumentos (las pruebas) que garantizan su seguridad. Hacer esto no es
tarea sencilla (si lo fuera, el juego ya estaría ganado para uno de los
bandos) pues involucra hacer suposiciones sobre las capacidades de
los adversarios y, basándose en esto, garantizar que los recursos necesarios
para romper el método probado se encuentran muy por encima de las capacidades
supuestas en un inicio. En este contexto, las capacidades de un adversario
se determinan por su poder de cómputo: ¿cuántas computadoras tiene a su
disposición?, ¿qué tan rápidas son?, etcétera.

Lo anterior nos lleva a otro principio muy importante de la criptografía
moderna: la seguridad se encuentra en los datos, no en el método. La historia
de la criptografía muestra que durante mucho tiempo, la seguridad de un método
consistía totalmente en mantenerlo en secreto: mientras los adversarios
no supieran cómo se estaba cifrando algo, los mensajes permanecerían seguros;
sin embargo, si el método se filtraba, todos los mensajes cifrados con ese
método se veían comprometidos. Este esquema presenta un serio problema,
pues hace dependiente de un solo secreto, el método, la seguridad de todos
los mensajes. Esto llevó a la introducción del concepto de la llave dentro
de la criptografía: un valor que solo conocen las entidades autorizadas a leer
el mensaje, sin el cuál no pueden descifrarlo; de esta forma la seguridad de un
mensaje se basa tanto en el conocimiento del método, como en el conocimiento
de la llave. En la mayoría de los métodos usados hoy en día, la seguridad se
basa enteramente en la llave, dejando el método abierto al escrutinio de todos.
Hacer esto presenta varias ventajas: una muestra de que el algoritmo está
bien diseñado y es lo suficientemente fuerte es que no necesita mantenerse en
secreto; cuando el algoritmo es público y es usado por muchas entidades (entre
más mejor) son mayores las posibilidades de encontrar vulnerabilidades y
el tiempo de respuesta de las correcciones es menor.

La presentación que se hace en este trabajo de los métodos de tokenización
tiene un enfoque totalmente criptográfico. De esta forma se busca demostrar
que la tokenización es una aplicación de la criptografía, no una alternativa
a esta. Es por esto que todo el capítulo del marco teórico está dedicado a los
mecanismos criptográficos que se relacionan con los métodos para generar
\glspl{gl:token}.

\subimport{/}{objetivos}
\subimport{/}{justificacion}
\subimport{/}{estructura}

  %
% Capítulo de antecedentes.
% Proyecto Lovelace.
%

\chapter{Antecedentes}

%
% Capítulo de introducción,
% reporte técnico.
%
% Proyecto Lovelace.
%

\capitulo{Introducción}{sec:introduccion}
{
  \epigrafe
  {%
    In the beginning the Universe was created. This has made a lot of people
    very angry and has been widely regarded as a bad move.%
  }
  {%
    The Hitchhiker's Guide to the Galaxy,\\
    \textsc{Douglas Adams}.%
  }
  \epigrafe
  {%
    Welcome to a new year at Hogwarts! Before we begin our banquet, I would
    like to say a few words. And here they are: Nitwit! Blubber!
    Oddment! Tweak!%
  }
  {%
    Harry Potter and the Sorcerer's Stone,\\
    \textsc{J. K. Rowling}.%
  }
}

\noindent
A finales de los ochenta, el uso de las computadoras y el internet comenzó a
popularizarse; compañías ya establecidas, como aerolíneas y tiendas
departamentales, y comerciantes independientes vieron una oportunidad de
expandirse y el comercio en línea comenzó a tomar fuerza; sin embargo, casi
nadie previó el impacto y auge que iba a tener, por lo que la mayoría de los
sitios no se encontraban preparados para los ataques y robos de información.
Las principales emisoras de tarjetas (MasterCard y Visa) reportaron entre 1988 y
1998 pérdidas de 750 millones de dólares debidas a fraudes con
tarjetas bancarias: el crecimiento del comercio electrónico aunado a
sistemas débilmente protegidos dio lugar a un rápido crecimiento de los
fraudes relacionados con tarjetas bancarias; de hecho, para el año 2001,
según \cite{wallethub}, se tuvieron pérdidas de 1.7 miles de millones de
dólares y, para el siguiente año habían aumentado a 2.1 miles de millones de
dólares.

Las compañías emisoras de tarjetas comenzaron a proponer soluciones y,
a finales de 1999, Visa publicó un documento con una serie de recomendaciones
de seguridad para quienes realizaban transacciones en línea llamado
\gls{gl:cisp}, este programa es el primer precursor del estándar actual
\gls{gl:pci}~\gls{gl:dss}. Lamentablemente, las recomendaciones no estaban
unificadas y había inconsistencias entre ellas, por lo que pocas compañías
fueron capaces de satisfacer completamente alguna de las normas publicadas.

Fue hasta finales de 2004 cuando se publicó el primer estándar unificado,
respaldado por las compañías emisoras de tarjetas más importantes: el
\gls{gl:pci}~\gls{gl:dss} 1.0, en el cuál se indica a los comercios cómo
mantener los datos bancarios seguros mediante protocolos de seguridad, y se hizo
obligatorio para todos aquellos que realizaran más de 20,000 transacciones
anuales. Aunque cada vez más compañías comenzaron a seguirlo e invertir para
satisfacerlo, de nuevo fueron pocas las que alcanzaron a cumplirlo
completamente, pues tiene una gran cantidad de requerimientos (controles
estrictos de acceso, monitoreo regular de las redes, mantener programas de
vulnerabilidades y políticas de seguridad de información, etcétera) y las
compañías tendían a subestimar los costos que implica seguir el
estándar~\cite{uk_association, search_security}.

A finales del 2006 ocurrió una de las primeras grandes violaciones de datos
que puso en guardia a todos: hubo una intrusión en los servidores de la empresa
americana TJX y piratas cibernéticos robaron información de tarjetas de crédito,
débito y transacciones de 94 millones de clientes registrados en su sistema.
Ataques como este han continuado a lo largo de los años: la cadena
de supermercados Hannaford Bros. sufrió un embate en 2008 y se vieron
comprometidas 4.2 millones de cuentas, Target fue atacado en 2013 y 40 millones
de cuentas fueron afectadas, y, al año siguiente, arremetieron contra Home Depot
y la información de 56 millones de usuarios fue robada.

Durante la primera década del siglo XXI, el enfoque que se tenía era
salvaguardar la información sensible en todo el sistema; tómese por ejemplo el
caso de una tienda en línea: el número de tarjeta queda registrado
en el área de clientes, pues se puede asociar con un perfil y evita tener que
estar ingresando continuamente toda la información de la tarjeta; también queda
registrada en el área de ventas, pues queda asociada a una compra o transacción;
la información sensible parece estar en todos lados (al menos, no está
concentrada) y tener que protegerla constantemente resulta muy costoso. Pasados
unos años, surge la idea de cambiar completamente el paradigma que se había
estado utilizando hasta ahora: ¿por qué intentar proteger la información
sensible en todos los lados donde sea que se encuentre, cuando se puede cambiar
esa información por un valor sustituto y solo proteger esa pequeña parte del
sistema? Así que, a mediados del 2011, el \gls{gl:pci}~\gls{gl:dss} publicó
un documento cuyo título en inglés es «\textit{PCI DSS tokenization
guidelines}»; a su vez, varias compañías comenzaron a ofrecer soluciones de
tokenización\footnotemark{} que quitaban casi por completo el peso de cumplir
con el \gls{gl:pci}~\gls{gl:dss} a los comerciantes, pues ellas se encargan de
generar los \glspl{gl:token} y almacenar los datos sensibles; mientras que
ellos, los comerciantes, se quedan solo con el \gls{gl:token}; así, si la
seguridad de su sitio es violada, los datos siguen estando seguros, pues no se
puede robar lo que no existe dentro del sistema.

%
% El grupo vacío después de «footnotemark» no es necesario por la sintaxis,
% sin embargo, si no se pone, LaTeX deja un espacio muy corto entre el número
% y la siguiente palabra. Extraño.
%
% https://tex.stackexchange.com/questions/42982/wrong-space-with-footnotemark
%

\footnotetext{
  El término \textit{tokenización} es un anglicismo y, en este documento,
  se refiere a la acción y efecto de \textit{tokenizar}; es decir, dado un
  valor, obtener un valor sustituto (\gls{gl:token}). La \textit{detokenización}
  es el proceso inverso: dado el valor sustituto (\gls{gl:token}), obtener el
  valor original. Estos préstamos lingüísticos se utilizan de manera constante
  a lo largo de este reporte.
}

Aunque en este trabajo se hace especial referencia a la tokenización de los
números de tarjetas, o \gls{gl:pan}, este proceso sirve para proteger otros
datos, como números de seguridad social, claves de registro, números de
teléfono, etcétera. De hecho, la tokenización no está limitada al mundo digital
y ha estado presente desde hace mucho tiempo, por ejemplo, los billetes y
monedas, o las fichas en un casino: uno deposita dinero y obtiene su
equivalente en fichas (este proceso es el de tokenización) para jugar; aunque
las fichas (o \glspl{gl:token}) están vigiladas, si alguien las roba, el casino
no pierde tanto (siempre y cuando no sean cambiadas por dinero, o sea, realizar
el proceso de detokenización), pues es una mera representación, y no el dinero
mismo. Lo mismo sucede con los \glspl{gl:token} digitales: sustituyen
información valiosa, por valores que carecen de significado y cuya pérdida
no representa un peligro inminente o una violación de la privacidad.

Todos los métodos para generar \glspl{gl:token} que se presentan en este trabajo
(y la gran mayoría de los métodos conocidos) competen enteramente a la
criptografía. Entre otras cosas, la criptografía permite proteger información de
terceros no autorizados, esto es, permite obtener confidencialidad. La idea
básica es transformar un mensaje de forma que solo el destinatario sea capaz
de hacer la transformación inversa y leer el mensaje original. Aunque con
métodos un poco distintos a los actuales, la criptografía tiene una larga
historia: el uso más antiguo del que se tiene noticia es en jeroglíficos
egipcios de alrededor de 1900 a. C.; a partir de ahí, hay evidencias de su uso
en muchas otras culturas antiguas~\cite{codebreakers}. La historia de la
criptografía moderna está muy relacionada con la historia de la computación: en
muchas ocasiones, la principal motivación para el desarrollo de máquinas más
potentes fue la posibilidad de crear un método de cifrado infalible (o del
método para romperlo); el ejemplo más claro de esto es el desarrollo de
métodos para descifrar los mensajes de los nazis, durante la segunda guerra
mundial, lo que permitió a Alan Turing sentar las bases de la computación
actual~\cite{simon_singht}. Esta dinámica de juego, en la cuál cada participante
está siempre buscando un método que los adversarios no pueden romper,
hace de la criptografía una ciencia en constante movimiento: una vez que
se encuentra una nueva vulnerabilidad en algún método, se trabaja para
corregirlo o para reemplazarlo.

Uno de los aspectos más importantes de las herramientas criptográficas
modernas es que buscan probar, dentro de los límites posibles, la
seguridad de un algoritmo. Es por esto que hoy en día no basta con
describir un nuevo método y esperar que sea lo suficientemente fuerte,
sino que la presentación de un método debe de ir acompañada de los
argumentos (las pruebas) que garantizan su seguridad. Hacer esto no es
tarea sencilla (si lo fuera, el juego ya estaría ganado para uno de los
bandos) pues involucra hacer suposiciones sobre las capacidades de
los adversarios y, basándose en esto, garantizar que los recursos necesarios
para romper el método probado se encuentran muy por encima de las capacidades
supuestas en un inicio. En este contexto, las capacidades de un adversario
se determinan por su poder de cómputo: ¿cuántas computadoras tiene a su
disposición?, ¿qué tan rápidas son?, etcétera.

Lo anterior nos lleva a otro principio muy importante de la criptografía
moderna: la seguridad se encuentra en los datos, no en el método. La historia
de la criptografía muestra que durante mucho tiempo, la seguridad de un método
consistía totalmente en mantenerlo en secreto: mientras los adversarios
no supieran cómo se estaba cifrando algo, los mensajes permanecerían seguros;
sin embargo, si el método se filtraba, todos los mensajes cifrados con ese
método se veían comprometidos. Este esquema presenta un serio problema,
pues hace dependiente de un solo secreto, el método, la seguridad de todos
los mensajes. Esto llevó a la introducción del concepto de la llave dentro
de la criptografía: un valor que solo conocen las entidades autorizadas a leer
el mensaje, sin el cuál no pueden descifrarlo; de esta forma la seguridad de un
mensaje se basa tanto en el conocimiento del método, como en el conocimiento
de la llave. En la mayoría de los métodos usados hoy en día, la seguridad se
basa enteramente en la llave, dejando el método abierto al escrutinio de todos.
Hacer esto presenta varias ventajas: una muestra de que el algoritmo está
bien diseñado y es lo suficientemente fuerte es que no necesita mantenerse en
secreto; cuando el algoritmo es público y es usado por muchas entidades (entre
más mejor) son mayores las posibilidades de encontrar vulnerabilidades y
el tiempo de respuesta de las correcciones es menor.

La presentación que se hace en este trabajo de los métodos de tokenización
tiene un enfoque totalmente criptográfico. De esta forma se busca demostrar
que la tokenización es una aplicación de la criptografía, no una alternativa
a esta. Es por esto que todo el capítulo del marco teórico está dedicado a los
mecanismos criptográficos que se relacionan con los métodos para generar
\glspl{gl:token}.

\subimport{/}{objetivos}
\subimport{/}{justificacion}
\subimport{/}{estructura}


\section{Cifrados por bloques}

%
% Sección de cifrado por bloques, capítulo de antecedentes.
% Proyecto Lovelace.
%

\section{Cifrados por bloques}

La información presentada a continuación puede consultarse con más profundidad
en las siguientes referencias
\cite{menezes, DBLP:series/isc/DelfsK07}

Los cifrados por bloque son esquemas de cifrado que, como bien
lo explica su nombre, operan mediante bloques de datos. Normalmente los
bloques tienen una longitud de 64 o de 128 bits, mientras que las llaves
pueden ser de 56, 128, 192 o 256 bits.

En muchos sistemas criptográficos, los cifrados por bloque simétricos
son elementos importantes, pues su versatilidad permite construir con
ellos generadores de números pseudoaleatorios, cifrados de flujo MACs y
funciones hash. Sirven también como componentes centrales en técnicas de
autenticación de mensajes, mecanismos de integridad de datos, protocolos
de autenticación de entidad y esquemas de firma electrónica que usan
llaves simétricas.

Los cifrados por bloque están limitados en la práctica por varios
factores, tales como el límite de memoria, la velocidad requerida o
restricciones impuestas por el hardware o el software en el que se
implementan. Normalmente, se debe escoger entre eficiencia y seguridad

Idealmente, al cifrar por bloques, cada bit del bloque cifrado depende
de todos los bits de la llave y del texto en claro; no debería existir
una relación estadística evidente entre el texto en claro y el texto
cifrado; el alterar tan solo un bit en el texto en claro o en la llave
debería alterar cada uno de los bits del texto cifrado con una
probabilidad de $\frac{1}{2}$; y alterar un bit del texto cifrado
debería provocar resultados impredecibles al recuperar el texto en
claro.

\subsection{Definición}

\begin{equation}
  \label{cifrado_bloques_def}
  \begin{aligned}
    E: {}& \{0,1\}^r \times \{0,1\}^n \longrightarrow \{0,1\}^n \\
         &(k,m) \longmapsto  E(k,m)
  \end{aligned}
\end{equation}

Utilizando una llave secreta $k$ de longitud binaria $r$ el algoritmo de
cifrado $E$ cifra bloques en claro $m$ de una longitud binaria fija $n$ y
da como resultado bloques cifrados $c = E (k,m)$ cuya longitud también es
$n$. $n$ es el tamaño de bloque del cifrado.
El espacio de llave está dado por $K = \{0,1\}^r$, para cada llave existe una
función $D_k(c)$ que permite tomar un bloque cifrado $c$ y regresarlo a su
forma original $m$.

Generalmente, los cifrados por bloque procesan el texto claro en bloques
relativamente grandes ($n \geq 64$), contrastando con los cifradores de
flujo, que toman bit por bit. Cuando la longitud del mensaje en claro excede
el tamaño de bloque, se utilizan los \glspl{gl:modo_de_operacion}.

Los parámetros más importantes de los cifrados por bloque son los
siguientes:
\begin{itemize}
  \item Tamaño de bloque
  \item Tamaño de llave
\end{itemize}

\subsection{Criterios para evaluar los cifrados por bloque}

A continuación se listan algunos de los criterios que pueden ser tomados
en cuenta para evaluar estos cifrados:
\begin{enumerate}
  \item \textbf{Nivel de seguridad.} La confianza que se le tiene a un
    cifrado va creciendo con el tiempo, pues va siendo analizado y
    sometido a pruebas.
  \item \textbf{Tamaño de llave.} La entropía del espacio de la llave
    define un límite superior en la seguridad del cifrado al tomar en
    cuenta la búsqueda exhaustiva. Sin embargo, hay que tener cuidado
    con su tamaño, pues también aumentan los costos de generación,
    transmisión, almacenamiento, etcétera.
  \item \textbf{Tamaño de bloque.} Impacta la seguridad, pues entre más
    grandes, mejor; sin embargo, tiene repercusiones en el costo de la
    implementación, además de que puede afectar el rendimiento del
    cifrado.
  \item \textbf{Expansión de datos.} Es extremadamente deseable que los
    datos cifrados no aumenten su tamaño respecto a los datos en claro.
  \item \textbf{Propagación de error.} Descifrar datos que contienen
    errores de bit puede llevar a recuperar incorrectamente el texto en
    claro, además de propagar errores en los bloques pendientes por
    descifrar. Normalmente, el tamaño de bloque afecta el error de
    propagación.
\end{enumerate}

A continuación se listan algunos algoritmos de cifrado por bloques.

%
% Explicación sobre redes Feistel, presentación de FPE.
% Proyecto Lovelace.
%

\subsection{Redes Feistel}

\begin{frame}{FFX}{Redes Feistel}

  \only<1>
  {
    \begin{figure}[H]
      \begin{center}
        \includegraphics[height=0.7\textheight]
          {../../../diagramas_comunes/redes_feistel/feistel.png}
        \caption{Red Feistel, versión original.}
      \end{center}
    \end{figure}
  }

  \note<1>
  {
    Una de las características más importantes es que no se necesita una
    función de ronda inversa. La operación de descifrado se obtiene
    despejando, solamente. Por ejemplo, en una última instancia solamente se
    conoce $ L_n $ y $ R_n $:

    $$ R_{n-1} = L_n $$
    $$ L_{n-1} = f_{k_n}(R_{n-1}) \oplus R_n $$
  }

  \only<2>
  {
    \begin{figure}[H]
      \centering
      \begin{subfigure}{0.45\textwidth}
        \begin{center}
          \includegraphics[height=0.65\textheight]
            {../../../diagramas_comunes/redes_feistel/desbalanceadas.png}
          \caption{Redes desbalanceadas.}
        \end{center}
      \end{subfigure}
      \begin{subfigure}{0.45\textwidth}
        \begin{center}
          \includegraphics[height=0.65\textheight]
            {../../../diagramas_comunes/redes_feistel/alternantes.png}
          \caption{Redes alternantes.}
        \end{center}
      \end{subfigure}
      \caption{Generalizaciones de las redes Feistel.}
    \end{figure}
  }

  \note<2>
  {
    Puede haber distintos grados de desbalanceo. En caso de que este sea igual
    a cero, la red está balanceada (versión original).

    Las desbalanceadas llevan el costo extra de hacer la partición del bloque
    en cada ronda. Las alternantes necesitan de dos funciones de ronda.

    Después de terminar explicación, regresar a 1 y poner ejemplo con
    alfabeto binario y con alfabeto decimal. FFSEM estaba pensado para
    alfabetos binarios solamente, por lo que había que utilizar caminata
    cíclica en la función de ronda.
  }

  \only<3>
  {
    La operación de una red Feistel desbalanceada se puede resumir en las
    siguientes ecuaciones:
    \begin{align*}
      L_{i} &= R_{i - 1} \\
      R_{i} &= F_k(R_{i - 1}) \oplus L_{i - 1}
    \end{align*}

    Estas son las mismas que las de una red Feistel balanceada, con el costo
    extra de que en cada iteración hay que estar redistribuyendo los bloques.
  }

  \note<3>
  {
    El problema de estas es que hay que estar concatenando y volviendo a partir
    en cada iteración.
  }

  \only<4>
  {
    La operación de una red Feistel alternante se puede resumir en las
    siguientes ecuaciones:

    Si la ronda es par:
    \begin{align*}
      L_{i} &= F^1_k(R_{i - 1}) \oplus L_{i - 1} \\
      R_{i} &= R_{i - 1}
    \end{align*}

    Si la ronda es impar:
    \begin{align*}
      R_{i} &= F^2_k(L_{i - 1}) \oplus R_{i - 1} \\
      L_{i} &= L_{i - 1}
    \end{align*}
  }

  \note<4>
  {
    La implementación de estas es mucho más sencilla: no hay que hacer
    reparticiones en cada ronda; el proceso de cifrado y el de descifrado es
    prácticamente el mismo.
  }

\end{frame}

\begin{frame}{FFX}{Función de ronda}

  \only<1>
  {
    La función de ronda propuesta en \cite{ffx_2} es AES CBC MAC, aunque también
    puede ser usado cualquier otro cifrador por bloques o función hash.

    \begin{figure}[H]
      \begin{center}
        \includegraphics[height=0.35\textheight]{diagramas/cbc_mac.png}
        \caption{CBC MAC.}
      \end{center}
    \end{figure}

    La salida de la primitiva utilizada determina el tamaño del espacio de
    mensajes aceptado.
  }

  \note<1>
  {
    Esta última característica es lo que permite que FFX funciones sobre
    cadenas de cualquier longitud. Por ejemplo, se puede poner un cifrador
    de flujo en el lugar de la función de ronda.
  }

  \only<2>
  {
    La salida de la función de ronda se debe adaptar al alfabeto usado:

    \begin{itemize}

      \item En el caso de un alfabeto binario, tomar solamente el número
        de bits que la red Feistel requiere.

      \item En el caso de un alfabeto de caracteres, se debe interpretar de
        manera que produzca el número de caracteres necesarios. La forma más
        simple para hacer esto es tomar la salida de la primitiva módulo
        $ m^n $, en donde $ m $ es la cardinalidad del alfabeto y $ n $ el
        número de caracteres ocupados por la red.

        En \cite{ffx_2} se propone partir la salida de CBC MAC en dos: usar
        la primera mitad para producir $ n/2 $ caracteres, y la segunda
        mitad para los restantes.

    \end{itemize}
  }

  \note<2>
  {
    \textbf{TODO:} En el caso de los alfabetos de caracteres, ¿cuál es la
    diferencia en cuestión de seguridad entre uno y otro esquema? Creo que,
    el segundo es mejor, dado que toma en cuenta más bits de la salida de
    la primitiva.

    Antes de pasar a la siguiente sección, hablar de los \textit{tweaks} y
    de los números de rondas recomendados.
  }

\end{frame}

\subsection{Data Encryption Standard (DES)}

Este es, probablemente, el cifrado simétrico por bloques más conocido; 
ya que en la década de los 70 estableció un precedente al ser el primer 
algoritmo a nivel comercial que publicó abiertamente sus 
especificaciones y detalles de implementación. Se encuentra definido
en el estándar americano FIPS 46-2.

DES es un cifrado Feistel que procesa bloques de $n=64$ bits y produce
bloques cifrados de la misma longitud. Aunque la llave es de 64 bits, 8
son de paridad, por lo que el tamaño \textit{efectivo} de la llave es de
56 bits. Las $2^{56}$ llaves implementan, máximo, $2^{56}$ de las 
$2^{64}!$ posibles biyecciones en bloques de 64 bits.

Con la llave $K$ se generan 16 subllaves $K_i$ de 48 bits; una para cada 
ronda. En cada ronda se utilizan 8 \textit{cajas-s} (mapeos de 
sustitución de 6 a 4 bits). La entrada de 64 bits es dividida por la mitad
en $L_0$ y $R_0$. Cada ronda $i$ va tomando las entradas $L_{i-1}$ y 
$R_{i-1}$ de la ronda anterior y produce salidas de 32 bits $L_i$ y $R_i$
mientras $1 \leq i \leq 16$ de la siguiente manera:

\begin{equation}
  \label{cifrado_des_li}
 	L_i = R_{i-1}
\end{equation}

\begin{equation}
  \label{cifrado_des_di}
 	R_i = L_{i-1} \oplus f(R_{i-1}, K_i); f(R_{i-1}, K_i) = P(S(E(R_{i-1})\oplus K_i))
\end{equation}

$E$ se encarga de expandir $R_{i-1}$ de 32 bits a 48, $P$ es una 
permutación de 32 bits y $S$ son las cajas-s. 

\begin{pseudocodigo}[caption={DES, cifrado.}, label={des:1}]
  entrada:	64 bits de texto en claro $M = m_1 \dots m_64$; llave de 64 bits $K = k_1 \dots k_{64}$.
  salida: 	bloque de texto cifrado de 64 bits $C = c_1 \dots c_{64}$.
  inicio
    Calcular 16 subllaves $K_i$ de 48 bits partiendo de $K$.
    $(L_0, R_0) \leftarrow IP(m_1m_2\dots m_{64}$
    Para $i$ desde 1 hasta 16: Calcular $L_i$ y $R_i$ mediante las ecuaciones mostradas anteriormente. 
    Obtener $f(R_{i-1}, K_i)$ así:
    	a) Expandir $R_{i-1} = r_1r_2\dots r_{32}$ de 32 a 48 bits usando $E$: $T \leftarrow E(R_{i-1})$.
    	b) $T^\prime \leftarrow T \oplus K_i$. Donde $T^\prime$ es representado como ocho cadenas de 6 bits cada una $(B_1, \dots, B_8)$.
    	c) $T'' \leftarrow (S_1(B_1), S_2(B_2), \dots S_8(B_8))$
    	d) $T''' \leftarrow P(T'')$
    fin
    $b_1b_2 \dots b_{64} \leftarrow (R_{16}, L{16})$.
    $C \leftarrow IP^{-1}(b_1b_2 \dots b_{64})$
  fin
\end{pseudocodigo}

El descifrado DES consiste en el mismo algoritmo de cifrado, con la 
misma llave $K$, pero utilizando las subllaves en orden inverso:
$K_{16}, K{15}, \dots, K_1$.

\subsubsection{Llaves débiles}
Tomando en cuenta las siguientes definiciones

\begin{itemize}
	\item Llave débil: una llave $K$ tal que $E_K(E_K(x)) = x$ para toda 
		$x$.
	\item Llaves semidébiles: se tiene un par de llaves $K_1, K_2$ tal que
		$E_{K_1}(E_{K_2}(x)) = x$
\end{itemize}

DES tiene cuatro llaves débiles y seis pares de llaves semidébiles.
Las cuatro llaves débiles generan subllaves $K_i$ iguales y, debido a 
que DES es un cifrado Feistel, el cifrado es autorreversible. O sea que 
al final se obtiene de nuevo el texto en claro, pues cifrar dos veces
con la misma llave regresa la entrada original.
Respecto a los pares semidébiles, el cifrado con una de las llaves del 
par es equivalente al descifrado  con la otra (o viceversa). 

%
% Explicación de AES, capítulo de antecedentes.
% Proyecto Lovelace.
%

\subsection{Advanced Encryption Standard (AES)}
\label{sec:aes}

Dado que el tamaño de bloque y la longitud de la llave de \acrshort{gl:des}
se volvieron muy pequeños para resistir los embates del progreso
de la tecnología, el \acrshort{gl:nist} comenzó la búsqueda de un nuevo cifrado
estándar en 1997; este cifrado debía tener un tamaño de bloque de,
al menos, 128 bits y soportar tres tamaños de llave: 128, 192 y 256 bits.

Después de pasar por un proceso de selección, la propuesta Rijndael fue
seleccionada. Se le hicieron algunas modificaciones, pues Rijndael soporta
combinaciones de llaves y bloques de longitud 128, 169, 192, 224 y 256;
mientras que \acrshort{gl:aes} tiene fijo el tamaño de bloque y solo utiliza
los tres tamaños de llave mencionados anteriormente. Dependiendo del tamaño
de la llave, se tiene el número de \glspl{gl:ronda}: 10 para las de 128 bits,
12 para las de 192 y 14 para las de 256.

El cifrado requiere de una matriz de $4 \times 4$ denominada matriz de
estado.

\begin{pseudocodigo}[caption={AES, cifrado.}, label={aes:1}]
  entrada:    128 bits de texto en claro $M$; llave de $n$ bits $K$.
  salida:     bloque de texto cifrado de 64 bits $C = c_1 \dots c_{64}$.
  inicio
    Obtener las subllaves de 128 bits necesarias: una para cada ronda y una extra.
    Iniciar matriz de estado con el bloque en claro.
    Realizar $AddRoundKey(matriz\_estado, k_0)$
    para_todo $i$ desde 1 hasta $num\_rondas-1$:
      $SubBytes(matriz\_estado)$
      $ShiftRows(matriz\_estado)$
      $MixColumns(matriz\_estado)$
      $AddRoundKey(matriz\_estado, k_i)$
    fin
    $SubBytes(matriz\_estado)$
    $ShiftRows(matriz\_estado)$
    $MixColumns(matriz\_estado)$
    $AddRoundKey(matriz\_estado, k_{num\_rondas})$
    regresa $matriz\_estado$
  fin
\end{pseudocodigo}

Como todos los pasos realizados en las \glspl{gl:ronda} son invertibles, el
proceso de descifrado consiste en aplicar las funciones inversas a
$SubBytes$, $ShiftRows$, $MixColumns$ y $AddRoundKey$ en el orden
opuesto. Tanto el algoritmo como sus pasos están pensados con bytes. 
En el algoritmo Rijndael los bytes son considerados como elementos del
campo finito $\mathbb{F}_{2^8}$ con ${2^8}$ elementos; $\mathbb{F}_{2^8}$
es construido como una extensión del campo  $\mathbb{F}_{2}$ con 2 elementos
mediante el uso del polinomio irreducible $X^8+X^4+X^3+X+1$.
Por lo tanto, las operaciones que se hagan a continuación de adición y el
producto entre bytes significa sumarlos y multipilcarlos como elementos del 
campo  $\mathbb{F}_{2^8}$.

\subsubsection{SubBytes}
Esta es la única transformación no lineal de Rijndael. Sustituye
los bytes de la matriz de estado byte a byte al aplicar la función
$S_{RD}$ a cada elemento de la matriz. La función $S_{RD}$ es también
conocida como Caja-S y no depende de la llave. La misma caja es utilizada
para los bytes en todas las posiciones.

\begin{figure}[H]
  \begin{center}
    \includegraphics[width=0.6\linewidth]
      {contenidos/antecedentes/bloques/diagramas/subBytes}
     \caption{Diagrama de la operación $SubBytes$.}
   \end{center}
\end{figure}

\subsubsection{ShiftRows}
Esta transformación hace un corrimiento cíclico hacia la izquierda de las
filas de la matriz de estado. Los desplazamientos son distintos para cada
fila y dependen de la longitud del bloque ($N_b$).

\begin{figure}[H]
  \begin{center}
    \includegraphics[width=0.6\linewidth]
      {contenidos/antecedentes/bloques/diagramas/shiftRows}
     \caption{Diagrama de la operación $ShiftRows$.}
   \end{center}
\end{figure}

\subsubsection{MixColumns}
Esta transformación opera en cada columna de la matriz de estado
independientemente. Se considera una columna $a = (a_0, a_1, a_2, a_3)$
como el polinomio $a(X) = a_3X^3 + a_2X^2 + a_1X + a_0$.
Entonces este paso transforma una columna $a$ al multiplicarla con el
siguiente polinomio fijo:
\begin{equation}
  \label{cifrado_aes_poli}
  c(X) = 03X^3 + 01X^2 + 01X+ 02
\end{equation}
y se toma el residuo del producto módulo $X^4+1$:
\begin{equation}
  \label{cifrado_aes_mix}
  a(X) \mapsto a(X) \cdotp c(X) \mod (X^4+1)
\end{equation}

\begin{figure}[H]
  \begin{center}
    \includegraphics[width=0.6\linewidth]
      {contenidos/antecedentes/bloques/diagramas/mixColumns}
     \caption{Diagrama de la operación $MixColumns$.}
   \end{center}
\end{figure}

\subsubsection{AddRoundKey}
Esta es la única operación que depende de la llave secreta $k$. Añade una
llave de ronda para intervenir en el resultado de la matriz de estado.
Las llaves de ronda son derivadas de la llave secreta $k$ al aplicar el
algoritmo de generación de llaves. Las llaves de ronda tienen la misma
longitud que los bloques. Esta operación es simplemente una operación
$XOR$ bit a bit de la matriz de estado con la llave de ronda en turno.
Para obtener el nuevo valor de la matriz de estado se realiza lo
siguiente:
\begin{equation}
  \label{cifrado_aes_addkey}
  (matriz\_estado, k_i) \mapsto matriz\_estado \oplus k_i
\end{equation}

Como se tiene una matriz\_estado, la llave de ronda ($k_i$) también
es representada como una matriz de bytes con 4 columnas y $N_b$ columnas.
Cada una de las $N_b$ palabras de la llave de ronda corresponde a una
columna. Entonces se realiza la operación $XOR$ bit a bit sobre las
entradas correspondientes de la matriz de estado y la matriz de la llave
de ronda.

\begin{figure}[H]
  \begin{center}
    \includegraphics[width=0.6\linewidth]
      {contenidos/antecedentes/bloques/diagramas/addRoundKey}
     \caption{Diagrama de la operación $AddRoundKey$.}
   \end{center}
\end{figure}

Esta operación, claro está, es invertible: basta con aplicar la misma
operación con la misma llave para revertir el efecto.

\subsection{Fast Data Encipherment Algorithm (FEAL)}

Es una familia de algoritmos que ha tenido una participación crítica
en el desarrollo y refinamiento de varias técnicas del criptoanálisis, 
tales como el criptoanálisis lineal y diferencial. FEAL-N mapea
bloques de texto en claro de 64 bits a bloques de 64 bits de texto
cifrado mediante una llave secreta de 64 bits. Es un cifrado Feistel de 
$n-$rondas parecido a DES, pero con una función $f$ más simple.

FEAL fue diseñado para ser veloz y simple, especialmente para 
microprocesadores de 8 bits: usa operaciones orientadas a bytes, evita
el uso de permutaciones de bit y tablas de consulta. La versión inicial
de cuatro rondas (FEAL-4), propuesto como una alternativa rápida a DES, 
fue encontrado mucho más inseguro de lo planeado; por lo que se propuso
realizar más rondas (FEAL-16 y FEAL-32) para compensar y ofrecer un nivel
de seguridad parecido a DES; sin embargo, el rendimiento se ve fuertemente
afectado mientras el número de rondas aumenta; y, mientras DES puede 
mejorar su velocidad con tablas de consulta, resulta más complicado 
para FEAL.

\begin{pseudocodigo}[caption={FEAL-8, cifrado.}, label={feal8:1}]
  entrada:	64 bits de texto en claro $M = m_1 \dots m_64$; llave de 64 bits $K = k_1 \dots k_64$.
  salida: 	bloque de texto cifrado de 64 bits $C = c_1 \dots c_64$.
  inicio
    Calcular 16 subllaves de 16 bits para $K$.
    Definir $M_L = m_1 \dots m_32; M_R = m_33 \dots m_64$.
    $(L_0, R_0) \leftarrow (M_L, M_R) \oplus ((K_8, K_9), (K_10, K_11))$
    $R_0 \leftarrow R_0 \oplus L_0$.
    Para $i$ desde 1 hasta 8:
      $L_i \leftarrow R_{i-1}$
      $R_i \leftarrow L_{i-1} \oplus f(R_{i-1}, K_{i-1})$
    fin
    $L_8 \leftarrow L_8 \oplus R_8$
    $(R_8, L_8) \leftarrow (R_8, L_8) \oplus ((K_12, K_13), (K_14, K_15))$
    $C \leftarrow (R_8, L_8)$.
  fin
\end{pseudocodigo}

Para descifrar se utiliza el mismo algoritmo, con la misma llave $K$ y el
texto cifrado $C = (R_8, L_8)$ se utiliza como la entrada $M$; sin 
embargo, la generación de llaves se hace al revés: las subllaves 
$((K_12, K_13), (K_14, K_15))$ se utilizan para la $\oplus$ inicial, las 
$((K_8, K_9), (K_10, K_11))$ para la $\oplus$ final y en las rondas se 
utiliza de la subllave $K_7$ a la $K_0$.

FEAL con una llave de 64 bits puede ser generalizado a $N-$ rondas con 
$N$ par, aunque se recomienda $N = 2^x$.
%
% Explicación de IDEA, capítulo de antecedentes.
% Proyecto Lovelace.
%

\subsection{International Data Encryption Algorithm (IDEA)}

Cifra bloques de 64 bits utilizando una llave de 128 bits. Este cifrado
está basado en una generalización de la estructura Feistel y consiste en
8 \glspl{gl:ronda} idénticas seguidas por una transformación. Cada ronda $r$
utiliza 6 subllaves $K^{(r)}_i$ ($1 \leq i \leq 6$) de 16 bits que se
encargan de transformar una entrada $X$ de 64 bits en una salida de
cuatro bloques de 16-bits, que son utilizados como entrada en la
siguiente ronda. La salida de la ronda 8 tiene como entrada la
transformación de salida que, al emplear cuatro llaves adicionales
$K^{(9)}_i$ ($1 \leq i \leq 4$), produce los datos cifrados
$Y = (Y_1, Y_2, Y_3, Y_4)$.

%   Lo siento, pero si corto la línea de la entrada, la entrada queda a
%  la mitad y se ve muy raro.

\begin{pseudocodigo}[caption={IDEA, cifrado.}, label={idea:1}]
  entrada:   $64-$bits de datos en claro $M = m_1 \dots m_{64}$;
             llave de $128-$bits $ K = k_1 \dots k_{128}$.
  salida:    bloque cifrado de $64-$bits $Y = (Y_1, Y_2, Y_3, Y_4)$.
  inicio
    Calcular las subllaves $K^{(r)}_1, \dots, K^{(r)}_{6}$ para las rondas $1 \leq r \leq 8$ y $K^{(9)}_1, \dots, K^{(9)}_{4}$
    para la transformación de salida.
    $(X_1, X_2, X_3, X_4) \leftarrow (m_1 \dots m_{16}, m_{17} \dots m_{32}, m_{33} \dots m_{48}, m_{49} \dots m_{64})$
        donde $X_i$ almacena 16 bits.
    para_todo $r$ desde 1 hasta 8:
      a) $X_1 \leftarrow X_1 \times K_1^{(r)} \mod2^{16} + 1$
         $X_4 \leftarrow X_4 \times K_4^{(r)} \mod2^{16} + 1$
         $X_2 \leftarrow X_2 + K_2^{(r)} \mod2^{16}$
         $X_3 \leftarrow X_3 + K_3^{(r)} \mod2^{16}$
      b) $t_0 \leftarrow K_5^{(r)} \times (X_1 \oplus X_3) \mod2^{16} + 1$
         $t_1 \leftarrow K_6^{(r)} \times (t_0 + (X_2 \oplus X_4)) \mod2^{16} + 1$
         $t_2 \leftarrow t_0 + t_1 \mod2^{16}$
      c) $X_1 \leftarrow X_1 \oplus t_1$
         $X_4 \leftarrow X_4 \oplus t_2$
         $a \leftarrow X_2 \oplus t_2$
         $X_2 \leftarrow X_3 \oplus t_1$
         $X_3 \leftarrow a$
    fin
    Realizar la transformación de salida:
      $Y_1 \leftarrow X_1 \times K_1^{(9)} \mod2^{16} + 1$
      $Y_4 \leftarrow X_4 \times K_4^{(9)} \mod2^{16} + 1$
      $Y_2 \leftarrow X_3 + K_2^{(9)} \mod2^{16}$
      $Y_3 \leftarrow X_2 + K_3^{(9)} \mod2^{16}$
  fin
\end{pseudocodigo}

El descifrado se realiza con el mismo algoritmo de cifrado, pero
utilizando como entrada los datos cifrados $Y$ como entrada $M$. Se usa la
misma llave $K$; aunque las subllaves sufren una modificación al ser
generadas, pues se utiliza una tabla y se realizan las operaciones
contrarias (inverso de la adición y el inverso del producto).

Descartando los ataques a las llaves débiles, no hay un mejor ataque
publicado para el \gls{gl:idea} de 8 \glspl{gl:ronda} que el de la búsqueda
exhaustiva en el espacio de llave. Por lo que la seguridad está ligada a la
creciente debilidad de su tamaño de bloque relativamente pequeño.

%
% Explicación de SAFER, capítulo de antecedentes.
% Proyecto Lovelace.
%

\subsection{Secure And Fast Encryption Routine (SAFER)}

El cifrado \gls{gl:safer} K-64 es un cifrado por bloques de 64 bits
iterativo. Consiste en $r$ \glspl{gl:ronda} idénticas seguidas por una
transformación. Originalmente se recomendaban $6$ \glspl{gl:ronda} seguidas,
sin embargo, ahora se utiliza una generación de claves ligeramente modificada y
el uso de $8$ \glspl{gl:ronda} (máximo 10). Ambas generaciones de llaves
expanden la llave de 64 bits en $2r+1$ subllaves, cada una de 64 bits (dos por
cada ronda y una más para la transformación de salida).

Este cifrado consiste completamente en operaciones de bytes, por lo que
es adecuado para procesadores con tamaños de palabra pequeños, como los
chips de tarjetas.

%   Lo siento, pero si corto la línea de la entrada, la entrada queda a
%  la mitad y se ve muy raro.

\begin{pseudocodigo}[caption={SAFER K-64, cifrado.}, label={safer:1}]
  entrada: $r, 6\leq r \leq10$; $64-$bits de datos en claro $M = m_1 \dots m_{64}$; $ K = k_1 \dots k_{64}$.
  salida: bloque cifrado de $64-$bits $Y = (Y_1, \dots, Y_8)$.
  inicio
    Calcular las subllaves $K_1, \dots, K_{2r+1}$
    $(X_1, X_2, \dots X_8) \leftarrow (m_1 \dots m_8, m_9 \dots m_{16}, \dots, m_{57} \dots m_{64})$
    para_todo $i$ desde $1$ hasta $r$:
      a) Para $j = 1, 4, 5, 8: X_j \leftarrow X_j \oplus K_{2i-1}[j]$
        Para $j = 2, 3, 6, 7: X_j \leftarrow X_j + K_{2i-1}[j]$$mod$ $2^8$
      b) Para $j = 1, 4, 5, 8: X_j \leftarrow S$[$X_j$]
        Para $j = 2, 3, 6, 7: X_j \leftarrow S_{inversa}X_j$
      c) Para $j = 1, 4, 5, 8: X_j \leftarrow X_j + K_{2i}[j]$$mod$ $2^8$
        Para $j = 2, 3, 6, 7: X_j \leftarrow X_j \oplus K_{2i}[j]$
      d) Para $j = 1, 3, 5, 7: (X_j, X_{j+1}) \leftarrow f(X_j, X_{j+1})$.
      e) $(Y_1, Y_2 ) \leftarrow f(X_1, X_3), (Y_3, Y_4 ) \leftarrow f(X_5, X_7)$,
        $(Y_5, Y_6 ) \leftarrow f(X_2, X_4), (Y_7, Y_8 ) \leftarrow f(X_6, X_8 )$.
        Para $j$ desde 1 hasta 8: $X_j \leftarrow Y_j$
      f) $(Y_1, Y_2) \leftarrow f(X_1, X_3), (Y_3, Y_4) \leftarrow f(X_5, X_7)$,
        $(Y_5, Y_6 ) \leftarrow f(X_2, X_4), (Y_7, Y_8) \leftarrow f(X_6, X_8)$.
        Para $j$ desde 1 hasta 8: $X_j \leftarrow Y_j$.
    fin
    Para $j = 1, 4, 5, 8: Y_j \leftarrow X_j \oplus K_{2r+1}[j]$.
    Para $j = 2, 3, 6, 7: Y_j \leftarrow X_j + K_{2r+1} [j] \mod 2^8$.
  fin
\end{pseudocodigo}

Para descifrar, se utiliza la misma llave $K$ y las subllaves $K_i$
que fueron utilizadas al cifrar. Cada paso del cifrado se hace en orden
inverso, del último al primero; comenzando con una transformación de entrada
utilizando la llave $K_{2r+1}$ para deshacer la transformación de salida, se
sigue con las \glspl{gl:ronda} de descifrado utilizando las llaves de $K_{2r}$
a $K_1$, invirtiendo los pasos cada ronda.

\subsection{RC5}
Este cifrado por bloques tiene una arquitectura orientada a palabras (ya
sea $w = 16, 32, 64$bits) y tiene una descripción muy compacta adecuada
tanto para hardware como para software. Tanto la longitud $b$ de la
llave y el número de \glspl{gl:ronda} $r$ es variable; aunque se recomiendan 12
\glspl{gl:ronda} para 32 bits y 16 para cuando se tienen palabras de 64.

\begin{pseudocodigo}[caption={RC5, cifrado.}, label={rc5:1}]
  entrada:  $2w-$bits de datos en claro $M = (A, B)$; $r$;
      llave $K$ = $K$[0]$\dots K$[$b-1$]
  salida:   $2w-$bits de datos cifrados $C$.
  inicio
    Calcular $2r + 2$ subllaves $K_0, \dots, K_{2r+1}$
    $A \leftarrow A + K_0 \mod2^w, B \leftarrow B + K_1 \mod2^w$
    para_todo $i$ desde $1$ hasta $r$:
      $A \leftarrow ((A \oplus B) \hookleftarrow B) + K_{2i} \mod2^w$
      $B \leftarrow ((B \oplus A) \hookleftarrow A) + K_{2i+1} \mod2^w$
    fin
    Regresar $C \leftarrow (A,B)$
  fin
\end{pseudocodigo}

Para descifrar, RC5 utiliza el siguiente algoritmo.
\begin{pseudocodigo}[caption={RC5, descifrado.}, label={rc5:2}]
  entrada:  $2w-$bits de datos cifrados $C = (A, B)$; $r$;
      llave $K$ = $K$[0]$\dots K$[$b-1$]
  salida:   $2w-$bits de datos en claro $M$.
  inicio
    Calcular $2r + 2$ subllaves $K_0, \dots, K_{2r+1}$
    $A \leftarrow A + K_0 \mod2^w, B \leftarrow B + K_1 \mod2^w$
    Para $i$ desde $r$ hasta $1$:
      $B \leftarrow ((B - K_{2i+1} \mod2^w) \hookrightarrow A) \oplus A$
      $A \leftarrow ((A - K_{2i} \mod2^w) \hookrightarrow B) \oplus B$
    fin
    Regresar $M \leftarrow (A-K_0 \mod2^w, B-K_1 \mod2^w)$
  fin
\end{pseudocodigo}

%
% Sección de modos de operación, capítulo de antecedentes.
% Proyecto Lovelace.
%

\subsection{Modos de operación}
\label{sec:modos}

La información que aquí se presenta se puede consultar a mayor detalle en
\cite{modos_de_operacion} y~\cite{menezes}.

Por sí solos, los cifrados por bloques solamente permiten el cifrado y
descifrado de bloques de información de tamaño fijo; donde, en la mayoría de
los casos, los bloques son de menos de 256 bits, lo cual es equivalente a
alrededor de 8 caracteres. Es fácil darse cuenta de que esta restricción no es
ningún tema menor: en la gran mayoría de las aplicaciones, la longitud de lo
que se quiere ocultar es arbitraria.

Los \glspl{gl:modo_de_operacion} permiten extender la funcionalidad de los
cifrados por bloques para poder aplicarlos a información de tamaño irrestricto:
reciben el texto original (de tamaño arbitrario) y lo cifran, ocupando en el
proceso un cifrado por bloques.

Un primer enfoque (y quizás el más intuitivo) es partir el mensaje original
en bloques del tamaño requerido y después aplicar el algoritmo a cada bloque
por separado; en caso de que la longitud del mensaje no sea múltiplo del
tamaño de bloque, se puede agregar información extra al último bloque para
completar el tamaño requerido. Este es, de hecho, uno de los modos de operación
que existen, el \gls{gl:ecb}; su uso no es recomendado, pues es muy inseguro
cuando el mensaje original es simétrico a nivel de bloque.

A pesar de que existen más modos de operación, como \gls{gl:cfb}
u \gls{gl:ofb}, en este trabajo solo se describirá el funcionamiento de
\gls{gl:ecb}, \gls{gl:cbc} y \gls{gl:ctr}, dado que son los modos de operación
necesarios para el proceso de tokenización.

% También se enlistan otros tres modos, los cuales junto con
% \gls{gl:ecb}, son los más comunes.

\subimport{/}{ecb}
\subimport{/}{cbc}
\subimport{/}{ctr}
% \subimport{/}{cfb}
% \subimport{/}{ofb}



\section{Cifrados de flujo}

%
% Sección de modos de operación, capítulo de antecedentes.
% Proyecto Lovelace.
%

\subsection{Modos de operación}
\label{sec:modos}

La información que aquí se presenta se puede consultar a mayor detalle en
\cite{modos_de_operacion} y~\cite{menezes}.

Por sí solos, los cifrados por bloques solamente permiten el cifrado y
descifrado de bloques de información de tamaño fijo; donde, en la mayoría de
los casos, los bloques son de menos de 256 bits, lo cual es equivalente a
alrededor de 8 caracteres. Es fácil darse cuenta de que esta restricción no es
ningún tema menor: en la gran mayoría de las aplicaciones, la longitud de lo
que se quiere ocultar es arbitraria.

Los \glspl{gl:modo_de_operacion} permiten extender la funcionalidad de los
cifrados por bloques para poder aplicarlos a información de tamaño irrestricto:
reciben el texto original (de tamaño arbitrario) y lo cifran, ocupando en el
proceso un cifrado por bloques.

Un primer enfoque (y quizás el más intuitivo) es partir el mensaje original
en bloques del tamaño requerido y después aplicar el algoritmo a cada bloque
por separado; en caso de que la longitud del mensaje no sea múltiplo del
tamaño de bloque, se puede agregar información extra al último bloque para
completar el tamaño requerido. Este es, de hecho, uno de los modos de operación
que existen, el \gls{gl:ecb}; su uso no es recomendado, pues es muy inseguro
cuando el mensaje original es simétrico a nivel de bloque.

A pesar de que existen más modos de operación, como \gls{gl:cfb}
u \gls{gl:ofb}, en este trabajo solo se describirá el funcionamiento de
\gls{gl:ecb}, \gls{gl:cbc} y \gls{gl:ctr}, dado que son los modos de operación
necesarios para el proceso de tokenización.

% También se enlistan otros tres modos, los cuales junto con
% \gls{gl:ecb}, son los más comunes.

\subimport{/}{ecb}
\subimport{/}{cbc}
\subimport{/}{ctr}
% \subimport{/}{cfb}
% \subimport{/}{ofb}


\section{Funciones hash}

%
% Sección de funciones hash, capítulo de antecedentes.
% Proyecto Lovelace.
%

\section{Funciones hash}
\label{sec:hash}

La información presentada a continuación puede consultarse con más profundidad
en las siguientes referencias
\cite{hash_hussein, menezes, DBLP:series/isc/DelfsK07, hash_gupta}.

Se refiere al conjunto de funciones computacionalmente eficientes que
mapean cadenas binarias de una longitud arbitraria a cadenas binarias
de una longitud fija, llamadas valores hash.

\begin{figure}
  \begin{center}
    \subimport{diagramas/}{hash.tikz.tex}
    \caption{Diagrama del funcionamiento de una función hash.}
   \end{center}
\end{figure}

Matemáticamente, una función hash es una función
\begin{equation}
  \label{funcion_hash_def}
  \begin{split}
    h: \{0, 1\}^* \longrightarrow \{0,1\}^n \\
    m \longmapsto h(m)
  \end{split}
\end{equation}
La longitud de $n$ suele ser entre 128 y 512 bits. Las funciones hash
$h$ tienen las siguientes propiedades:

\begin{enumerate}
  \item Compresión: $h$ mapea una entrada $x$ (cuya longitud
    finita es arbitraria) a una salida $h(x)$ de longitud fija $n$.
  \item Facilidad de cómputo: dada $x$ y $h$, $h(x)$ es
    calculada ya sea sin necesitar mucho espacio, tiempo de cómputo, o
    requiere pocas operaciones, etcétera.
\end{enumerate}

De manera general, las funciones hash se pueden dividir en dos
categorías: las que no utilizan llave y su único parámetro es la entrada
$x$, y las que necesitan una llave secreta $k$ y la entrada $x$.

Sea una función hash sin llave $h$ con entradas $x$, $x^\prime$ y
salidas $y$ y $y^\prime$, respectivamente. A continuación se listan
algunas de las propiedades que puede tener:

\begin{enumerate}
  \item Resistencia de \gls{gl:preimagen}: no es computacionalmente factible
    para una salida específica $y$ encontrar una entrada $x^\prime$ que
    dé como resultado el mismo valor hash $h(x^\prime) = y$ si no se
    conoce $x$. Esta propiedad también es llamada
    \textit{de un sentido}.
  \item Resistencia de segunda \gls{gl:preimagen}: no es computacionalmente
    factible encontrar una segunda entrada $x^\prime$  que tenga la
    misma salida que una entrada específica $x$: $x \neq x^\prime$
    tal que $h(x) = h(x^\prime)$. Esta propiedad también es conocida
    como \textit{de débil resistencia a colisiones}.
  \item Resistencia a las colisiones: no es computacionalmente factible
    encontrar dos entradas distintas $x$, $x^\prime$ que lleven al
    mismo valor hash, o sea, $h(x) = h(x^\prime)$. A diferencia de la
    anterior, la selección de ambas entradas no está restringida. Esta
    propiedad también es conocida como
    \textit{de gran resistencia a colisiones}.
\end{enumerate}

Una función hash $h$ que cumple con las propiedades de resistencia de
\gls{gl:preimagen} y resistencia de segunda \gls{gl:preimagen} es conocida como
una función hash de un solo sentido o \gls{gl:owhf}.
Las que cumplen con la resistencia de segunda \gls{gl:preimagen} y
resistencia a las colisiones son conocidas como funciones hash
resistentes a colisiones o \gls{gl:crhf}. Aunque casi siempre las
funciones \gls{gl:crhf} cumplen con la resistencia de \gls{gl:preimagen},
no es obligatorio que lo hagan.

Algunos ejemplos de las funciones \gls{gl:owhf} son el \gls{gl:sha}-1
y el \gls{gl:md5}. En los esquemas de firma electrónica, se obtiene el
valor hash del mensaje ($h(m)$) y se pone en el lugar de la firma. Los valores
hash también son utilizados para revisar la integridad de las llaves
públicas y, al utilizarse con una llave secreta, las funciones
criptográficas hash se convierten en códigos de autenticación de mensaje
(\gls{gl:mac}, por sus siglas en inglés), una de las herramientas más
utilizadas en protocolos como \gls{gl:ssl} e IPSec para revisar la
integridad de un mensaje y autenticar al remitente.

Una de las aplicaciones más conocidas de las funciones hash es la de
cifrar las contraseñas: en un sistema, en vez de almacenar la contraseña
$clave$, se guarda su valor hash $h(clave)$. Así, cuando un usuario
ingresa su contraseña, el sistema calcula su valor hash y lo compara con
el que se tiene guardado. Realizar esto ayuda a evitar que las
contraseñas sean conocidas para los usuarios con privilegios, como
pueden ser los administradores.

% \subsection{Integridad de datos}
%
% Las funciones criptográficas hash también son conocidas como funciones
% \textit{procesadoras de mensajes} y el valor hash $h(m)$ de un mensaje
% $m$ dado es llamado \textit{huella} de $m$; ya que es una representación
% compacta de m y, dada la resistencia a la segunda \gls{gl:preimagen}, la huella
% es prácticamente única. Si el mensaje fuese modificado, el valor hash
% sería distinto; por lo que si se tienen almacenados los valores hash,
% basta con calcular su valor $h(m)$ y compararlo con el que se tiene
% guardado para detectar modificaciones. Por esta razón, las funciones
% hash también son llamadas códigos de detección de modificaciones (como el
% \gls{gl:mdc2}).
%
% \subsection{Firmas}
%
% Sea $(n, e)$ la llave pública \gls{gl:rsa} y $d$ el exponente decodificador
% secreto de Alice. En el esquema básico de firma \gls{gl:rsa}, Alice puede
% firmar mensajes que estén codificados por números $ m \in \{0, \dots, n-1\}$.
% Para firmar $m$, aplica el algoritmo de descifrado y obtiene la firma
% $\sigma = m^d$ $mod$ $n$ de $m$.
% Normalmente, $n$ es un número de 1024 bits y Alice puede firmar una
% cadena de bits $m$ tal que, cuando es interpretada como número, sea
% menor que $n$. Esto es una cadena de, máximo, 128 caracteres
% \gls{gl:ascii}: la mayoría de los documentos que se desean firmar suelen
% ser mucho más grandes. Este problema existe en todos los esquemas de firma
% digital y usualmente es resuelto al aplicar una función hash resistente a
% colisiones $h$. De esta forma, primero se obtiene el valor hash del mensaje
% $h(m)$ y esto es lo que se firma en lugar del mensaje mismo ($m$):
% \begin{equation}
%   \label{funcion_hash_sign}
%   \sigma = h(m)^d \quad mod \quad n
% \end{equation}
%
% Los mensajes que tengan el mismo valor hash tienen la misma firma. En
% este caso, es primordial que la función hash $h$ sea resistente a
% colisiones para garantizar el no repudio. De otra manera, Alice podría
% firmar el mensaje $m$ y después decir que había firmado un mensaje
% distinto ($n$).
% La resistencia a segundas preimágenes previene que un atacante Eve tome
% un mensaje $m$ firmado por Alice, genere un mensaje nuevo $n$ y utilice
% $\sigma$ como una firma válida de Alice para $n$.

% \subsection{Message Digest-4 (MD4)}
%
% En la década de 1990 esta función hash fue diseñada por Ronald Rivest.
% Tiene entradas de longitud arbitraria y la longitud de la salida
% procesada es de 128 bits. El \gls{gl:md4} fue innovador y
% clave en el diseño para los algoritmos venideros de esta clase (como
% el \gls{gl:md5}).
%
% \subsection{RIPEMD}
%
% Esta función hash, publicada en 1996, está basada en \gls{gl:md4} y fue
% diseñada por Hans Dobbertin y otros. Consiste en dos formas equivalentes de la
% función de compresión de \gls{gl:md4}. El algoritmo original (RIPEMD-160)
% devuelve bloques \textit{procesados} de 160 bits; cuando en 1996 Hans
% descubrió una colisión en dos rondas, se desarrollaron nuevas versiones
% mejoradas: RIPEMD-128, RIPE-256, RIPE-320; las cuales dan bloques procesados de
% 128, 256 y 320 bits respectivamente.

\subsection{\textit{Secure Hash Algorithm (SHA)}}

El algoritmo \gls{gl:sha} fue publicado por \gls{gl:nist} y
\gls{gl:nsa} en 1993; este algoritmo produce bloques de 160 bits y
fue desarrollado para reemplazar al \gls{gl:md4}; sin embargo, poco
después de haber sido publicado tuvo que ser quitado por problemas de
seguridad. Actualmente, \gls{gl:sha} es conocido como \gls{gl:sha}-0.

En 1995, \gls{gl:sha}-0 fue reemplazado por \gls{gl:sha}-1; tiene
una salida de la misma longitud que su predecesor y es una de las funciones
hash más populares. Hay que destacar que la seguridad que brinda esta función
es limitada, pues tiene el mismo nivel que un cifrado por bloques de 80 bits.

En 2002 \gls{gl:nist} publicó tres funciones hash más:
\gls{gl:sha}-256, \gls{gl:sha}-384 y \gls{gl:sha}-512; esta
familia de funciones hash es conocida como \gls{gl:sha}-2 y fue
desarrollada para cubrir la necesidad de una llave más grande para poder
empatar su tamaño con \gls{gl:aes}. Dos años más tarde, una nueva
función hash fue agregada a la familia \gls{gl:sha}-2:
\gls{gl:sha}-224.

Finalmente, en 2008, \gls{gl:nist} inició un concurso para buscar al
\gls{gl:sha}-3 y en 2012 anunció al ganador: Keccak, una función hash
desarrollada por Guido Bertoni, Joan Daemen, Michael Peeters y Gilles Van
Assche. Esta función tiene una construcción completamente distinta a las
familias anteriores.



  \printbibliography[
    heading=bibintoc,
    title={Bibliografía}
  ]

  \newpage
  \addcontentsline{toc}{chapter}{\listfigurename}
  \listoffigures

  \newpage
  \addcontentsline{toc}{chapter}{\listtablename}
  \listoftables

  \newpage
  \addcontentsline{toc}{chapter}{\lstlistlistingname}
  \lstlistoflistings

\end{document}
