%
% Archivo principal de documento final.
% Proyecto Lovelace.
%

\documentclass[11pt, letterpaper, twosides]{book}
\usepackage[spanish, mexico, es-noindentfirst]{babel}
\usepackage[utf8]{inputenc}
\newcommand{\Titulo}{Generación de \textit{tokens} para proteger
                     los datos de tarjetas bancarias}
\newcommand{\Subtitulo}{Trabajo terminal No. 2017-B008}
\newcommand{\Director}{Dra. Sandra Díaz Santiago}
\newcommand{\Autor}{Daniel Ayala Zamorano \\
                    Laura Natalia Borbolla Palacios \\
                    Ricardo Quezada Figueroa}
\newcommand{\Fecha}{Enero de 2018}
\usepackage{formato}

\begin{document}

  \Portada

  % TODO: crear dos índices, el primero general (sólo dos niveles) y el
  % segundo a detalle (todos los niveles).
  \pdfbookmark[1]{\contentsname}{Contenido}
  \setcounter{page}{2}
  \setcounter{tocdepth}{4}
  \tableofcontents
  \newpage

  \import{contenidos/}{simbologia.tex}
  \import{contenidos/intro/}{intro.tex}
  \import{contenidos/antecedentes/}{antecedentes.tex}
  \import{contenidos/analisis_y_disenio/}{analisis_y_disenio.tex}

  \clearpage
  \phantomsection
  \addcontentsline{toc}{chapter}{Bibliografía}
  \printbibliography[
    heading=subbibliography,
    title={Bibliografía}
  ]

  \glossarystyle{altlist}
  \printglossary[
    type=main,
    entrycounter=true
  ]

  \section*{Siglas y acrónimos}
  \phantomsection
  \addcontentsline{toc}{chapter}{Siglas y acrónimos}
  \glossarystyle{list}
  \setglossarysection{subsection}

  \printglossary[
    type=siglas_cripto,
    title={Criptográficos}
  ]

  \printglossary[
    type=\acronymtype,
    title={Comunes}
  ]

  \newpage
  \phantomsection
  \addcontentsline{toc}{chapter}{\listfigurename}
  \listoffigures

  \newpage
  \phantomsection
  \addcontentsline{toc}{chapter}{\listtablename}
  \listoftables

  \newpage
  \phantomsection
  \addcontentsline{toc}{chapter}{\lstlistlistingname}
  \lstlistoflistings

\end{document}
