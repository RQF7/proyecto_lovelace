%
% Definición de entradas del glosario.
% Proyecto Lovelace.
%
% Más información en https://www.sharelatex.com/learn/Glossaries
%

\makeglossaries

\newglossaryentry{gl:modo_de_operacion}
{
  name = modo de operación,
  plural = modos de operación,
  description = {
    Construcción que permite extender la funcionalidad de un cifrado a bloques
    para operar sobre tamaños de información arbitrarios%
  }
}

\newglossaryentry{gl:vector_de_inicializacion}
{
  name = vector de inicialización,
  plural = vectores de inicialización,
  description = {
    Cadena de bits de tamaño fijo que sirve como entrada a muchas primitivas
    criptográficas (e. g. algunos \glspl{gl:modo_de_operacion}). Generalmente
    se requiere que sea generado de forma aleatoria%
  }
}

\newglossaryentry{gl:ronda}
{
  name = ronda,
  description = {
    Bloque compuesto por un conjunto de operaciones que es ejecutado
    múltiples veces. Las \textit{rondas} son definidas por el algoritmo de
    cifrado%
  }
}

\newglossaryentry{gl:integridad_datos}
{
  name = integridad de datos,
  description = {
    Propiedad en la que los datos no han sido alterados sin autorización desde
    que fueron creados, transmitidos o almacenados por una fuente autorizada.
    Operaciones que insertan, eliminan, modifican o reordenan bits invalidan
    la \textit{integridad de los datos}. La \textit{integridad de los datos}
    incluye que los datos estén completos y, cuando los datos son divididos en
    bloques, cada bloque cumplir con lo mencionado anteriormente%
  }
}

\newglossaryentry{gl:autenticacion_origen}
{
  name = autenticación de origen,
  description = {
    Tipo de autenticación donde se corrobora que una entidad es la fuente
    original de la creación de un conjunto de datos en un tiempo específico.
    Por definición, la \textit{autenticación de origen} incluye la integridad
    de datos, pues cuando se modifican los datos, se tiene una nueva fuente%
  }
}

\newglossaryentry{gl:cifrado_iterativo}
{
  name = cifrado iterativo,
  plural = cifrados iterativos,
  description = {
    Cifrado de bloque que involucra la repetición secuencial de
    una función interna llamada función de \gls{gl:ronda}. Los
    parámetros incluyen en número de rondas, el tamaño de bloque y
    el tamaño de llave%
  },
  see=[véase también]{gl:ronda}
}

\newglossaryentry{gl:entropia}
{
  name = entropía,
  description = {
    Definida para una \gls{gl:funcion} de probabilidad de distribución discreta,
    mide cuánta información en promedio es requerida para identificar
    muestras aleatorias de esa distribución%
  },
  see=[véase también]{gl:funcion}
}

\newglossaryentry{gl:preimagen}
{
  name = preimagen,
  plural = preimágenes,
  description = {
    Suponga que se tiene $x \in X$ y $y \in Y$ tal que $f(x) = y$;
    se dice entonces que $x$ es \textit{preimagen} de $y$, o que
    $y$ es la \gls{gl:imagen} de $x$ bajo $f$%
  },
  see=[véase también]{gl:imagen}
}

\newglossaryentry{gl:imagen}
{
  name = imagen,
  plural = imágenes,
  description = {
    Suponga que se tiene $x \in X$ y $y \in Y$ tal que $f(x) = y$;
    se dice entonces que $y$ es la \textit{imagen} de $x$ bajo $f$,
    o que $x$ es \gls{gl:preimagen} de $y$%
  },
  see=[véase también]{gl:preimagen}
}

\newglossaryentry{gl:biyeccion}
{
  name = biyección,
  plural = biyecciones,
  description = {
    Dicho de las funciones que son \glspl{gl:inyectiva} y
    \glspl{gl:suprayectiva} al mismo tiempo; en otras palabras, que todos los
    elementos del conjunto de salida tengan una imagen distinta en el conjunto
    de llegada y a cada elemento del conjunto de llegada le corresponde
    un elemento del conjunto de salida%
  },
  see=[véase también]{gl:funcion}
}

\newglossaryentry{gl:inyectiva}
{
  name = inyectiva,
  description = {
    Una \gls{gl:funcion} $f:D_f \rightarrow C_f$ es \textit{inyectiva} (o uno a
    uno) si a diferentes elementos del \gls{gl:dominio} le corresponden
    diferentes elementos del codominio; se cumple para dos
    valores cualesquiera $x_1, x_2 \in D_f$ que
    $x_1 \neq x_2 \implies f(x_1) \neq f(x_2)$%
  },
  see=[véase también]{gl:funcion}
}

\newglossaryentry{gl:suprayectiva}
{
  name = suprayectiva,
  description = {
    Una \gls{gl:funcion} $f:D_f \rightarrow C_f$ es \textit{suprayectiva} si
    todo elemento de su codominio $C_f$ es \gls{gl:imagen} de
    por lo menos un elemento de su \gls{gl:dominio} $D_f$: $\forall b \in C_f$
    $\exists a \in D_f$ tal que $f(a)=b$%
  },
  see=[véase también]{gl:funcion}
}

\newglossaryentry{gl:dominio}
{
  name = dominio,
  description = {
    El \textit{dominio} de una \gls{gl:funcion} $f(x)$ es el conjunto de
    valores para los cuales la \gls{gl:funcion} está definida%
  },
  see=[véase también]{gl:funcion}
}

\newglossaryentry{gl:codominio}
{
  name = codominio,
  description = {
    Una \gls{gl:funcion} mapea a los elementos de un conjunto $A$ con elementos
    de un conjunto $B$; $A$ es el \gls{gl:dominio} y $B$ es el
    \textit{codominio}%
  },
  see=[véase también]{gl:funcion}
}

\newglossaryentry{gl:funcion}
{
  name = función,
  plural = funciones,
  description = {
    Regla entre dos conjuntos $A$ y $B$ de manera que a cada elemento del
    conjunto $A$ le corresponda un único elemento del conjunto $B$%
  }
}

\newglossaryentry{gl:token}
{
  name = \textit{token},
  description = {
    Valor representativo que se usa en lugar de información valiosa%
  }
}

\newglossaryentry{gl:criptografia_fuerte}
{
  name = criptografía fuerte,
  description = {
    De acuerdo al \gls{gl:pci} \gls{gl:ssc} (en \cite{dss_glosario}), es la
    criptografía basada en algorítmos probados y aceptados en la industria,
    junto con longitudes de llaves fuertes y buenas prácticas de
    administración de llaves%
  }
}

\newglossaryentry{gl:criptologia}
{
  name = criptología,
  description = {
    Estudio de los sistemas, claves y lenguajes secretos u ocultos%
  }
}

\newglossaryentry{gl:permutacion}
{
  name = permutación,
  plural = permutaciones,
  description = {
    Sea $ S $ un conjunto finito de elementos. Una \textit{permutación} $ p $
    en $ S $ es una \gls{gl:biyeccion} de $ S $ a sí misma (i. e.
    $ p: S \rightarrow S $)%
  },
  see=[véase también]{gl:biyeccion}
}

\newglossaryentry{gl:computacionalmente_no_factible}
{
  name = computacionalmente no factible,
  description = {
    Se dice que una tarea es \textit{computacionalmente no factible} si su
    costo (medido en términos de espacio o de tiempo) es finito pero
    ridículamente grande%
  }
}

\newglossaryentry{gl:primitiva_criptografica}
{
  name = primitiva criptográfica,
  plural = primitivas criptográficas,
  description = {
    Algoritmos criptográficos que son usados con frecuencia para la
    construcción de protocolos de seguridad. En \cite{menezes} (figura 1.1)
    se clasifican en tres categorías principales: de llave simétrica, de
    llave pública y sin llave%
  }
}

\newglossaryentry{gl:fuerza_efectiva}
{
  name = fuerza efectiva,
  plural = fuerzas efectivas,
  description = {
    Para un espacio de llaves $ K $, su \textit{fuerza efectiva} es
    $ \log_2 | K | $ (el logaritmo base dos de su cardinalidad)%
  }
}

\longnewglossaryentry{gl:probabilidad_condicional}
{
  name = probabilidad condicional
}
{%
  Sean $ E_1 $ y $ E_2 $ dos eventos, con $ P(E_2) \ge 0 $. La
  \textit{probabilidad condicional} se denota por $ P(E_1 | E_2) $, y es
  igual a
  $$ P(E_1 | E_2) = \frac{P(E_1 \cap E_2)}{P(E_2)} $$
  Esto mide la probabilidad de que ocurra $ E_1 $ sabiendo que ya ocurrió
  $ E_2 $.
}

\newglossaryentry{gl:estidisticamente_independiente}
{
  name = estadísticamente independiente,
  description = {
    Dicho de la ocurrencia de dos eventos $ E_1 $ y $ E_2 $: si
    $ P(E_1 \cap E_2) = P(E_1) P(E_2) $ entonces $ E_1 $ y $ E_2 $ son
    \textit{estadísticamente independientes} entre sí. Es importante notar
    que si esto ocurre, entonces $ P(E_1 | E_2) = P(E_1) $ y
    $ P(E_2 | E_1) = P(E_2) $, es decir, la ocurrencia de uno no tiene ninguna
    influencia en las probabilidades de ocurrencia del otro%
  },
  see=[véase también]{gl:probabilidad_condicional}
}

\newglossaryentry{gl:equiprobable}
{
  name = equiprobable,
  plural = equiprobables,
  description = {
    Distribución de probabilidad en la que todos los elementos tienen la misma 
    ocurrencia%
  }
}

\glsaddall
