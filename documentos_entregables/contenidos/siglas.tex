%
% Definición de siglas y acrónimos.
% Proyecto Lovelace.
%

\newacronym{gl:ecb}{ECB}{\textit{Electronic Codebook}}

\newacronym{gl:cbc}{CBC}{\textit{Cipher-block Chaining}}

\newacronym{gl:cfb}{CFB}{\textit{Cipher Feedback}}

\newacronym{gl:ofb}{OFB}{\textit{Output Feedback}}

\newacronym{gl:des}{DES}{\textit{Data Encryption Standard}}

\newacronym{gl:aes}{AES}{\textit{Advanced Encryption Standard}}

\newacronym{gl:feal}{FEAL}{\textit{Fast Data Encipherment Algorithm}}

\newacronym{gl:idea}{IDEA}{\textit{International Data Encryption Algorithm}}

\newacronym{gl:safer}{SAFER}{\textit{Secure And Fast Encryption Routine}}

\newacronym{gl:md4}{MD4}{\textit{Message Digest 4}}

\newacronym{gl:md5}{MD5}{\textit{Message Digest 5}}

\newacronym{gl:sha}{SHA}{\textit{Secure Hash Algorithm}}

\newacronym{gl:nist}{NIST}{
    \textit{National Institute of Standards and Technology}}

\newacronym{gl:nsa}{NSA}{\textit{National Security Agency}}

% Compañía o algoritmo ¿?
% Por eso no funden compañías con sus nombres...
\newacronym{gl:rsa}{RSA}{\textit{Ron Rivest, Adi Shamir, Leonard Adleman}}

\newacronym{gl:tls}{TLS}{\textit{Transport Layer Security}}

\newacronym{gl:wep}{WEP}{\textit{Wired Equivalent Privacy}}

\newacronym{gl:wpa}{WPA}{\textit{WiFi Protected Access}}

\newacronym{gl:ieee}{IEEE}{
    \textit{Institute of Electrical and Electronic Engineers}}

\newacronym{gl:lan}{LAN}{\textit{Local Area Network}}

\newacronym{gl:mac}{MAC}{\textit{Message Authentication Code}}

\newacronym{gl:hmac}{HMAC}{\textit{Keyed-Hashed Message Authentication Code}}

\newacronym{gl:mdc}{MDC}{\textit{Message Digest Cipher}}

\newacronym{gl:maa}{MAA}{\textit{Message Authenticator Algorithm}}

\newacronym{gl:fips}{FIPS}{\textit{Federal Information Processing Standard}}

\newacronym{gl:ecrypt}{ECRYPT}{
    \textit{European Network of Excellence in Cryptology}}

\newacronym{gl:nessie}{NESSIE}{
    \textit{New European Schemes for Signatures, Integrity and Encryption}}

\newacronym{gl:owhf}{OWHF}{\textit{One-Way Hash Function}}

\newacronym{gl:crhf}{CRHF}{\textit{Collision-Resistant Hash Function}}

\newacronym{gl:ssl}{SSL}{\textit{Secure Sockets Layer}}

\newacronym{gl:mdc2}{MDC-2}{\textit{Modification Detection Code 2}}

\newacronym{gl:ascii}{ASCII}{
    \textit{American Standard Code for Information Interchange}}

\newacronym{gl:pan}{PAN}{\textit{Personal Account Number}}

\newacronym{gl:pci}{PCI}{\textit{Payment Card Industry}}

\newacronym{gl:ssc}{SSC}{\textit{Security Standard Council}}

\newacronym{gl:dss}{DSS}{\textit{Data Security Standard}}

% ... Nuestro único acrónimo en español es el de la RAE.
\newacronym{gl:rae}{RAE}{Real Academia Española}

\newacronym{gl:ecc}{ECC}{\textit{Elliptic Curve Crypthosystem}}

\newacronym{gl:dh}{DH}{\textit{Diffie-Hellman}}

\newacronym{gl:dsa}{DSA}{\textit{Digital Signature Algorithm}}

\newacronym{gl:tdes}{TDES}{\textit{Triple \gls{gl:des}}}

\newacronym{gl:ctr}{CTR}{\textit{Counter Mode}}

\newacronym{gl:ocb}{OCB}{\textit{Offset Codebook}}

% ¿FFn? No lo encuentro (búsqueda superficial en duck duck go)
% \newacronym{gl:ffn}{FFn}{\textit{Counther Mode}}

% No estoy seguro de que la ES realmente sea eso; no lo dicen claramente
% en ningún lado:
% http://www.inf.pucrs.br/~calazans/graduate/TPVLSI_I/RSA-oaep_spec.pdf
\newacronym{gl:rsaes}{RSAES}{\textit{\gls{gl:rsa} Encryption Scheme}}

\newacronym{gl:oaep}{OAEP}{\textit{Optimal Asymmetric Encryption Padding}}

\newacronym{gl:ecdh}{ECDH}{\textit{Elliptic Curve \gls{gl:dh}}}

\newacronym{gl:ecmqv}{ECMQV}{\textit{Elliptic Curve Menezes-Qu-Vanstone}}

\newacronym{gl:ecdsa}{ECDSA}{\textit{Elliptic Curve \gls{gl:dsa}}}

\newacronym{gl:ecies}{ECIES}{
  \textit{Elliptic Curve Integrated Encryption Scheme}}

\newacronym{gl:dhe}{DHE}{\textit{\gls{gl:dh} Ephemeral}}

\newacronym{gl:iso}{ISO}{
  \textit{International Organization for Standaridization}}

\newacronym{gl:iec}{IEC}{
  \textit{International Electrotechnical Commission}}

\newacronym{gl:cdv}{CDV}{\textit{Card Data Vault}}
