%
% Construcción de tkr2 (y tkr2a) de
% «A Cryptographic Study of Tokenization Systems»
% Proyecto Lovelace.
%

\subsubsection{\textit{A Cryptographic Study of Tokenization Systems}}

Por claridad de contenidos, se establecen las siglas \textit{ACSTS} para
referirse a este algoritmo en futuras referencias.

En \cite{doc_sandra} se analiza formalmente el problema de la generación de
\glspl{gl:token} y se propone un algoritmo que no está basado en \gls{gl:fpe}.
Hasta antes de la publicación de este documento, los únicos métodos para
generar \glspl{gl:token} cuya seguridad estaba formalmente demostrada eran los
basados en \gls{gl:fpe}.

El algoritmo propuesto usa \glspl{gl:primitiva_criptografica} para generar
\glspl{gl:token} aleatorios y almacena en una base de datos (\gls{gl:cdv}) la
relación original de estos con los \gls{gl:pan}. Es, por tanto, un algoritmo
tokenizador reversible, y también, pese a la contradiccón con el nombre, no
criptográfico.  En el pseudocódigo \ref{tkr2_cifrado} se muestra el proceso
de tokenización, mientras que en \ref{tkr2_descifrado} está la detokenización.

\begin{pseudocodigo}[%
    caption={\textit{ACSTS}, método de tokenización},
    label={tkr2_cifrado}%
  ]
  entrada: PAN p; información asociada d; llave k
  salida:  token
  inicio
    $ S_1 $ $ \gets $ buscarPAN(p)
    $ S_2 $ $ \gets $ buscarInfoAsociada(d)
    si $ S_1 $ y $ S_2 $ = 0:
      t $ \gets $ RN(k)
      insertar(t, p, d)
    sino:
      t $ \gets $ $ S_1 $
    fin
    regresar t
  fin
\end{pseudocodigo}

\begin{pseudocodigo}[%
    caption={\textit{ACSTS}, método de detokenización},
    label={tkr2_descifrado}%
  ]
  entrada: token t; información asociada d; llave k
  salida:  PAN
  inicio
    $ S_1 $ $ \gets $ buscarToken(t)
    $ S_2 $ $ \gets $ buscarInfoAsociada(d)
    si $ S_1 $ y $ S_2 $ = 0:
      regresar error
    sino:
      p $ \gets $ $ S_1 $
    fin
    regresar p
  fin
\end{pseudocodigo}

La mayor parte del proceso de tokenización y toda la detokenización son
bastante fáciles de comprender; lo único que queda por esclarecer el la
función generadora de \glspl{gl:token} aleatorios $ RN_k $.
