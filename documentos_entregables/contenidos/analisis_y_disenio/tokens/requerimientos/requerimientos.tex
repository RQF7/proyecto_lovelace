%
% Requerimientos de tokens, análisis y diseño de generación de tokens.
% Proyecto Lovelace.
%

\subsection{Requerimientos}

En \cite{pci_tokens}, el \gls{gl:pci} \gls{gl:ssc} divide a los
\glspl{gl:token} en reversibles e irreversibles. A su vez, los reversibles se
dividen en criptográficos y en no criptográficos; mientras que los
irreversibles se dividen en autenticables y no autenticables (figura
\ref{fig:division_tokens}).

\begin{figure}[H]
  \begin{center}
    \includegraphics[width=0.75\linewidth]
      {contenidos/analisis_y_disenio/tokens/requerimientos/diagramas/clasificacion.png}
    \caption{Clasificación de los \glspl{gl:token}.}
    \label{fig:division_tokens}
  \end{center}
\end{figure}

En esta sección se explica cada una de las categorías y se analizan los
requerimientos que deben tener. Para comenzar, se enlistan los requerimientos
aplicables a todos los \glspl{gl:token}, sin importar en qué categoría estén:

% GT1 Es específico a productos de hardware.
% GT2 se lo chutaron.
% GT3 Validación FIPS que debe cumplir el software; fuera de alcance.

% GT4
\requerimiento{Resistencia a texto claro conocido}
{
  Un atacante con acceso a múltiples pares de \glspl{gl:token} y
  \gls{gl:pan} no debe de ser capaz de determinar otros \gls{gl:pan} a partir
  de solamente \glspl{gl:token}. En otras palabras, los \glspl{gl:token}
  deben ser resistentes a ataques con texto en claro conocido (sección
  \ref{sec:criptoanalisis}).
}

% GT5
% Me parece que esto es reduntante: si GT4 ya estableció la necesidad de
% resistencia a texto claro conocido, la resistencia a sólo texto cifrado
% ya va incluida. De hecho, en la misma redacción de GT5 incluyen claramente
% a GT4 con una disyunción.
% Creo que ya había leído esto mismo en alguno de los artículos que nos pasó
% Sandra.
\requerimiento{Resistencia a sólo texto cifrado}
{
  Recuperar un \gls{gl:pan} a partir de un \gls{gl:token} debe de ser
  \gls{gl:computacionalmente_no_factible} (resistencia a ataques con sólo
  texto cifrado, sección \ref{sec:criptoanalisis}).
}

% GT6
% ¿Requerimiento de API web?
\requerimiento{Detección de anomalías}
{
  Se deben de implementar disparadores que permitan detectar
  irregularidades en el sistema (anomalías, funcionamientos erróneos,
  comportamientos sospechosos). El producto debe registrar dichos eventos y
  avisar al personal correspondiente.
}

% GT7
\requerimiento{Distinción entre \glspl{gl:token} y \gls{gl:pan}}
{
  Se debe contar con un mecanismo para distinguir entre \glspl{gl:token}
  y \gls{gl:pan}. Los proveedores del servicio de tokenización deben
  compartir este mecanismo con la entidad (o entidades) que usa los
  \glspl{gl:token}.
}

% GT8 Guía de instalación; fuera de alcance. ¿Documentación de API web?
% de cualquier forma, es requerimiento de api web.
% GT9 Integridad de ejecutables; fuera de alcance.
% GT10 Autenticación de usuarios; requerimiento de api web.

% GT11
\requerimiento{Mapeos de \gls{gl:token} a \gls{gl:token} prohibidos}
{
  No se debe poder pasar de un primer \gls{gl:token} válido a un segundo,
  también válido; forzosamente debe existir un estado intermedio: del primer
  \gls{gl:token} se pasa al \gls{gl:pan} correspondiente (operación de
  detokenización) y de este se pasa al segundo \gls{gl:token}.
}

% GT12
\requerimiento{Protección contra vulnerabilidades comunes}
{
  Se deben implementar medidas en contra de las vulnerabilidades de
  seguridad más comunes (\cite{dss_pa}, requerimiento 5.2). Algunas de estas
  medidas pueden ser el uso de herramientas de análisis de código estático,
  o el uso de lenguajes de programación especializados.
}

% GT13 Primitivas criptográficas usadas
% También se habla sobre un documento de validación estadísitca; fuera de
% alcance.
\requerimiento{Primitivas criptográficas usadas}
{
  Las primitivas criptográficas que se usen deben estar basadas en
  estándares nacionales (Estados Unidos) o internacionales (e. g.
  \gls{gl:aes}).

  Esta es una prueba de un segundo párrafo agrgado a la definición de un requerimiento;
  la idea es que conserve la identación de las líneas anteriores.
}

Los \glspl{gl:token} irreversibles no pueden, bajo ninguna circunstancia, ser
reconvertidos al \gls{gl:pan} original. Esta restricción aplica tanto para
cualquier entidad en el entorno del negocio (comerciante, proveedor de
\glspl{gl:token}, banco) como para cualquier posible atacante. Dados un
\gls{gl:pan} y un \gls{gl:token}, los identificables permiten validar cuando el
primero fue utilizado para la creación del segundo, mientras que los no
identificables, no.

% TODO: ¿Para qué demonios sirven los no identificables?

% TODO: este párrafo me está dando problemas. No quiero copiar tal cuál la
% clasificación del PCI sin manifestar disconformidad, pero tampoco quiero
% entrar en demasiados detalles aún sobre las propias implementaciones
% (las cuales, evidentemente, sí son criptográficas); PD: hay que citar más
% publicaciones, a parte de la de Sandra.

La clasificación del \gls{gl:pci} \gls{gl:ssc} con respecto a los reversibles
resulta un poco confusa (esto ya ha sido señalado antes, \cite{doc_sandra}).
Establece que los criptográficos son generados utilizando
\gls{gl:criptografia_fuerte}, el \gls{gl:pan} nunca se almacena, solamente se
guarda una llave; los no criptográficos guardan la relación entre
\glspl{gl:token} y \gls{gl:pan} en una base de datos. El problema está en que
no se menciona \textit{cómo} generar los no criptográficos. A pesar del nombre,
los métodos más comunes para esta categoría ocupan primitivas criptográficas
(e. g. generadores pseudoaleatorios); además de que, en una implementación
real, para poder cumplir con el \gls{gl:pci} \gls{gl:dss}, la propia base de
datos debe de estar cifrada \cite{pci_dss}.

%
% Requerimientos para tokens irreversibles,
% análisis y diseño de generación de tokens.
% Proyecto Lovelace.
%

\subsubsection{Irreversibles}

%
% Requerimientos para tokens criptográficos reversibles,
% análisis y diseño de generación de tokens.
% Proyecto Lovelace.
%

\subsubsection{Criptográficos reversibles}

% RC1A
\requerimiento{Seguridad de la administración de llaves}
{
  la administración de las llaves criptográficas debe ser segura.

  % RC1A-1
  \subrequerimiento{Exportar llave en claro prohibido}
  {
    Las llaves usadas para generar \glspl{gl:token} no se deben poder
    exportar en claro desde el programa.
  }

  % RC1A-2
  % TODO: ¿Cómo funciona la entropía de una fuente generadora?
  \subrequerimiento{Entropía de generación de llaves}
  {
    La fuente generadora de llaves debe tener, al menos, 128 bits de
    \gls{gl:entropia}.
  }

  % RC1A-3
  \subrequerimiento{Llaves de uso único}
  {
    Las llaves criptográficas usadas para generar \glspl{gl:token} no
    deben ser usadas para ningún otro fin.
  }
}

% RC1B
% TODO: el documento explica porque esta cantidad; aún no lo entiendo.
\requerimiento{Probabilidad de adivinar relaciones}
{
  La probabilidad de adivinar la relación entre un \gls{gl:token} y un
  \gls{gl:pan} debe de ser menor que $ 1 $ en $ 10^6 $.

  % RC1B-1 (relacionado con RC1A-2, creo)
  \subrequerimiento{Distribución uniforme}
  {
    Para un \gls{gl:pan} dado, todos los \glspl{gl:token} deben ser
    equiprobables; esto es, el mecanismo tokenizador no debe exhibir
    tendencias probabilísticas que lo expongan a ataques estadísticos.
  }

  % RC1B-2 permutación aleatoria
  % TODO: ¿Cuál es la diferencia entre una «permutación aleatoria» y una
  % «permutación aleatoria fuerte»?
  % Si el mapeo es entre dos espacios distintos... ya no se llama
  % permutación, ¿o sí?
  \subrequerimiento{Permutación aleatoria}
  {
    El método de tokenización debe actuar como una familia de
    \glspl{gl:permutacion} aleatoria desde el espacio de \gls{gl:pan} al
    espacio de \glspl{gl:token}.
  }

  % RC1B-3
  \subrequerimiento{Cambio de llave}
  {
    Un cambio en la llave se debe ver reflejado en un cambio en el
    \gls{gl:token} resultado.
  }

  % RC1B-4
  \subrequerimiento{Cambio de \gls{gl:pan}}
  {
    Un cambio en el \gls{gl:pan} se debe ver reflejado en un cambio en el
    \gls{gl:token} resultado.
  }

  % RC1B-5 Aleatoriedad de dígitos, se debe validar que se sigue la
  % recomendación del NIST para los generadores de números pseudoaleatorios;
  % fuera de alcance.
  \subrequerimiento{Verificación de la aleatoriedad}
  {
    Se debe tener un medio para verificar de forma práctica la aleatorización
    de dígitos.
  }

}

% RC1C almacenamiento duplicado
% Este no lo entiendo: primero, ¿qué no se suponía que en los criptográficos
% nunca se almacenaba el token?; segundo, aún suponiendo que este
% requerimiento fuera en realidad para los no criptográficos, ¿Cuál es la
% diferencia entre guardar el PAN completo y guardar el PAN truncado?
\requerimiento{Almacenamiento de tokens}
{
  Los \glspl{gl:token} generados a partir de un \gls{gl:pan} completo no
  deben se ser almacenados en el producto de tokenización ni sistemas
  dependientes.
}

%\item Los \glspl{gl:token} que se basan en el \gls{gl:pan} completo no se
%  deben almacenar si ya hay guardado un \gls{gl:token} equivalente del
%  \gls{gl:pan} truncado.

% Los del dominio 2 son requerimeintos para la api web.

% RC4A Protección de llaves
% Se debe ocupar un SCD (secure cryptographyc device) para administrar
% llaves; puro capitalismo: puedes comprar los del PCI o hacer que el NIST
% valide los tuyos; fuera de alcance, evidentemente.
% De cualquier forma, sería requerimiento de la api web.
\requerimiento{Seguridad de la administración de llaves}
{
  Todas la operaciones sobre la administración de las llaves criptográficas
  deben realizarse en un dispositivo criptográfico seguro y aprobado.
}


% RC4B
% Claros y concisos: el primer documento son unas recomendaciones 160 hojas
% y la ISO ni siquiera la puedo ver (más capitalismo); puedo «comprar» el
% estándar, ¿qué clase de estándar se vende?
\requerimiento{Prácticas para la administración de llaves}
{
  La administración de llaves criptográficas debe llevarse a cabo de
  acuerdo al estándar del \gls{gl:nist} \cite{nist_llaves} y a
  \acrshort{gl:iso}/\acrshort{gl:iec} 11770.

  % RC4B1 Ciclo de vida de las llaves.
  % Un ISO más... empiezo a extrañar los RFC de la IETF.
  % TODO: incluir sección con resumen del contenido del anexo D.
  \subrequerimiento[rq:cripto:ciclo_de_vida]{Ciclo de vida de las llaves}
  {
    Las llaves para \textit{tokenizar} deben seguir una política de
    ciclo de vida como la descrita en  \acrshort{gl:iso}/\acrshort{gl:iec}
    11568-1.
  }

  % RC4B2
  \subrequerimiento{Descripción de periodos para cada llave}
  {
    La política de la vida útil de cada llave debe incluir una
    descripción del periodo activo.
  }

  % RC4B3
  \subrequerimiento{Permitir la destrucción de todas las llaves}
  {
    El sistema debe permitir la destrucción de las llaves sin necesidad
    de una manipulación externa.
  }
}

% RC4C-1 y -2 Longitud de llaves
% (esto se repite con la sección de primitivas).
\requerimiento{Sobre la longitud de las llaves}
{
  Las llaves para \textit{tokenizar} deben tener una
  \gls{gl:fuerza_efectiva} de, al menos, 128 bits. Cualquier llave utilizada
  para proteger o para derivar la llave del token debe de ser de igual o
  mayor \gls{gl:fuerza_efectiva}.
}

% RC4D
\requerimiento{Independencia estadística}
{
  Si el espacio de llaves es usado para producir \glspl{gl:token} es dos
  contextos distintos (e. g. para distintos comerciantes), estas deben ser
  \glspl{gl:estidisticamente_independiente}.
}

\begin{table}[H]
  \centering
  \begin{tabular}{| p{5.5cm} | p{2cm} | p{4cm} | p{4cm} |}

    \hline
      \textbf{Requerimiento}    &
      \textbf{PCI}              &
      \textbf{Clasificación}    &
      \textbf{Observación}     \\ [0.8ex]
    \hline

    SUBREQUTO-15/1 Exportar llave en claro prohibido.  
    &  RC 1A-1  &  \hyperref[dm:gen_tokens]{Generación de tokens}              &  Aplicable        \\ \hline
    SUBREQUTO-15/2 Entropía de generación de llaves.  
    &  RC 1A-2  &  \hyperref[dm:gen_tokens]{Generación de tokens}              &  -                \\ \hline
    SUBREQUTO-15/3 Llaves de uso único.  
    &  RC 1A-3  &  \hyperref[dm:gen_tokens]{Generación de tokens}              &  Aplicable        \\ \hline
    SUBREQUTO-16/1 Distribución uniforme.  
    &  RC 1B-1  &  \hyperref[dm:gen_tokens]{Generación de tokens}              &  Aplicable        \\ \hline
    SUBREQUTO-16/2 Permutación aleatoria.  
    &  RC 1B-2  &  \hyperref[dm:gen_tokens]{Generación de tokens}              &  Aplicable        \\ \hline
    SUBREQUTO-16/3 Cambio de llave.  
    &  RC 1B-3  &  \hyperref[dm:gen_tokens]{Generación de tokens}              &  Aplicable        \\ \hline
    SUBREQUTO-16/4 Cambio de PAN.  
    &  RC 1B-4  &  \hyperref[dm:gen_tokens]{Generación de tokens}              &  Aplicable        \\ \hline
    SUBREQUTO-16/5 Verificación de la aleatoriedad.  
    &  RC 1B-5  &  \hyperref[dm:gen_tokens]{Generación de tokens}              &  Fuera de alcance \\ \hline
    REQUTO-17 Almacenamiento de tokens.  
    &  RC 1C    &  \hyperref[dm:gen_tokens]{Generación de tokens}              &  -                \\ \hline
    REQUTO-18 Seguridad de la administración de llaves.  
    &  RC 4A    &  \hyperref[dm:man_llaves]{Manejo criptográfico de llaves}    &  Fuera de alcance \\ \hline
    SUBREQUTO-19/1 Ciclo de vida de las llaves.  
    &  RC 4B-1  &  \hyperref[dm:man_llaves]{Manejo criptográfico de llaves}    &  Fuera de alcance \\ \hline
    SUBREQUTO-19/2 Descripción de periodos para cada llave.  
    &  RC 4B-2  &  \hyperref[dm:man_llaves]{Manejo criptográfico de llaves}    &  Fuera de alcance \\ \hline
    SUBREQUTO-19/3 Permitir la destrucción de todas las llaves.  
    &  RC 4B-3  &  \hyperref[dm:man_llaves]{Manejo criptográfico de llaves}    &  Aplicable        \\ \hline
    REQUTO-20 Sobre la longitud de las llaves.  
    &  RC 4C    &  \hyperref[dm:man_llaves]{Manejo criptográfico de llaves}    &  Aplicable        \\ \hline
    REQUTO-21 Independencia estadística.  
    &  RC 4D    &  \hyperref[dm:man_llaves]{Manejo criptográfico de llaves}    &  Aplicable        \\ \hline

  \end{tabular}
  \caption{Requerimientos para tokens criptográficos reversibles.}
\end{table}


%
% Requerimientos para tokens no criptográficos reversibles,
% análisis y diseño de generación de tokens.
% Proyecto Lovelace.
%

\subsubsection{No criptográficos reversibles}

%
% Requerimientos para las primitivas criptográficas usadas,
% análisis y diseño de generación de tokens
%.
% Proyecto Lovelace.
% Anexo C de las guías para tokens del PCI.
%

\capitulo{Requerimientos para las primitivas criptográficas}{sec:primitivas}
{
  \epigrafe
  {%
    Lock up your libraries if you like; but there is no gate, no lock, no bolt
    that you can set upon the freedom of my mind.%
  }
  {%
    Virginia Woolf, A Room of One's Own.%
  }
}

En este apéndice se resumen los requerimientos mínimos que debe de tener
cualquier \gls{gl:primitiva_criptografica} que se use dentro del sistema
\textit{tokenizador}. Esta información se presenta en el anexo C de
\cite{pci_tokens}, el cual está clasificado como \textit{informativo} solamente
(no \textit{normativo}). Con respecto a las \glspl{gl:primitiva_criptografica},
la parte \textit{normativa} está controlada por el programa \gls{gl:cavp} del
\gls{gl:nist}; el cual evalúa las implementaciones usadas de una serie de
primitivas comunes\footnote{Esta lista se puede encontrar en
\url{https://csrc.nist.gov/Projects/Cryptographic-Algorithm-Validation-Program}}.

En la tabla~\ref{minimo_llaves} se colocan los tamaños mínimos de llaves y
\glspl{gl:modo_de_operacion} (sección~\ref{sec:modos}) asociados
para las \glspl{gl:primitiva_criptografica} de los «algoritmos
criptográficos»\footnote{El \gls{gl:pci} \gls{gl:ssc} parece dividir a las
\glspl{gl:primitiva_criptografica} en \textit{algoritmos criptográficos},
\textit{funciones hash} y \textit{generadores de números pseudoaleatorios};
esto resulta confuso dado que las tres categorías pertenecen al campo de
estudio de la criptografía.} permitidos. Estos son: \gls{gl:aes} (sección
\ref{sec:aes}), \gls{gl:rsa} (sección~\ref{sec:clasificacion}), \gls{gl:ecc} y
\gls{gl:dsa}/\gls{gl:dh}. Se hace especial énfasis en que \gls{gl:tdes} no
está permitido.

% La última lista en el párrafo es para evitar que las siglas de los
% protocolos no enlistados antes aparezcan expandidos en la tabla y para poder
% poner referencias cruzadas (en las celdas se ven raras). Para cuando no haya
% nada que hacer, estaría bien agregar a los antecedentes a las curvas
% elípticas, y para el intercambio de llaves Diffie-Hellman.
% Lástima que son muchos modos de operación para hacer lo mismo; creo que
% aquí sí ocuparé \acrshort (ahora sí que crecerá nuestra sección de siglas
%y acrónimos).

% Para AES había un FFn que no encuentro en ningún otro lado.
% Luego el PCI les cambia el nombre, tengo que buscar más a fondo.
% Misma historia para EHD.

% ¿Qué son los bits de seguridad? (glosario)
La tabla~\ref{hash_permitidos} enlista los algoritmos hash (sección
\ref{sec:hash}) permitidos. Para evitar introducir fallas de seguridad a
través de las funciones hash, estas deben proveer al menos tantos bits de
seguridad como el algoritmo criptográfico usado, y en cualquier caso, no
menos de 128 bits (lo que deja fuera a \gls{gl:md4} y \gls{gl:md5}).

El número de bits de entropía utilizados para los generadores de números
aleatorios debe de ser mayor o igual al número de bits de seguridad utilizados
para las primitivas anteriores. Cuando se utilicen generadores determinísticos,
estos deben seguir las recomendaciones del \gls{gl:nist} en
\cite{nist_aleatorios}.

\newpage

\begin{table}[H]
  \centering
  \begin{tabular}{| c | c | c |}
    \hline
    \textbf{Algoritmo} &
    \textbf{Tamaño de llave} &
    \textbf{Modo de operación} \\ [0.5ex]
    \hline
    \textbf{\gls{gl:aes}} &
    128 &
    \acrshort{gl:ctr},
    \acrshort{gl:ocb},
    \acrshort{gl:cbc},
    \acrshort{gl:ofb},
    \acrshort{gl:cfb} \\
    \hline
    \textbf{\gls{gl:rsa}} &
    3072 &
    \acrshort{gl:rsaes}-\acrshort{gl:oaep} \\
    \hline
    \textbf{\gls{gl:ecc}} &
    256 &
    \acrshort{gl:ecdh},
    \acrshort{gl:ecmqv},
    \acrshort{gl:ecdsa},
    \acrshort{gl:ecies} \\
    \hline
    \textbf{\gls{gl:dsa}/\gls{gl:dh}} &
    3072/256 &
    \acrshort{gl:dhe}\\
    \hline
  \end{tabular}
  \caption{Longitudes de llave mínimas y \glspl{gl:modo_de_operacion}
      permitidos para algoritmos criptográficos}
  \label{minimo_llaves}
\end{table}

\begin{table}[H]
  \centering
  \begin{tabular}{| c | c |}
    \hline
    \textbf{Bits de seguridad} &
    \textbf{Algoritmo hash} \\ [0.5ex]
    \hline
    128 & \gls{gl:sha}-256 \\
    \hline
    128 & \gls{gl:sha}3-256 \\
    \hline
    192 & \gls{gl:sha}3-384 \\
    \hline
    256 & \gls{gl:sha}-512 \\
    \hline
    256 & \gls{gl:sha}3-512 \\
    \hline
  \end{tabular}
  \caption{Algoritmos hash permitidos}
  \label{hash_permitidos}
\end{table}

