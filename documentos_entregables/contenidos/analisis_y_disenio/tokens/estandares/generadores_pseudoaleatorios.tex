%
% Recomendaciones del NIST para la generación de bits pseudoaleatorios,
% capítulo de análisis y diseño para la generación de tokens.
% Proyecto Lovelace.
%

\subsubsection{Generación de bits pseudoaleatorios}
\label{sec:generadores_pseudoaleatorios}

Existen dos maneras de generar bits aleatorios: la primera es producir bits
de manera no determinística, donde el estado de cada uno (uno o cero) está
determinado por un proceso físico impredecible. Estos generadores de bits
aleatorios (\gls{gl:rbg}) son conocidos como generadores no determinísticos, o
\gls{gl:nrbg}. La otra manera, que será explorada a continuación, es calcular
determinísticamente los bits mediante un algoritmo; estos generadores
determinísticos son conocidos como \gls{gl:drbg}.

Un \gls{gl:drbg} tiene un mecanismo que utiliza un algoritmo que produce una
secuencia de bits partiendo de un valor inicial que es determinado por una
\gls{gl:semilla} que, a su vez, está determinada por la salida de la fuente de
aleatoriedad. Una vez que se tiene la \gls{gl:semilla} y se determina el valor
inicial, el \gls{gl:drbg} es instanciado y puede producir valores. Dado a su
naturaleza determinística, se dice que los valores producidos por el
\gls{gl:drbg} son pseudoaleatorios y no aleatorios; si la \gls{gl:semilla} es
mantenida oculta y el algoritmo fue bien diseñado, los bits de salida del
\gls{gl:drbg} serán impredecibles.

%% MODELO DE UN DRBG

La entrada de \gls{gl:entropia} es provista a un mecanismo \gls{gl:drbg} para
obtener una \gls{gl:semilla} utillizando una fuente de aleatoriedad. La entrada
de \gls{gl:entropia} y la \gls{gl:semilla} deben mantenerse secretas; que estos
valores permanezcan secretos es una de las bases de la seguridad del
\gls{gl:drbg}. Otras entradas, como un \gls{gl:nonce} o una cadena de
personalización pueden ser utilizadas como entradas; estas pueden o no requerir
ser mantenidas secretas también, y ser utilizadas para crear la \gls{gl:semilla}
inicial para el \gls{gl:drbg}.

El estado interno es la memoria del \gls{gl:drbg} y consiste en todos los
valores que requiere el mecanismo (parámetros, variables, etcétera).

El mecanismo \gls{gl:drbg} requiere cinco funciones; estas son explicadas con
más detalle abajo:
\begin{enumerate}
    \item Instanciación (\textit{instantiate function}): obtiene la entrada de
        \gls{gl:entropia} para crear una \gls{gl:semilla} con la cual será
        creado un nuevo estado interno. La entrada puede ser combinada con
        una \gls{gl:nonce} o una cadena de personalización.
    \item Generación (\textit{generate function}): genera bits pseudoaleatorios
        utilizando el estado interno actual; también tiene como salida un nuevo
        estado interno que es utilizado para el siguiente pedido.
    \item Cambio de \gls{gl:semilla} (\textit{reseed function}): obtiene una
        nueva entrada de \gls{gl:entropia} y la combina con el estado interno
        actual para crear una nueva \gls{gl:semilla} y un nuevo estado interno.
    \item Desinstanciación (\textit{uninstantiate function}): elimina el estado
        interno actual.
    \item Prueba de salud (\textit{healt htest function}): determina que el
        mecanismo \gls{gl:drbg} siga funcionando correctamente.
\end{enumerate}

Cuando a un \gls{gl:drbg} se le aplica la función de cambio de \gls{gl:semilla},
la \gls{gl:semilla} es imperativo que la \gls{gl:semilla} sea distinta a la que
se utilizó en la función de instanciación. Cada \gls{gl:semilla} define un nuevo
periodo de \gls{gl:semilla} (\textit{seed period}) para la instanciación del
\gls{gl:drbg}. Una instanciación consiste en uno o más periodos de
\gls{gl:semilla}, estos comienzan cuando se obtiene una nueva \gls{gl:semilla} y
terminan cuando la siguiente \gls{gl:semilla} es obtenida o el \gls{gl:drbg}
deja de utilizarse.

El estado interno deriva de la \gls{gl:semilla}; este incluye el estado de
trabajo (uno o más valores derivados de la \gls{gl:semilla} que deben
permanecer secretos, y la cuenta con el número de salidas que se han producido
con esa semilla) y la información administrativa (el nivel de seguridad, etc).
Es menester proteger el estado interno del \gls{gl:drbg}. La implementación del
mecanismo \gls{gl:drbg} puede haber sido diseñado para tener múltiples
instancias; en este caso, cada instancia debe tener su propio estado interno y
el estado interno de una instancia \gls{gl:drbg} jamás debe ser utilizado como
estado interno para una instancia distinta. El estado interno no debe ser
accesible a funciones distintas a las cinco del \gls{gl:drbg}, ni a otras
instancias del \gls{gl:drbg} o a otros \gls{gl:drbg}.

Los mecanismos especificados en \cite{nist_aleatorios} soportan cuatro niveles
de seguridad: 112, 128, 192 y 256 bits. Este es uno de los parámetros que
se necesitan para instanciar un \gls{gl:drbg}; además, dependiendo de su diseño,
cada mecanismo  \gls{gl:drbg} tiene sus restricciones de nivel de seguridad.
El nivel de seguridad depende de la implementación del \gls{gl:drbg} y la
cantidad de \gls{gl:entropia} que se da como entrada a la función de
instanciación.

Los bits pseudoaleatorios obtenidos mediante un \gls{gl:drbg} no deben ser
utilizados por una aplicación que requiera un nivel mayor de seguridad que con
el que fue instanciado el \gls{gl:drbg}. La concatenación de dos salidas del
\gls{gl:drbg} tampoco proveen un nivel de seguridad más alto que del que fueron
instanciados (por ejemplo, dos cadenas concatenadas de 128 bits no dan como
resultado una cadena de 256 bits con el nivel de seguridad de 256 bits).

\paragraph{Semillas}
Las \glspl{gl:semilla} deben ser obtenidas antes de generar bits
pseudoaleatorios en el \gls{gl:drbg}, pues esta es utilizada para instanciar al
\gls{gl:drbg} y determinar el estado inicial interno del mecanismo.

Cambiar la \gls{gl:semilla} restaura el secreto de la salida del \gls{gl:drbg}
si el estado interno o la \gls{gl:semilla} son conocidos. Hacer este cambio
periódicamente es una buena manera de mantener a raya el peligro de que valores
como la entrada de \gls{gl:entropia}, la \gls{gl:semilla} o el estado interno
de trabajo; hayan sido comprometidos.

Los ingredientes para determinar una nueva \gls{gl:semilla} para la función de
instanciación son la entrada de entropía de una fuente aleatoria, un
\gls{gl:nonce} y una cadena de personalización (recomendada, pero no
obligatoria). Para hacer un cambio de semilla, se necesita el estado interno
actual, la entrada de \gls{gl:entropia} y una entrada adicional opcional.

La longitud de la \gls{gl:semilla} depende del mecanismo \gls{gl:drbg} y el
nivel de seguridad requerido; sin embargo, siempre debe ser de, al menos, el
mismo número de bits de \gls{gl:entropia} requerida.

La entrada de \gls{gl:entropia} y la \gls{gl:semilla} resultante deben de ser
protegidas con el mismo cuidado con el que se protege la salida del
\gls{gl:drbg}; por ejemplo, si el mecanismo es utilizado para generar llaves,
estos valores deben protegerse con la misma seguridad como son protegidas las
llaves generadas. Además, la seguridad del \gls{gl:drbg} depende de mantener
en secreto la entrada de \gls{gl:entropia}, por lo que esa entrada debe ser
tratada como un parámetro crítico de seguridad (\gls{gl:csp}) y ser obtenido
desde un módulo criptográfico que contenga la función necesaria o ser
transmitido desde un canal seguro.

Cuando se requiera un \gls{gl:nonce} para la construcción de la
\gls{gl:semilla}, esta debe cumplir con una de las siguientes dos condiciones:
tener $nivel\_seguridad/2$ bits de \gls{gl:entropia} o un valor que se espera
no se repita más de lo que se repetiría una cadena aleatoria de
$nivel\_seguridad/2$ bits. Aunque no debe ser mantenido en secreto, cada
\gls{gl:nonce} debe ser considerado como un \gls{gl:csp} y debe ser único en
el módulo criptográfico en donde se realiza la instanciación. El \gls{gl:nonce}
puede estar compuesto por uno o más de los siguientes valores:
\begin{enumerate}
    \item Valor aleatorio generado para cada \gls{gl:nonce} por un generador
        de bits aleatorios aprobado.
    \item Una marca de tiempo con la resolución suficiente para que sea distinto
        cada vez que sea utilizado.
    \item Un número de secuencia que se incremente constantemente.
    \item Una combinación de una marca de tiempo y un número de secuencia que se
        incremente constantemente; tal que el número de secuencia regrese a su
        valor inicial solo cuando la marca de tiempo cambie.
\end{enumerate}

Generar demasiadas salidas partiendo de una misma \gls{gl:semilla} puede proveer
suficiente información para ser capaz de predecir las salidas futuras; por lo
que el cambio de \glspl{gl:semilla} reduce riesgos de seguridad. Las
\glspl{gl:semilla} tienen una vida finita que depende el mecanismo \gls{gl:drbg}
utilizado. Es imperativo que las implementaciones respeten el límite de la vida
de las \glspl{gl:semilla} especificado para el mecanismo; y, cuando se alcance
el límite de la vida de una \gls{gl:semilla}, el \gls{gl:drbg} no debe generar
salidas hasta que se haya cambiado la \gls{gl:semilla} o se cree una nueva
instancia del \gls{gl:drbg} (aunque se prefiere que se cambie la
\gls{gl:semilla}). Una \gls{gl:semilla} jamás debe ser utilizada para
inicializar o cambiar la semilla de otra instancia del \gls{gl:drbg} o la suya.

La cadena de personalización (que es opcional pero recomendada) es utilizada
para derivar la \gls{gl:semilla}, puede ser obtenida dentro o fuera del
módulo criptográfico y hasta puede ser una cadena vacía, pues el \gls{gl:drbg}
no depende de esta cadena para obtener \gls{gl:entropia}. De hecho, el que
el adversario conozco la cadena de personalización no disminuye el nivel de
seguridad de una instancia de \gls{gl:drbg} siempre y cuando la entrada de
\gls{gl:entropia} se mantenga desconocida. Esta cadena no es considerada un
\gls{gl:csp}; puede introducir datos adicionales al \gls{gl:drbg}, tales como
identificadores de usuario, aplicación, versiones, protocolos, marcas de tiempo,
direcciones de red, números de serie, etćetera.

\paragraph{Funciones}

\paragraph{Mecanismos basados en funciones hash}

\paragraph{Mecanismos basados en cifradores por bloque}

\paragraph{Funciones auxiliares}

\paragraph{Garantías}

\paragraph{Documentación mínima}

\paragraph{Pruebas}

\paragraph{Errores}
