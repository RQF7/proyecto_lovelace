%
% Recomendaciones del NIST para la generación de bits pseudoaleatorios,
% capítulo de análisis y diseño para la generación de tokens,
% Mecanismos con hash
% Proyecto Lovelace.
%
\paragraph{Mecanismos basados en funciones hash}

Los mecanismos \gls{gl:drbg} pueden estar basados en funciones hash de un
solo sentido; dos mecanismos basados en estas funciones son los HASH\_DRBG 
y HMAC\_DRBG. El nivel de seguridad que puede soportar cada uno es el nivel de
resistencia a la \gls{gl:preimagen} que tiene la función hash (véase sección
\ref{sec:hash}).
El mecanismo HASH\_DRBG requiere el uso de una misma función hash en las 
funciones de instanciación, cambio de \gls{gl:semilla} y generación. El nivel
de seguridad de la función hash a utilizar debe igualar o ser mayor al
nivel que se requiere por la aplicación consumidora del \gls{gl:drbg}. En
cambio, el mecanismo HMAC\_DRBG utiliza múltiples ocurrencias de una función
hash con llave; la misma función hash debe ser utilizada a lo largo del proceso
de instanciación. Al igual que HASH\_DRBG, la función hash debe tener, al menos,
el mismo nivel de seguridad que la aplicación consumidora requiere para la 
salida del \gls{gl:drbg}.