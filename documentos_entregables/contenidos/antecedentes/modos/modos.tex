%
% Sección de modos de operación, capítulo de antecedentes.
% Proyecto Lovelace.
%

\newpage
\section{Modos de operación}

Por sí solos, los cifrados por bloques solamente permiten el cifrado y
descifrado de bloques de información de tamaño fijo. Para la mayoría de los
casos, menos de 256 bits\cite{modos_de_operacion}, lo cual es equivalente a
alrededor de 8 caracteres. Es fácil darse cuenta de que esta restricción no
es ningún tema menor: en la gran mayoría de las aplicaciones, la longitud de
lo que se quiere ocultar es arbitraria.

Los modos de operación permiten extender la funcionalidad de los cifrados por
bloques para poder aplicarlos a información de tamaño irrestricto. Formalizamos
este concepto definiendo a un cifrado por bloques como una función $ C $
(ecuación \ref{cifrado_por_bloques}) y a un modo de operación como una función
$ M $ (ecuación \ref{modo_de_operacion}).

\begin{equation}
  \label{cifrado_por_bloques}
  C(L, B) \rightarrow Bc
\end{equation}

En donde $ L $ es la llave y $ B $ es el bloque a cifrar; ambos con un tamaño
definido: $ L \in \{0, 1\}^k $ ($ k $ es el tamaño de la llave) y
$ B \in \{0, 1\}^n $ ($ n $ es el tamaño de bloque). $ Bc $ representa al
bloque cifrado, el cuál también tiene longitud $ n $.

\begin{equation}
  \label{modo_de_operacion}
  M(L, T) \rightarrow Tc
\end{equation}

En este caso $ L $ es la misma que en \ref{cifrado_por_bloques}, $ T $ y
$ Tc $ son el texto original y el texto cifrado, respectivamente, y ambos
son de longitud arbitraria: $ T, Tc \in \{0, 1\}^* $.

Un primer enfoque (y quizás el más intuitivo) es partir el mensaje original
en bloques del tamaño requerido y después aplicar el algoritmo a cada bloque
por separado; en caso de que la longitud del mensaje no sea múltiplo del
tamaño de bloque, se puede agregar información extra al último bloque para
completar el tamaño requerido. Este es, de hecho, el primero de los modos que
presentamos a continuación (\textit{Electronic Codebook}, ECB); su uso no es
recomendado, pues es muy inseguro cuando el mensaje original es simétrico a
nivel de bloque \cite{modos_de_operacion}. También presentamos otros tres
modos, los cuales junto con ECB, son los más comunes.

% TODO: indagar un poco más en la inseguridad de ECB (dentro de su propia
% scción).

%
% Modo de operación ECB, capítulo de antecedentes.
% Proyecto Lovelace.
%

\subsubsection{\textit{Electronic Codebook} (ECB)}

La figura \ref{figura:ecb} muestra un diagrama esquemático de este modo de
operación. Según la ecuación \ref{modo_de_operacion}, el algoritmo recibe a la
entrada una llave y un mensaje de longitud arbitraria: la llave se pasa
sin ninguna modificación a cada función del cifrado por bloques; el mensaje
se debe de partir en bloques ($ T = B_1 || B_2 || \dots || B_n $).

\begin{figure}[H]
  \centering
  \begin{subfigure}{0.45\textwidth}
      \begin{center}
          \includegraphics[width=0.7\linewidth]
            {contenidos/antecedentes/bloques/modos/diagramas/modo_ecb.png}
          \caption{Cifrado.}
      \end{center}
  \end{subfigure}
  \begin{subfigure}{0.45\textwidth}
      \begin{center}
          \includegraphics[width=0.7\linewidth]
            {contenidos/antecedentes/bloques/modos/diagramas/modo_ecb_inverso.png}
          \caption{Descifrado.}
      \end{center}
  \end{subfigure}
  \caption{Modo de operación ECB.}
  \label{figura:ecb}
\end{figure}

\begin{pseudocodigo}[caption={Modo de operación ECB, cifrado.}]
  entrada: llave $ L $; bloques de mensaje $ B_1, B_2 \dots B_n $.
  salida:  bloques de mensaje cifrado $ Bc_1, Bc_2 \dots Bc_n $.
  inicio
    para_todo $B$
      $Bc_i$ $\gets$ C($L$, $B_i$)
    fin
    regresar $Bc$
  fin
\end{pseudocodigo}

\begin{pseudocodigo}[caption={Modo de operación ECB, descifrado.}]
  entrada: llave $ L $; bloques de mensaje cifrado $ Bc_1, Bc_2 \dots Bc_n $.
  salida:  bloques de mensaje original $ B_1, B_2 \dots B_n $.
  inicio
    para_todo $Bc$
      $B_i$ $\gets$ $C^{-1}$($L$, $Bc_i$)
    fin
    regresar $B$
  fin
\end{pseudocodigo}

%
% Modo de operación CBC, capítulo de antecedentes.
% Proyecto Lovelace.
%

\subsubsection{\textit{Cipher-block Chaining} (CBC)}
\label{sec:cbc}

\nopagebreak[4]
En \gls{gl:cbc} la salida del bloque uno se introduce (junto con
el siguiente bloque del mensaje) en el bloque dos, y así en sucesivo.
Para poder replicar este comportamiento en todos los bloque, este
\gls{gl:modo_de_operacion} necesita un argumento extra a la entrada: un
\gls{gl:vector_de_inicializacion}. De esta manera la salida del bloque $ i $
depende de todos los bloques anteriores; esto incrementa la seguridad con
respecto a \gls{gl:ecb}.

\begin{figure}
  \centering
  \begin{subfigure}{0.45\textwidth}
    \begin{center}
      \subimport{diagramas/}{modo_cbc.tikz.tex}
      \caption{Cifrado.}
    \end{center}
  \end{subfigure}
  \begin{subfigure}{0.45\textwidth}
    \begin{center}
      \subimport{diagramas/}{modo_cbc_inverso.tikz.tex}
      \caption{Descifrado.}
    \end{center}
  \end{subfigure}
  \caption{\Gls{gl:modo_de_operacion} \gls{gl:cbc}.}
  \label{figura:cbc}
\end{figure}

En la figura~\ref{figura:cbc} se muestran los diagramas esquemáticos para
cifrar y descifrar; en los pseudocódigos~\ref{cbc:1} y~\ref{cbc:2} se muestran
unos de los posibles algoritmos a seguir. Es importante notar que mientras que
el proceso de cifrado debe ser forzosamente secuencial (por la dependencias
entre salidas), el proceso de descifrado puede ser ejecutado en paralelo.

% Lástima que con los escapes en modo matemático dentro de los pseudocódigos
% se pierda totalmente la ventaja de trabajar con fuentes mono: por eso
% la alineación tan rara de los comentarios del próximo pseudocódigo.

\begin{pseudocodigo}[%
    caption={\Gls{gl:modo_de_operacion} \gls{gl:cbc}, cifrado.},
    label={cbc:1}%
  ]
    entrada: llave $ k $; vector de inicialización $ VI $;
             bloques de mensaje $ Bm_1, Bm_2 \dots Bm_n $.
    salida:  bloques de mensaje cifrado $ Bc_1, Bc_2 \dots Bc_n $.
    inicio
      $Bc_0$ $\gets$ $ VI $                        // El vector de inicialización
      para_todo $Bm$               // entra al primer bloque.
        $Bc_i$ $\gets$ E($k$, $Bm_i \oplus Bc_{i - 1}$)
      fin
      regresar $Bc$
    fin
\end{pseudocodigo}

\begin{pseudocodigo}[%
    caption={\Gls{gl:modo_de_operacion} \gls{gl:cbc}, descifrado.},
    label={cbc:2}%
  ]
    entrada: llave $ k $; vector de inicialización $ VI $;
             bloques de mensaje cifrado $ Bc_1, Bc_2 \dots Bc_n $.
    salida:  bloques de mensaje original $ Bm_1, Bm_2 \dots Bm_n $.
    inicio
      $Bc_0$ $\gets$ $ VI $
      para_todo $Bc$
        $Bm_i$ $\gets$ $D_k$($Bc_i$) $\oplus$ $Bc_{i-1}$
      fin
      regresar $Bm$
    fin
\end{pseudocodigo}

\paragraph{CBC-MAC}
Este algoritmo está basado en el modo de operación \gls{gl:cbc} y una función
$F$ que puede ser, por ejemplo, un cifrado por bloques. Se encarga de
cifrar con una llave $l_1$ todo el mensaje $m$, pero lo único que se toma en
cuenta es el último bloque, que es tomado como el código de autenticación
(véase figura~\ref{mac:cbc1}). En algunos casos, se cifra de nuevo, con la
misma función $F$, pero utilizando una llave $l_2$ distinta (véase
figura~\ref{mac:cbc2}).

CBC-MAC es considerado seguro cuando el mensaje $m$ tiene una longitud
múltiplo del tamaño de bloque del cifrado por bloques. Como se le han
encontrado varias vulnerabilidades, algoritmos basados en CBC-MAC han sido
propuestos, tales como el \textit{eXtended CBC} o el \gls{gl:omac}.

\begin{figure}
  \begin{center}
    \includegraphics[width=0.9\linewidth]{diagramas/cbcmac.png}
    \caption{Esquema de CBC-MAC simple.}
    \label{mac:cbc1}
  \end{center}
\end{figure}

\begin{figure}
  \begin{center}
    \includegraphics[width=0.9\linewidth]{diagramas/cbcmaclb.png}
    \caption{Esquema de CBC-MAC con el último bloque cifrado.}
    \label{mac:cbc2}
  \end{center}
\end{figure}

%
% Modo de operación CFB, capítulo de antecedentes.
% Proyecto Lovelace.
%

\subsubsection{\textit{Cipher Feedback} (CFB)}

Al igual que la operación de cifrado de CBC, ambas operaciones de CFB (cifrado
y descifrado) están encadenadas bloque a bloque, por lo que son de naturaleza
secuencial. En este caso, lo que se cifra en el primer paso es el vector de
inicialización; la salida de esto se opera con un \verb|xor| sobre el primer
bloque de texto en claro, para obtener el primer bloque cifrado (figura
\ref{fig:cfb}).

Esta distribución presenta varias ventajas con respecto a CBC: las operaciones
de cifrado y descifrado son sumamente similares, lo que permite ser
implementadas por un solo algoritmo (pseudocódigo \ref{cfb:1}); tanto para
cifrar como para descifrar solamente se ocupa la operación de cifrado del
algoritmo a bloques subyacente. Estas ventajas se deben principalmente a las
propiedades de la operación \verb|xor| (ecuación \ref{xor:inverso_igual}).

\begin{equation}
  \label{xor:inverso_igual}
  A \oplus B = C \quad \Rightarrow \quad A = B \oplus C
\end{equation}

\begin{figure}[H]
  \centering
  \begin{subfigure}{0.45\textwidth}
      \begin{center}
          \includegraphics[width=0.6\linewidth]
            {contenidos/antecedentes/bloques/modos/diagramas/modo_cfb.png}
          \caption{Cifrado.}
      \end{center}
  \end{subfigure}
  \begin{subfigure}{0.45\textwidth}
      \begin{center}
          \includegraphics[width=0.6\linewidth]
            {contenidos/antecedentes/bloques/modos/diagramas/modo_cfb_inverso.png}
          \caption{Descifrado.}
      \end{center}
  \end{subfigure}
  \caption{Modo de operación CFB.}
  \label{fig:cfb}
\end{figure}

\begin{pseudocodigo}[caption={Modo de operación CFB (cifrado y descifrado).}, label={cfb:1}]
  entrada: llave $ L $; vector de inicialización $ VI $;
           bloques de mensaje (cifrado o descifrado) $ B_1, B_2 \dots B_n $.
  salida:  bloques de mensaje (cifrado o descifrado) $ Bc_1, Bc_2 \dots Bc_n $.
  inicio
    $Bc_0$ $\gets$ $ VI $
    para_todo $B$
      $Bc_i$ $\gets$ C($L$, $Bc_{i - 1}$) $\oplus$ $B_i$
    fin
    regresar $Bc$
  fin
\end{pseudocodigo}

%
% Modo de operación OFB, capítulo de antecedentes.
% Proyecto Lovelace.
%

\newpage
\subsection{\textit{Output Feedback} (OFB)}

Este es muy similar al anterior (CFB), salvo porque la retroalimentación va
directamente de la salida del cifrador a bloques. De esta forma, nada que
tenga que ver con el texto en claro, llega al cifrado a bloques; este
solamente se la pasa cifrando una y otra vez el vector de inicialización.

\vspace{0.5cm}

\begin{figure}[H]
  \centering
  \begin{subfigure}{0.45\textwidth}
      \begin{center}
          \includegraphics[width=0.7\linewidth]
            {contenidos/antecedentes/modos/diagramas/modo_ofb.png}
          \caption{Cifrado.}
      \end{center}
  \end{subfigure}
  \begin{subfigure}{0.45\textwidth}
      \begin{center}
          \includegraphics[width=0.7\linewidth]
            {contenidos/antecedentes/modos/diagramas/modo_ofb_inverso.png}
          \caption{Descifrado.}
      \end{center}
  \end{subfigure}
  \caption{Modo de operación OFB.}
\end{figure}

\begin{pseudocodigo}[caption={Modo de operación OFB (cifrado y descifrado).}]
  entrada: llave $ L $; vector de inicialización $ VI $;
           bloques de mensaje (cifrado o descifrado) $ B_1, B_2 \dots B_n $.
   salida: bloques de mensaje (cifrado o descifrado) $ Bc_1, Bc_2 \dots Bc_n $.
  inicio
    aux $\gets$ $ VI $
    para_todo $B$
      aux $\gets$ C($L$, aux)
      $Bc_i$ $\gets$  aux $\oplus$ $B_i$
    fin
    regresar $Bc$
  fin
\end{pseudocodigo}

