%
% Sección de funciones hash, capítulo de antecedentes.
% Proyecto Lovelace.
%
\newpage
\section{Funciones hash}

Se refiere al conjunto de funciones computacionalmente eficientes que
mapean cadenas binarias de una longitud arbitraria a cadenas binarias
de una longitud fija, llamadas valores hash.

Matemáticamente, una función hash es una función
\begin{equation}
  \label{funcion_hash_def}
 	h: \{0, 1\}^* \longrightarrow \{0,1\}^n 
 	m \longmapsto h(m)
\end{equation}

La longitud de $n$ suele ser entre 128 y 512 bits. Las funciones hash 
$h$ tienen las siguientes propiedades:
\begin{enumerate}
	\item Compresión: $h$ mapea una entrada $x$ (cuya longitud
		finita es arbitraria) a una salida $h(x)$ de longitud fija $n$.
	\item Facilidad de cómputo: dada $x$ y $h$, $h(x)$ es 
		calculada ya sea sin necesitar mucho espacio, tiempo de cómputo, o 
		requiere pocas operaciones, etcétera.
\end{enumerate} 

De manera general, las funciones hash se pueden dividir en dos 
categorías: las que no utilizan llave y su único parámetro es la entrada
$x$, y las que necesitan una llave secreta $k$ y la entrada $x$.

Sea una función hash sin llave $h$ con entradas $x$, $x^\prime$ y 
salidas $y$ y $y^\prime$, respectivamente. A continuación se listan
algunas de las propiedades que puede tener:
\begin{enumerate}
	\item Resistencia de preimagen: no es computacionalmente factible
		para una salida específica $y$ encontrar una entrada $x^\prime$ que 
		dé como resultado el mismo valor hash $h(x^\prime) = y$ si no se
		conoce $x$. Esta propiedad también es llamada 
		\textit{de un sentido}.
	\item Resistencia de segunda preimagen: no es computacionalmente
		factible encontrar una segunda entrada $x^\prime$  que tenga la 
		misma salida que una entrada específica $x$: $x \neq x^\prime$
		tal que $h(x) = h(x^\prime)$. Esta propiedad también es conocida 
		como \textit{de débil resistencia a colisiones}.
	\item Resistencia a las colisiones: no es computacionalmente factible 
		econtrar dos entradas distintas $x$, $x^\prime$ que lleven al mismo 
		valor hash, o sea, $h(x) = h(x^\prime)$. A diferencia de la 
		anterior, la selección de ambas entradas no está restringida. Esta
		propiedad también es conocida como 
		\textit{de gran resistencia a colisiones}.
\end{enumerate}

Una función hash $h$ que cumple con las propiedades de resistencia de
preimagen y resistencia de segunda preimagen es conocida como una 
función hash de un solo sentido o OWHF (\textit{one-way hash function}
). Las que cumplen con la resistencia de segunda preimagen y 
resistencia a las colisiones son conocidas como funciones hash 
resistentes a colisiones o CRHF 
(\textit{collision resistant hash function}). Aunque casi siempre las 
funciones CRHF cumplen con la resistencia de preimagen, no es 
obligatorio que lo hagan.

Algunos ejemplos de las funciones hash criptográficas son el SHA-1 y el 
MD5. En los esquemas de firma electrónica, se obtiene el valor hash del
mensaje ($h(m)$) y se pone en el lugar de la firma. Los valores hash 
también son utilizados para revisar la integridad de las llaves 
públicas y, al utilizarse con una llave secreta, las funciones 
criptográficas hash se convierten en códigos de autenticación de 
mensaje (MAC, por sus siglas en inglés), una de las herramientas más 
utilizadas en protocolos como SSL e IPSec para revisar la integridad de 
un mensaje y autenticar al remitente.

Una de las aplicaciones más conocidas de las funciones hash es la de
cifrar las contraseñas: en un sistema, en vez de almacenar la contraseña
$clave$, se guarda su valor hash $h(clave)$. Así, cuando un usuario 
ingresa su contraseña, el sistema calcula su valor hash y lo compara con
el que se tiene guardado. Realizar esto ayuda a evitar que las 
contraseñas sean conocidas para los usuarios con privilegios, como 
pueden ser los administradores.

\subsection{Integridad de datos}
Las funciones criptográficas hash también son conocidas como funciones
\textit{procesadoras de mensajes} y el valor hash $h(m)$ de un mensaje
$m$ dado es llamado \textit{huella} de $m$; ya que es una representación
compacta de m y, dada la resistencia a la segunda preimagen, la huella 
es prácticamente única. Si el mensaje fuese modificado, el valor hash
sería distinto; por lo que si se tienen almacenados los valores hash, 
basta con calcular su valor $h(m)$ y compararlo con el que se tiene 
guardado para detectar modificaciones. Por esta razón, las funciones
hash también son llamadas códigos de detección de modificaciones (MDC, 
por sus siglas en inglés).

\subsection{Autenticación de mensajes}
Otra importante aplicación de las funciones hash es la autenticación 
del mensaje. Si una función hash es utilizada para la autenticación de 
mensajes, es llamada código de autenticación de mensajes (MAC, por sus
siglas en inglés). MAC es una técnica simétrica estándar muy utilizada 
para autenticar y proteger la integridad de un mensaje. Dependen de 
una llave secreta que tienen las partes interesadas y, al contrario de 
las firmas electrónicas donde solo una persona conoce la llave privada 
y es capaz de generar la firma, cada uno de los participantes puede 
producir el MAC válido para un mensaje. La llave privada $k$ es 
utilizada para parametrizar la función hash.

\begin{equation}
  \label{funcion_hash_mac}
 	(h_k: \{0, 1\}^* \longrightarrow \{0,1\}^n)_{k \in K}
\end{equation}

\subsection{Firmas}
Sea $(n, e)$ la llave pública RSA y $d$ el exponente decodificador 
secreto de Alice. En el esquema básico de firma RSA, Alice puede firmar 
mensajes que estén codificados por números $ m \in \{0, \dots, n-1\}$. 
Para firmar $m$, aplica el algoritmo de descifrado y obtiene la firma 
$\sigma = m^d$ $mod$ $n$ de $m$.
Normalmente, $n$ es un número de 1024 bits y Alice puede firmar una 
cadena de bits $m$ tal que, cuando es interpretada como número, sea 
menor que $n$. Esto es una cadena de, máximo, 128 caracteres ASCII: la 
mayoría de los documentos que se desean firmar suelen ser mucho más 
grandes. Este problema existe en todos los esquemas de firma digital y 
usualmente es resuelto al aplicar una función hash resistente a 
colisiones $h$. 
De esta forma, primero se obtiene el valor hash del mensaje $h(m)$ y 
esto es lo que se firma en lugar del mensaje mismo ($m$):

\begin{equation}
  \label{funcion_hash_sign}
 	\sigma = h(m)^d \quad mod \quad n
\end{equation}

Los mensajes que tengan el mismo valor hash tienen la misma firma. En 
este caso, es primordial que la función hash $h$ sea resistente a 
colisiones para garantizar el no repudio. De otra manera, Alice podría 
firmar el mensaje $m$ y después decir que había firmado un mensaje 
distinto ($n$).
La resistencia a segundas preimágenes previene que un atacante Eve tome 
un mensaje $m$ firmado por Alice, genere un mensaje nuevo $n$ y utilice 
$\sigma$ como una firma válida de Alice para $n$.