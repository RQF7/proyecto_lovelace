%
% Sección de composición del número de una tarjeta, capítulo de antecedentes.
% Proyecto Lovelace.
%

\section{Composición del número de una tarjeta}
Como se puede observar en la figura~\ref{figura:pan}, un número de tarjeta
\gls{gl:pan}, se compone por tres partes: el número identificador del emisor
(\gls{gl:iin}), el número de cuenta y un dígito verificador; la longitud del
\gls{gl:pan} puede variar e ir desde los 12 hasta los 19 dígitos. A
continuación se explica con más detalle cada uno de sus componentes

\begin{figure}[H]
  \begin{center}
    \includegraphics[width=0.4\linewidth]{diagramas/tarjeta}
    \caption{Componentes de un número de tarjeta.}\label{figura:pan}
   \end{center}
\end{figure}

\subsection{Identificador del emisor}
El \gls{gl:iin} comprende los primeros 6 dígitos del número de
tarjeta. El primer dígito es conocido como \gls{gl:mii} y se encarga de
identificar la industria a la que pertenece el emisor; en la
tabla~\ref{tabla:mii}, se puede observar la relación entre el dígito y el giro.
El identificador del emisor provee, entre otros, los siguientes datos:
\begin{enumerate}
    \item Emisor de la tarjeta
    \item Tipo de la tarjeta (ej. crédito o débito)
    \item Nivel de la tarjeta (ej. clásica, Gold, Black)
\end{enumerate}

\begin{table}[H]
  \centering
  \begin{tabular}{ c|l }
    Dígito & Industria \\ \hline
    1, 2 & Aerolíneas \\
    3 & Viajes y entretenimiento (ej. American Express) \\
    4, 5 &  Bancos e industria financiera (ej. Visa, Mastercard)\\
    6 & Comercio (ej. Discover) \\
    7 &  Industria petrolera \\
    8 &  Telecomunicaciones \\
    9 &  Asignación nacional \\
    \end{tabular}
    \caption{Identificador de industria (\gls{gl:mii}).}\label{tabla:mii}
\end{table}

\subsection{Número de cuenta}
Todos los dígitos posteriores al \gls{gl:iin} y anteriores al último dígito,
comprenden el número de cuenta. La longitud de este puede variar, pero, máximo
comprende 12 dígitos, por lo que cada emisor tiene $10^{12}$ posibles números
de cuenta.

\subsection{Dígito verificador}
Finalmente, se tiene el dígito verificador; este toma en cuenta todos los
dígitos anteriores y se calcula mediante el algoritmo de Luhn, que se describe
a continuación:
\begin{enumerate}
    \item Comenzando desde la derecha, se obtiene el doble de cada segundo
      dígito. Si el producto es mayor a $9$, se suman sus dígitos (ver
      figura~\ref{fig:luhn1}).
      \begin{figure}[H]
        \begin{center}
          \includegraphics[width=0.3\linewidth]{diagramas/luhn_1.png}
        \end{center}
        \caption{Primeros pasos del algoritmo de Luhn.}\label{fig:luhn1}
      \end{figure}
    \item Se suman todos los dígitos.
    \item Se multiplica la suma por $9 \mod  10$ (ver figura~\ref{fig:luhn2}).
      \begin{figure}[H]
        \begin{center}
          \includegraphics[width=0.5\linewidth]{diagramas/luhn_2.png}
        \end{center}
        \caption{Últimos pasos del algoritmo de Luhn.}\label{fig:luhn2}
      \end{figure}
\end{enumerate}
