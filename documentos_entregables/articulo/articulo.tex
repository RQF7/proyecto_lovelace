%
% Archivo principal de artículo.
% Proyecto Lovelace.
%

\documentclass[11pt]{article}
\usepackage[utf8]{inputenc}

\title{A Survey of Tokenization Methods}
\author{Daniel Ayala Zamorano
   \and Laura Natalia Borbolla Palacios
   \and Ricardo Quezada Figueroa}
\date{June, 2018}

% http://ctan.math.washington.edu/tex-archive/
% macros/latex/contrib/import/import.pdf
% Establece inputs relativos con respecto a una carpeta.
\RequirePackage{import}

% http://mirrors.ctan.org/macros/latex/contrib/xcolor/xcolor.pdf
\RequirePackage{xcolor}

% http://mirrors.ctan.org/macros/latex/contrib/hyperref/doc/manual.pdf
\RequirePackage{hyperref}
\hypersetup
{
  pdftitle={A Survey on Tokenization Methods},
  bookmarksnumbered=true,
  bookmarksopen=true,
  bookmarksopenlevel=1,
  pdfstartview=Fit,
  pdfpagemode=UseOutlines,
  linktocpage=true,
  breaklinks=true,
  linkbordercolor=white,
  citebordercolor=white,
  filebordercolor=white,
  menubordercolor=white,
  urlbordercolor=white,
  pdfborder={0 0 0}
}

% http://mirrors.ctan.org/macros/latex/contrib/biblatex/doc/biblatex.pdf
% Gestión de bibliografía.
\RequirePackage[backend=biber,
                style=numeric,
                sorting=none,
                citestyle=ieee,
                backref=true
                ]{biblatex}
\addbibresource{referencias.bib}

% http://mirrors.ctan.org/macros/latex/contrib/glossaries/glossariesbegin.pdf
\RequirePackage[acronym,
                section=section
                ]{glossaries}
%
% Definición de entradas del glosario.
% Proyecto Lovelace.
%
% Más información en https://www.sharelatex.com/learn/Glossaries
%

\makeglossaries

%% Glosario de términos %%%%%%%%%%%%%%%%%%%%%%%%%%%%%%%%%%%%%%%%%%%%%%%%%%%%%%%

%\newglossaryentry{latex}
%{
%  name=latex,
%  description={Is a mark up language specially suited
%  for scientific documents}
%}
%
%\newglossaryentry{maths}
%{
%  name=mathematics,
%  description={Mathematics is what mathematicians do}
%}

\newglossaryentry{gl:modo_de_operacion}
{
  name = modo de operación,
  description = {
    Permite extender la funcionalidad de un cifrado a bloques para operar
    sobre tamaños de información arbitrarios
  }
}

%% Siglas %%%%%%%%%%%%%%%%%%%%%%%%%%%%%%%%%%%%%%%%%%%%%%%%%%%%%%%%%%%%%%%%%%%%%

%\newacronym{gcd}{GCD}{Greatest Common Divisor}
%
%\newacronym{lcm}{LCM}{Least Common Multiple}

\newacronym{gl:ecb}{ECB}{\textit{Electronic Codebook}}

%
% Definición de siglas y acrónimos.
% Proyecto Lovelace.
%

\subimport{/}{criptograficos}
\subimport{/}{bancarios}
\subimport{/}{computacionales}
\subimport{/}{instituciones}


\begin{document}

  \maketitle

  %
  % Key points (abstract and introduction).
  %
  % Structure of abstract: first talk about the problematic, then about
  % what we do/did/doing to face it.
  %
  % What is the problematic?
  % A lot of misinformation around the tokenization problem. Companies
  % using the misinformation to sell his products/services. The PCI providing
  % and using a classification that don't help to clarify what tokenization is.
  %
  % To provide context:
  % First, the internet commerce gains a lot of popularity. Then come a lot
  % of frauds cards because the online stores weren't secured. Arises
  % organizations to rule the functioning of the transactions; the PCI SSC,
  % the winner organization, establishes a standard that all the entities related
  % to card transactions must follow. The standard is hard and expensive to
  % comply, especially to small stores. To help the stores to be PCI complience
  % arise a new paradigm: the tokenization, i. e. the replacement of the
  % sensitive information, the credit cards numbers, with surrogate useless
  % (for an attacker) values, namely, the tokens.
  %
  % What we are doing?
  % -> Explain what is tokenization and what is his relation to cryptography.
  % -> Provide an alternative to the PCI classification.
  % -> Enlist some of the more common methods.
  % -> Explain the general functioning of each of the methods.
  % -> Compare the performance of the algorithms.
  % -> Discuss about the advantages and disadvantages of each category and,
  %    in particular, of each method.
  %

  \begin{abstract}
    When the online commerece begin to arise, the credit card frauds become a
    very frequent problem. For this reason, the Payment Card Industry (PCI)
    Secuirity Standard Council (SSC), did a standard for rule the function of
    any entity related with the processing of payments through the internet.
    This standard has a lot of requirements, and for a small store, it's hard
    and expensive to be PCI compliance. In the last years, a process named
    tokenization has become a very popular solution for online stores to
    reduce their PCI scope. Regretabbly, there is a lot of misinformation
    around this subject and the PCI guides to tokenization don't help to
    clarify. In this paper we explain what tokenization is and its
    relation to cryptography. Over this line, we point out what is the problem
    with PCI DSS clasification and provide a more logic one. We analyze and
    compare the more common tokenization methods and conclude with a discussion
    on the advantages and disadvantages of each one.
  \end{abstract}

  \section{Introduction}

  \section{Preliminaries}

  \section{Reversible methods}

  \subsection{\Glsentrytext{gl:ffx}}

  References: \cite{ffx_1, ffx_2, sinopsis_rogaway}. \Gls{gl:ffx}.

  \subsection{\Glsentrytext{gl:bps}}

  References: \cite{bps}. \Gls{gl:bps}.

  \section{Irreversible methods}

  \subsection{TKR}

  References: \cite{doc_sandra}.

  \subsection{\Glsentrytext{gl:rha}}

  References: \cite{aragona}. \Gls{gl:rha}.

  \subsection{\Glsentrytext{gl:uto}}

  References: \cite{DBLP:conf/fc/CachinCFL17}. \Gls{gl:uto}.

  \subsection{\Glsentrytext{gl:drbg}}

  References: \cite{nist_aleatorios}. \Gls{gl:drbg}.

  \section{Experimental results}

  \section{Conclusion}

  \printbibliography

\end{document}
