%
% Sección de preliminares.
% Artículo.
% Proyecto Lovelace.
%

\section{Preliminares}

% TODO: ¿qué más va en los preliminares? ¿AES?
% TODO: Terminar de definir a los cifradores por bloque.

% Creo que cifrador no existe.
% Hasta donde he visto, cuando menos en algunas entradas de wikipedia, es que
% usan «cifrado» (¿escrito con cifras?) en su lugar.

\subsection{Notación}

Denotaremos a todas las cadenas de bits de longitud $ n $ como $ \{ 0, 1 \}^n $.
Un algoritmo tokenizador es una función $ E: \mathcal{X} \rightarrow
\mathcal{Y} $ en donde los conjuntos $ X $ y $ Y $ son el espacio de números de
tarjetas y el de tokens, respectivamente.

\subsection{Algoritmo de Luhn}

\subsection{Estructura de un número de tarjeta de crédito}

\subsection{Cifrado por bloques}

Un cifrado por bloques es un cifrado simétrico que se define por la función $ E:
\mathcal{M} \times \mathcal{K} \rightarrow \mathcal{C} $ en donde $ \mathcal{M} $
es el espacio de textos en claro, $ \mathcal{K} $ es el espacio de llaves y $
\mathcal{C} $ es el espacio de mensajes cifrados. Tanto los mensajes en claro
como los cifrados tienen una misma longitud $ n $, que representa el tamaño del
bloque \cite{menezes}.

Los cifrados por bloque son un elemento de construcción fundamental para otras
primitivas criptográficas. Muchos de los algoritmos tokenizadores que se
presentan en este trabajo los ocupan de alguna forma. Las definiciones de los
algoritmos son flexibles en el sentido de que permiten instanciar cada
implementación con el cifrado por bloques que se quiera; en el caso de las
implementaciones hechas para este trabajo se ocupó AES (\textit{Advanced
Encryption Standard}) en la mayoría de los casos.

\subsection{Cifrado que preserva el formato}

% Coregir letra del contradominio de la función de FPE e investigar qué
% demonios es.
% Pues no entiendo por el contexto qué significa; de momento la borré.

% ¿Tweak va en itálicas?
% http://www.rae.es/consultas/los-extranjerismos-y-
% latinismos-crudos-no-adaptados-deben-escribirse-en-cursiva
% Pues, a diferencia de token, me parece que con la «tw» sí hacemos uso de
% la pronunciación en inglés; cuando menos ahora no se me ocurre una palabra
% en español que tenga «tw».

Un cifrado que preserva el formato puede ser visto como un cifrado simétrico en
donde el mensaje en claro y el mensaje cifrado mantienen un formato en común.
Formalmente, de acuerdo a lo definido en \cite{DBLP:conf/sacrypt/BellareRRS09},
se trata de una función $ E: \mathcal{K} \times \mathcal{N} \times \mathcal{T}
\times \mathcal{X} \rightarrow \mathcal{X} $, en donde los conjuntos $
\mathcal{K} $, $ \mathcal{N} $, $ \mathcal{T} $, $ \mathcal{X} $ corresponden al
espacio de llaves, espacio de formatos, espacio de \textit{tweaks} y el dominio,
respectivamente. El proceso de cifrado de un elemento del dominio con respecto a
una llave $ K $, un formato $ N $ y un \textit{tweak} $ T $ se escribe como  $
E_K^{N,T}(X) $. El proceso inverso es también una función $ D: \mathcal{K}
\times \mathcal{N} \times \mathcal{T} \times \mathcal{X} \rightarrow
\mathcal{X} $, en donde $ D_K^{N,T}\big( E_K^{N,T}(X) \big) = X $.

% ¿Hacer aquí aclaración sobre la diferencia en los dígitos verificadores
% en los espacios de orígen y destino?
% Creo que en realidad eso le aplica a todos los esquemas, por lo que mejor lo
% ponemos en el propio desarrollo.

Para lo que a este trabajo respecta, el formato usado es el de las tarjetas de
crédito: una cadena de entre 12 y 19 dígitos decimales. Esto es $ N = \{0, 1,
\dots, 9\}^n $ en donde $ 12 \leq n \leq 19 $.

% TODO: Hmmmm... técnicamente al FPE solo le damos el identificador de la
% persona, por lo que los dígitos son menos.
