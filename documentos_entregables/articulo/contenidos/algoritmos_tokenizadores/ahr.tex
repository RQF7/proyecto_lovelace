%
% Sección de FFX.
% Artículo.
% Proyecto Lovelace.
%

\subsection{AHR (Algoritmo híbrido reversible)}
En 2017, Longo, Aragona y Sala~\cite{aragona} propusieron un algoritmo híbrido
reversible basado en un cifrador por bloques, una llave secreta y una entrada
adicional. Las entradas del algoritmo son la parte del PAN a cifrar y una
entrada adicional (por ejemplo, la fecha) que permite que se tengan varios
tokens relacionados con la misma tarjeta.

El algoritmo necesita una función $ f $ pública que se encarga de poner el
relleno en la entrada para obtener el bloque completo para el cifrador, pues,
dada una cadena de longitud $ m $ regrese una de longitud $ n $; requiere
también que solo se utilicen cifrados cuyo tamaño de bloque sea, mínimo, de 128
bits. Finalmente, como es un algoritmo reversible, se necesita una base de datos
segura para almacenar los pares PAN-token.

Como se desea obtener un token que tenga el mismo número de dígitos que el PAN
ingresado, se utiliza un método llamado \textit{cycle-walking} para asegurarse
de que el texto cifrado pertenezca al espacio del texto en claro.

A continuación se definen una serie de notaciones que se utilizarán en el
algoritmo:
\begin{itemize}
  \item $ M $ Tamaño de bloque del cifrado por bloques que se usará.
  \item $ l $ Longitud de la entrada. En este caso, $13 \geq l \geq 19$.
  \item $ n $ Número de bits necesarios para representar a la entrada:
    $n = log_2(10^l)$.
  \item $ [y]^s_b $ Indica que $y$ es menor que $b^s$: $y < b^s$.
  \item $\bar{x}$ Representación de $x$ en una cadena binaria cuando $x$ es
    representado en su forma decimal y viceversa.
\end{itemize}

El primer paso es obtener el valor del bloque $ t $; es decir, concatenar los
bytes más significativos de la salida de $ f (u,p) $ con la representación
binaria de $ p $. Después, se cifra el bloque $ t $ con la llave $ K $ y se
guarda en $ c $ la representación decimal de los  últimos bits del bloque
cifrado; aquí es donde se utiliza la caminata cíclica, pues si los dígitos de
$ c $ son menores a los que le corresponden con $ p $, se guarda $ t $ en $ c $
y se regresa al paso del cifrado. Finalmente, cuando se obtiene un token válido,
se comprueba que no esté registrado en la base de datos (si lo está, se regresa
al inicio, pero aumentando la entrada adicional $ u $ en $ 1 $) y se registra
el nuevo par PAN-token.
El pseudocódigo del algoritmo de tokenización se puede observar en~\ref{cod_ahr}.
\begin{figure}
  \begin{center}
    \begin{tabular}{|l|}
      \hline
      \begin{minipage}{220pt}
        {\scriptsize\begin{tabbing}
          \ \ \ \ \ \=\ \ \ \ \=\ \ \ \ \=\ \ \ \ \=\ \ \ \ \=\ \ \ \ \=\ \ \
          \ \kill \\
          \ \ \ \ {\bf Algoritmo} AHR($ p $, $ u $, $ k $)\\
          \> 1. \> $t = f(u, p) || [\bar{p}]^s_b$  \\
          \> 2. \> $c = E(k, t)$ \\
          \> 3. \> {\bf si} $(\bar{c} \mod 2^n) \geq 10^l$ {\bf entonces}:\\
          \> 4. \> \> $t = c$ \\
          \> 5. \> \> Regresar a 2. \\
          \> 6. \> $token = {[\bar{c} \mod 2^n]}^l_{10}$ \\
          \> 7. \> {\bf si} $comprobar(token) =$ verdadero {\bf entonces}: \\
          \> 8. \> \> $u = u + 1$ \\
          \> 9. \> \> Regresar al paso 1. \\
          \> 10. \> {\bf sino}: \\
          \> 11. \> \> {\bf regresar} $token$ \\
        \end{tabbing}}
      \end{minipage}\\
      \hline
    \end{tabular}
  \end{center}
  \caption{\label{cod_ahr} Algoritmo híbrido reversible.}
\end{figure}
