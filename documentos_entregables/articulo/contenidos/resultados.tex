%
% Sección de resultados de compraciones de desempeño.
% Artículo sobre tokenización.
%
% Proyecto Lovelace.
%

\section{Resultados y conclusiones}
\label{sec:conclusiones}

En la tabla \ref{tabla:tiempos_tokenizacion} y la figura
\ref{figura:tiempos_unitarios} se muestran los resultados en tiempo de
las ejecuciones de los algoritmos presentados en secciones
anteriores\footnote{También se hizo la implementación de un DRBG como lo
describe el NIST en \cite{nist_aleatorios}}. Estos se llevaron a cabo en una
computadora con las siguientes características:

\begin{itemize}
  \item \textbf{Procesador:} Intel i5-7200U (2.5 GHz) de 4 núcleos.
  \item \textbf{Sistema operativo:} Arch Linux, kernel 4.17.
  \item \textbf{Base de datos:} MariaDB 10.1.
  \item \textbf{Compilador:} GCC 8.1.1
\end{itemize}

La comparación de los tiempos de tokenización y detokenización muestra como los
algoritmos reversibles son considerablemente más rápidos que los irreversibles.
Este resultado puede resultar un poco contraintuitivo, pues la generación de
tokens reversibles involucra más operaciones; es por esto que en la figura
\ref{figura:tiempos_tokenizacion} se muestran los tiempos de la generación de
tokens solamente, sin tomar en cuenta tiempos de acceso a base de datos.

% Además de los tiempos de ejecución, también es importante señalar que los
% irreversibles, al operar como funciones de un solo sentido, son un poco más
% seguros que los reversibles: un atacante con acceso a la llave de cifrado puede
% obtener el número de tarjeta correspondiente si se trata de un método
% reversible, mientras que con un método irreversible necesita también acceso a la
% base de datos.

\begin{table}
  \renewcommand{\arraystretch}{1.3}
  \centering
  \caption{Comparación de tiempos de tokenización.}
  \label{tabla:tiempos_tokenizacion}
  \begin{tabular}{c c c}
    \hline
     & Tokenización ($\mu$s) & Detokenización ($\mu$s) \\
    \hline
    FFX & 83 & 64 \\\hline 
BPS & 247 & 127 \\\hline 
TKR & 46260 & 373 \\\hline 
AHR & 3427 & 390 \\\hline 
DRBG & 54060 & 387 \\\hline 

    \hline
  \end{tabular}
\end{table}

\begin{figure}[!t]
  \centering
  \subfloat[Tokenización y detokenización]{
    \includegraphics[width=1.62in]
      {../implementaciones/reportes/tiempos_unitarios.png}
    \label{figura:tiempos_unitarios}
  }
  \hfil
  \subfloat[Generación de tokens]{
    \includegraphics[width=1.62in]
      {../implementaciones/reportes/tiempos_tokenizacion.png}
    \label{figura:tiempos_tokenizacion}
  }
  \caption{Comparaciones de tiempos.}
  \label{fig_sim}
\end{figure}


% TODO: mejorar crítica a segunda categoría.

%La denominación \textit{no criptográficos} resulta totalmente confusa, pues en
%realidad todos los métodos conocidos que caen en las categorías de arriba ocupan
%primitivas criptográficas. La segunda categoría (los irreversibles) carece de
%utilidad para aplicaciones que procesan pagos con tarjetas de crédito, pues la
%habilidad de regresar al número de tarjeta a partir de su token es uno de los
%requerimientos principales para los sistemas tokenizadores. Por lo anterior, en
%este trabajo se propone una clasificación distinta:
%
%\begin{itemize}
%  \item Métodos criptográficos. Todos aquellos que ocupan herramientas
%    critográficas.
%    \begin{itemize}
%      \item Reversibles. Ocupan un esquema de cifrado simétrico: el número
%        de tarjeta y una llave entran al mecanismo de tokenización para obtener
%        un token; el token y la misma llave entran al mecanismo de
%        detokenización para obtener el número de tarjeta original.
%      \item Irreversibles. Ocupan herramientas criptográficas para generar el
%        token de un número de tarjeta. Ocupan herraminetas externas, como una
%        base de datos, para guardar las relaciones entre tokens y números de
%        tarjetas.
%    \end{itemize}
%  \item Métodos no criptográficos. Aquellos posibles métodos que no ocupen
%    herramientas relacionadas con la criptografía; por ejemplo, un generador de
%    números realmente aleatorio (TRNG, \textit{True Random Number Generator}).
%\end{itemize}
