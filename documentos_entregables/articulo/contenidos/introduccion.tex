%
% Sección de Introducción.
% Artículo.
% Proyecto Lovelace.
%

\section{Introducción}

% Contexto histórico: cómo llegó a ser popular la tokenización.

Cuando el comercio a través de Internet comenzó a popularizarse, los fraudes de
tarjetas bancarias se volvieron un problema alarmante: Visa y MasterCard
reportaron, entre 1988 y 1998, pérdidas de 750 millones de dólares; según
\cite{wallethub}, en el año 2001 se tuvieron pérdidas de 1.7 miles de millones
de dólares y, para 2002 aumentaron a 2.1. Como una medida para proteger los
sistemas procesadores de pagos, las principales compañías de tarjetas de crédito
publicaron un estándar obligatorio para todos aquellos que procesaran más de 20
000 transacciones anuales: el PCI DSS (en inglés, \textit{Payment Card Industry
Data Security Standard} \cite{pci_dss}).

El PCI DSS cuenta con una gran cantidad de requerimientos, entre ellos,
controles estrictos de acceso, monitoreo regular de redes, mantener programas de
vulnerabilidades y políticas de seguridad de información, etcétera
\cite{uk_association} \cite{search_security}; por lo que, para negocios
medianos y pequeños, resulta muy difícil obtener un resultado positivo. Además,
a pesar de la publicación del estándar en 2004, han seguido existiendo grandes
filtraciones de datos (\textit{TJX} en 2006, \textit{Hannaford Bros.} en 2008,
\textit{Target} en 2013, \textit{Home Depot} en 2014, por mencionar algnos
ejemplos).

% Introducción a los sistemas tokenizadores
% ¿Qué es la tokenización y cómo se usa normalmente (arquitectura).

Es en este contexto en el que surge el enfoque de la tokenización. Hasta antes
de este momento, la idea era proteger la información sensible en donde quiera
que se encontrase. Si los números de tarjetas de los usuarios se encontraban
regados en diversas partes de un sistema, había que proteger todas esas partes.
La tokenización consiste en concentrar la información sensible en un solo lugar
para hacer la tarea de protección más sencilla. Al momento de ingreso de un
nuevo valor sensible, por ejemplo, la información bancaria de un usuario, se
genera un token ligado a esa información: el token se usa en todo el sistema y
la información sensible se protege en un solo lugar. Un posible adversario con
acceso a los tokens no debe poder obtener la información sensible a partir de
estos.

Una de las ventajas de la tokenización es que puede verse como un sistema
autónomo, independiente del sistema principal. De esta manera se establece una
separación de responsabilidades: el sistema principal se ocupa de la operación
del negocio (por ejemplo, una tienda en línea) y el sistema tokenizador se
dedica a la protección de la información sensible. Hoy en día varias compañías
ofrecen serivicios de tokenización que permiten que los comerciantes se libren
casi por completo de cumplir con el PCI DSS. En la figura
\ref{figura:arquitectura_tokenizacion} se muestra una distribución bastante
común para un comercio en línea: el sistema tokenizador guarda la información
sensible en su base de datos y se encarga de realizar las transacciones
bancarias.

\begin{figure}
  \begin{center}
    \includegraphics[width=1.0\linewidth]
      {algoritmos_tokenizadores/diagramas/sistema_tokenizador.png}
    \caption{Arquitectura típica de un sistema tokenizador.}
    \label{figura:arquitectura_tokenizacion}
  \end{center}
\end{figure}

% Muestra del problema: ejemplos de discursos publicitarios comunes.

% Establecer relación entre criptografía y tokenización.

% Estructura del artículo.
