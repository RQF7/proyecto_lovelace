%
% Presentación de TT2.
% Proyecto Lovelace.
%
% Prototipo de tienda en línea (caso de prueba).
% Integración con el servicio web.
%

\subsection{Integración con el servicio web}

\begin{frame}{Integración con el servicio web}
  
  Esta tienda en línea se desarrolló con la finalidad de ser el caso de prueba
  del servicio web de tokenización.

  Las partes donde se integra el servicio son:

  \begin{itemize}
    \item \textbf{El registro de una forma de pago.}

      Se necesita el servicio de tokenización dado que no se desea guardar
      el número de tarjeta ingresado, sino su token.
      \newline

    \item \textbf{La realización de una compra.}

      Requiere de la detokenización, ya que se tiene almacenado el token de
      la tarjeta y se requiere el número original para la transacción bancaria.

  \end{itemize}

  \note{
    Dado la finalidad de de la implementación de esta tienda, que es el de ser
    un caso de prueba que nos permite corroborar el correcto funcionamiento del
    servicio web, los casos de uso que mas nos interesan son los de registrar
    una forma me pago, que es registrar un número de tarjeta de crédito, y el
    de la realización de una compra.

    Viendo a esta tienda como el software de un tercero, ajeno al servicio de
    tokenización. El registro de una forma de pago es importante debido a que
    la tienda desea delegar el la protección de los datos bancarios, razón por
    lo que no puede almacenar los datos de forma directa, sino sus tokens
    representativos, lo que se logra, haciendo una petición de tokenización.
    y guardado el valor regresado en ves de el número ingresado.

    En cuanto a la realización de una compra, también es relevante, dado que
    para hacer transacciones bancarias, son necesarios los número de las
    tarjetas, las cual no se tienen, pero se consiguen solicitando la
    detokenización del los tokens guardados como forma de pago.
  }

\end{frame}
