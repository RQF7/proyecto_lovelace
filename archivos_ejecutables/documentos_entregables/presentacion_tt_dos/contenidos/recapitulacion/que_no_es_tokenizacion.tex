%
% Presentación de TT2.
% Proyecto Lovelace.
%
% Recapitulación.
% Qué es la tokenización.
%

\subsection{¿Qué es la tokenización?}

\begin{frame}{¿Qué es la tokenización?}{Tokenización en otros contextos}
  \begin{itemize}
    \item Moneda de uso particular sin valor legal.
    \item Componente de seguridad en la comunicación por sesiones.
    \item Componente léxico de una gramática.
  \end{itemize}

  \note{
    \begin{itemize}
      \item Representaciones del dinero, como las fichas de un casino; o
      los mismos billetes, que solían representar una cantidad de oro.
      \item  Para evitar tener que autenticarse en cada interacción de
      un sistema, se inicia una sesión y se le da a los participantes
      un token de autenticación: es una suerte de identificación.
      \item En compiladores, un token es una cadena con un significado
      conocido.
      \item En procesamiento de lenguaje natural, las palabra
      lingüísticamente significativa.
      \item Se le dice tokenismo a la práctica de ser superficialmente
      diversos al incluir una minoría discriminada (racial, sexual). Los
      tokens son quienes pertenecen a la minoría; por ejemplo, que haya
      una mujer (entre 20 miembros) en la dirección o que en un programa
      haya un personaje gay entre 6 principales.
    \end{itemize}}
\end{frame}

\begin{frame}{¿Qué es la tokenización?}{Tokenización en criptografía}
  \begin{itemize}
    \item Es la sustitución de datos sensibles por valores representativos
      sin una relación directa.
    \item Existen muchas empresas que proveen el servicio de tokenización,
      pero lo hacen sin detallar la forma en la que se
      realiza~\cite{shif4_uno, braintree_uno, securosis}.
  \end{itemize}

  \begin{figure}
    \centering
    \subimport{diagramas/}{arquitectura.tikz.tex}
    \caption{Arquitectura de sistema tokenizador: operación de tokenización.}
  \end{figure}

  \note{
    \begin{itemize}
      \item Recalcar aquí que estamos tomando en cuenta el contexto de la
        criptografía Y en particular, el de la tokenización.
      \item Recordar que la tokenización es un paradigma RELATIVAMENTE
        nuevo (digo, ya tiene unos 10 años) y que cada quien lo hace como
        quiere (y como puede), así que el que no digan cómo están haciendo
        sus tokens (al menos públicamente) contribuye a la desinformación
        que aún rodea la tokenización.
    \end{itemize}}
\end{frame}
