%
% Clasificación de algoritmos tokenizadores,
% presentación en RCI.
% Proyecto Lovelace.
%

\section{Clasificación del PCI}

\begin{frame}{Clasificación de los algoritmos tokenizadores}
  {Clasificación del PCI \cite{pci_tokens}}
  \begin{itemize}
    \item \textbf{Reversibles:} se puede regresar, a partir del token, al
      número de tarjeta original.
      \begin{itemize}
        \item \textbf{Criptográficos:} cifran la tarjeta y descifran el
          token.
        \item \textbf{No criptográficos:} utilizan una base de datos para
          guardar la relación entre números de tarjeta y tokens.
      \end{itemize}
    \item \textbf{Irreversibles:} no se puede regresar al número de tarjeta a
      partir del token.
      \begin{itemize}
        \item \textbf{Autenticables:} permiten validar cuando un token
          corresponde a un número de tarjeta dado.
        \item \textbf{No autenticables:} no se puede hacer la validación
          anterior.
      \end{itemize}
  \end{itemize}
\end{frame}

\begin{frame}{Clasificación de los algoritmos tokenizadores}
  {Clasificación propuesta}
  \begin{itemize}
    \item \textbf{Criptográficos:} ocupan primitivas criptográficas en su
      operación.
    \begin{itemize}
      \item \textbf{Reversibles:} cifran la tarjeta y descifran el
        token.
      \item \textbf{Irreversibles:} requieren una base de datos para
        guardar la relación entre números de tarjetas y tokens.
    \end{itemize}
    \item \textbf{No criptográficos:} no utilizan nada relacionado con la
        criptografía.
  \end{itemize}
\end{frame}
