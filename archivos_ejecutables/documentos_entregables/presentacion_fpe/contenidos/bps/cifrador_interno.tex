%
% Descripción del cifrado BC de BPS, presentación de cifrados que preservan
% el formato.
% Proyecto Lovelace.
%

\subsection{Cifrado interno BC}

%------------------------------------------------------------------------------
\begin{frame}{BPS}{Cifrado interno $BC$}

  Este cifrado se define como:
    \begin{equation*}
      BC_{F,s,b,w}(X,K,T)
    \end{equation*}

  Donde:
  \begin{itemize}
    \item $F$ es un cifrado por bloques de $f$ bits de salida.
    \item $s$ es la cardinalidad del conjunto de caracteres $S$.
    \item $b$ es la longitud del bloque. ($b \leq 2 \cdot |log_s(2^{f-32})|$)
    \item $w$ es el número de rondas de la red Feistel interna. (par)
    \item $X$ es la cadena de longitud $b$ a cifrar.
    \item $K$ es una llave acorde a $F$.
    \item $T$ es un \textit{tweak} de 64 bits.
  \end{itemize}

\end{frame}

%------------------------------------------------------------------------------
\begin{frame}{BPS}{Cifrado interno $BC$}

  El cifrado $BC$ sigue el siguiente proceso para cifrar un bloque.

  \begin{enumerate}
    \item Dividir el \textit{tweak} $T$ en 2 \textit{subtweaks} $T_L$ y $T_R$
    de 32 bits.
      \begin{align*}
        T_R\: &=\: T\: \mod\: 2^{32}  \\
        T_L\: &=\: (T\: -\: T_R) / 2^{32}
      \end{align*}
  \end{enumerate}

\end{frame}

%------------------------------------------------------------------------------
\begin{frame}{BPS}{Cifrado interno $BC$}

  \begin{enumerate}
    \setcounter{enumi}{1}
    \item Dividir la cadena $X$ en 2 para obtener $X_L$ y $X_R$ con
      longitudes $l$ y $r$ respectivamente.

      Si $b$ es par:
      \begin{equation*}
        l =\: r\: =\: b/2
      \end{equation*}

      Si $b$ es impar:
      \begin{align*}
        l &=\: (b+1)/2 \\
        r &=\: (b-1)/2
      \end{align*}
  \end{enumerate}

\end{frame}

%------------------------------------------------------------------------------
\begin{frame}{BPS}{Cifrado interno $BC$}

  \begin{enumerate}
    \setcounter{enumi}{2}
    \item Definir e inicializar $L_0$ y $R_0$.
      \begin{align*}
        L_0\: &=\: \sum_{j=0}^{l-1} X_L[j] \cdot s^j \\
        R_0\: &=\: \sum_{j=0}^{r-1} X_R[j] \cdot s^j
      \end{align*}
  \end{enumerate}

\end{frame}

%------------------------------------------------------------------------------
\begin{frame}{BPS}{Cifrado interno $BC$}

  \begin{enumerate}
    \setcounter{enumi}{3}
    \item Partiendo de $i\: =\: 1$ hasta $i\: <\: w$:

      Si $i$ es par:
      \begin{align*}
        L_{i+1}\: &=\: L_i\: \boxplus\:
                      F_K((T_R \oplus i) \cdot 2^{f-32}\: +\: R_i)\quad
                      (mod\ s^l) \\
        R_{i+1}\: &=\: R_i
      \end{align*}

      Si $i$ es impar:
      \begin{align*}
        R_{i+1}\: &=\: R_i\: \boxplus\:
                      F_K((T_L \oplus i) \cdot 2^{f-32}\: +\: L_i)\quad
                      (mod\ s^r) \\
        L_{i+1}\: &=\: L_i
      \end{align*}
  \end{enumerate}

\end{frame}

%------------------------------------------------------------------------------
\begin{frame}{BPS}{Cifrado interno $BC$}

  \begin{figure}[H]
    \begin{center}
      \includegraphics[width=0.6\linewidth]
        {../../../diagramas_comunes/bps/cifrado_bc}
      \caption{Diagrama de rondas del cifrado.}
     \end{center}
  \end{figure}

\end{frame}

%------------------------------------------------------------------------------
\begin{frame}{BPS}{Cifrado interno $BC$}

  \begin{enumerate}
    \setcounter{enumi}{4}
    \item Descomponer $L_w$ y $R_w$ para obtener a $Y_L$ y a $Y_R$, las
      cuales concatenadas ($Y_L \parallel Y_R$) dan la cadena de salida $Y$.
  \end{enumerate}

    \begin{figure}[H]
      \begin{center}
        \includegraphics[width=0.75\linewidth]
          {../../../diagramas_comunes/bps/descomposicion}
        \caption{Proceso para descomponer $L_w$ y $R_w$.}
       \end{center}
    \end{figure}

\end{frame}

%------------------------------------------------------------------------------
\begin{frame}{BPS}{Descifrador $BC^{-1}$}

  Ahora, el proceso para descifrar la cadena $Y$ es:

  \begin{enumerate}
    \item Dividir $Y$ para obtener las subcadenas $Y_L$ y $Y_R$ con una
      longitud $l$ y $r$ respectivamente, de igual forma que se hizo con
      la cadena $X$ en el proceso de cifrado.

    \item Definir e inicializar $L_w$ y $R_w$ en:
      \begin{align*}
        L_w\: &=\: \sum_{j=0}^{l-1} Y_L[j] \cdot s^j \\
        R_w\: &=\: \sum_{j=0}^{r-1} Y_R[j] \cdot s^j
      \end{align*}
  \end{enumerate}
\end{frame}

%------------------------------------------------------------------------------
\begin{frame}{BPS}{Descifrador $BC^{-1}$}

  \begin{enumerate}
    \setcounter{enumi}{2}
    \item Comenzando con $i=w-1$, para cada ronda $i\: \geq\: 0$.

    Si $i$ es par:
    \begin{align*}
      L_i\: &=\: L_{i+1}\: \boxminus\:
                E_K((T_R \oplus i) \cdot 2^{f-32}\: +\: R_{i+1})\quad
                (mod\ s^l) \\
      R_i\: &=\: R_{i+1}
    \end{align*}

    Si $i$ es impar:
    \begin{align*}
      R_i\: &=\: R_{i+1}\: \boxminus\:
                E_K((T_L \oplus i) \cdot 2^{f-32}\: +\: L_{i+1})\quad
                (mod\ s^r) \\
      L_i\: &=\: L_{i+1}
    \end{align*}

  \end{enumerate}
\end{frame}

%------------------------------------------------------------------------------
\begin{frame}{BPS}{Descifrador $BC^{-1}$}

  \begin{figure}[H]
    \begin{center}
      \includegraphics[width=0.6\linewidth]
        {../../../diagramas_comunes/bps/descifrado_bc}
      \caption{Diagrama de rondas del descifrado.}
     \end{center}
  \end{figure}

\end{frame}

%------------------------------------------------------------------------------
\begin{frame}{BPS}{Descifrador $BC^{-1}$}

  \begin{enumerate}
    \setcounter{enumi}{3}
    \item Descomponer $L_0$ y $R_0$ (con el mismo proceso del cifrado) para
      obtener a $X_L$ y $X_R$, las cuales concatenadas ($X_L \parallel X_R$)
      dan la cadena de salida $X$.
  \end{enumerate}

\end{frame}

%------------------------------------------------------------------------------
